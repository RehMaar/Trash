\documentclass[a4paper,10pt]{article}
\usepackage[utf8]{inputenc}
\usepackage[english,russian]{babel}
\usepackage{fancyhdr}
\usepackage{caption}

\usepackage{listings,longtable,amsmath,amsfonts,graphicx,tikz,tabularx,amssymb}
\usetikzlibrary{automata}
\usepackage{algpseudocode}
\usepackage{pgfplots}

\captionsetup{labelsep=period}
\pagestyle{fancy}

\lstset{
    basicstyle=\footnotesize,
    breakatwhitespace=false,
    breaklines=true,
    extendedchars=true,
    keepspaces=true,
    keywordstyle=\bfseries,
    numbers=left,
    numbersep=3pt,
    numberstyle=\tiny,
    showspaces=false,
    showstringspaces=false,
    showtabs=false,
    stepnumber=1,
    stringstyle=\emph,
    tabsize=2
}


\tikzset{
    min/.style = {circle, draw=gray,fill=gray, text=white, thick}
}

\usepackage[left=1.5cm,right=1.5cm,top=2cm,bottom=1.5cm,bindingoffset=0cm]{geometry}

\captionsetup{labelsep=period}
\pagestyle{fancy}

\renewcommand{\headrulewidth}{0pt}
\fancyfoot[L] {\thepage\bf}
\fancyfoot[C] {}

\graphicspath{ {img/} }

\begin{document}
    \begin{titlepage}
        \begin{center}
            \large
            САНКТ-ПЕТЕРБУРГСКИЙ НАЦИОНАЛЬНЫЙ ИССЛЕДОВАТЕЛЬСКИЙ УНИВЕРСИТЕТ ИНФОРМАЦИОННЫХ ТЕХНОЛОГИЙ, МЕХАНИКИ И ОПТИКИ \\


            \vspace{3cm}


            Кафедра вычислительной техники
            \vspace{4cm}

            \textsc{ \textbf{Отчёт по лабораторной работе  № 2} \\
            по дисциплине <<Тестирование программного обеспечения>>\\}
            Вариант №88\\[8mm]

            \bigskip
        \end{center}
        \vspace{3cm}

        \hfill\begin{flushright}
             Студенты: \\ Куклина М. \\ Кириллова А. 
             \vfill
             Преподаватель: \\ Клименков C.В. 
        \end{flushright}
        \vfill
        \vfill
        \vfill
        \vfill
        \vfill
        \begin{center}
            Санкт-Петербург \\ 2017 г.
        \end{center}
    \end{titlepage}
\newpage

\section*{Задание}
Провести интеграционное тестирование программы, осуществляющей вычисление системы функций.
\[
    \begin{cases}
        (((sec(x) - cos(x)) ^ 3 - (tan(x) - tan(x))) \cdot sec(x) \cdot (\frac {sec(x) + tan(x) + sin(x) \cdot cos(x)} {\frac {cot(x)^2} {sec(x)}} + \frac {sin(x)} {sec(x)} \cdot cot(x)))) & \quad \text{if } x <= 0 \\
        \frac {((\frac {log_2(x)} {ln(x)} \cdot log_2(x) ^ 2) ^ 3) \cdot log_3(x)} {(log_3(x) \cdot ln(x)) ^ 2} & \quad \text{if } x > 0 \\
    \end{cases}
\]

\section*{UML диаграмма}

\section*{Тестовое покрытие}
    \subsection*{Модуль базовой функции $ln()$}
		\begin{figure}[h!]
			\caption{График функции ln(x)}
            \begin{tikzpicture}
                \begin{axis}[
                    domain=0.125:10,
                    xmin=0, xmax=7,
                    ymin=-3, ymax=3,
                    samples=500,
                    axis y line=center,
                    axis x line=middle,
					height = 5cm
                ]
                \addplot+[mark=none] {ln(x)};
                \end{axis}
            \end{tikzpicture}
		\end{figure}

		Область определения функции $(0, \infty)$. 
		\begin{enumerate}
			\item $\forall x \in (0, 1), f(x) \in (-\infty, 0)$
			\item $\text{Для } x = 1, f(x) := 0$
			\item $\text{Для } x = e, f(x) := 1$
			\item $\forall x \in (1, \infty), f(x) \in (0, \infty)$
			\item $\forall x \in (-\infty, 0), f(x) \in \varnothing$
		\end{enumerate}

    \subsection*{Модуль логарифмических функций}
		\subsubsection*{Функция $lb$}
		\begin{figure}[h!]
			\caption{График функции lb()}
            \begin{tikzpicture}
                \begin{axis}[
                    domain=0.125:10,
                    xmin=0, xmax=7,
                    ymin=-3, ymax=3,
                    samples=500,
                    axis y line=center,
                    axis x line=middle,
					height = 5cm
                ]
                \addplot+[mark=none] {ln(x)/ln(2)};
                \end{axis}
            \end{tikzpicture}
		\end{figure}

    		Функция выражена через натуральный логарифм: $lb(x) = ln(x)/ln(2)$.
    		Так как в данном модуле мы используем предположительно оттестированную функцию и математически обоснованное
    		преобразование функции, для тестирования функции двоичного логарифма достаточно оттестировать ряд значений,
    		являющихся степенью двойки.
		\subsubsection*{Функция $log_3$}

		\begin{figure}[h!]
			\caption{График функции $log_3()$}
            \begin{tikzpicture}
                \begin{axis}[
                    domain=0.125:10,
                    xmin=0, xmax=7,
                    ymin=-3, ymax=3,
                    samples=500,
                    axis y line=center,
                    axis x line=middle,
					height = 5cm
                ]
                \addplot+[mark=none] {ln(x)/ln(3)};
                \end{axis}
            \end{tikzpicture}
		\end{figure}
		Аналогично для логарифма по основанию 3.

    \subsection*{Модуль исходного логарифмического выражения}
		
		\begin{figure}[h!]
			\caption{График исходной логарифмической функции}
            \begin{tikzpicture}
                \begin{axis}[
                    domain=0.125:10,
                    xmin=0, xmax=7,
                    ymin=-200, ymax=200,
                    samples=500,
                    axis y line=center,
                    axis x line=middle,
                ]
                    \addplot+[mark=none] {ln(3)*(ln(x)^3)/(ln(2)^9)};
                \end{axis}
            \end{tikzpicture}
		\end{figure}

		Область определения функции $(0, \infty)$. 
		\begin{enumerate}
			\item $\forall x \in (0, 1), f(x) \in (-\infty, 0)$
			\item $\text{Для } x = 1, f(x) \in \varnothing$
			\item $\forall x \in (1, \infty), f(x) \in (0, \infty)$
			\item $\forall x \in (-\infty, 0), f(x) \in \varnothing$
		\end{enumerate}

    \subsection*{Модуль базовой функции $sin()$}
		\begin{figure}[h!]
			\caption{График функции $sin()$}
            \begin{tikzpicture}
                \begin{axis}[
                    domain=-10:10,
                    xmin=-10, xmax=10,
                    ymin=-1, ymax=1,
                    samples=500,
                    axis y line=center,
                    axis x line=middle,
					height = 5cm
                ]
                    \addplot+[mark=none] {sin(deg(x))};
                \end{axis}
            \end{tikzpicture}
		\end{figure}
		Классы эквивалентности.
    \subsection*{Модуль тригонометрических функции} 
		\subsubsection*{Функция $cos()$}
    		\begin{figure}[h!]
    			\caption{График функции $cos()$}
                \begin{tikzpicture}
                    \begin{axis}[
                        domain=-10:10,
                        xmin=-10, xmax=10,
                        ymin=-1, ymax=1,
                        samples=500,
                        axis y line=center,
                        axis x line=middle,
    					height = 5cm
                    ]
                        \addplot+[mark=none] {cos(deg(x))};
                    \end{axis}
                \end{tikzpicture}
    		\end{figure}
		Классы эквивалентности.
		\subsubsection*{Функция $tan()$}
    		\begin{figure}[h!]
    			\caption{График функции $tan()$}
                \begin{tikzpicture}
                    \begin{axis}[
                        domain=-2:2,
                        xmin=-2, xmax=2,
                        ymin=-10, ymax=10,
                        samples=500,
                        axis y line=center,
                        axis x line=middle,
    					height = 5cm
                    ]
                        \addplot+[mark=none] {tan(deg(x))};
                    \end{axis}
                \end{tikzpicture}
    		\end{figure}
		Классы эквивалентности.
		\subsubsection*{Функция $cot()$}
    		\begin{figure}[h!]
    			\caption{График функции $cot()$}
                \begin{tikzpicture}
                    \begin{axis}[
                        domain=-2:2,
                        xmin=-2, xmax=2,
                        ymin=-10, ymax=10,
                        samples=500,
                        axis y line=center,
                        axis x line=middle,
    					height = 5cm
                    ]
                        \addplot+[mark=none] {cot(deg(x))};
                    \end{axis}
                \end{tikzpicture}
    		\end{figure}
		Классы эквивалентности.
		\subsubsection*{Функция $sec()$}
    		\begin{figure}[h!]
    			\caption{График функции $sec()$}
                \begin{tikzpicture}
                    \begin{axis}[
                        domain=-2:2,
                        xmin=-2, xmax=2,
                        ymin=-10, ymax=10,
                        samples=500,
                        axis y line=center,
                        axis x line=middle,
    					height = 5cm
                    ]
                        \addplot+[mark=none] {sec(deg(x))};
                    \end{axis}
                \end{tikzpicture}
    		\end{figure}
		Классы эквивалентности.
    \subsection*{Модуль выражения с тригонометрическими функциями}
    	\begin{figure}[h!]
			\caption{График исходной тригонометрической функции}
			\includegraphics[scale=0.5]{./images/trigExpr.png}
		\end{figure} 
\section*{Графики, полученные в процессе интеграции}
\section*{Вывод}

\end{document}
		\end{enumerate}
