\documentclass[14pt, a4paper] {ncc}
\usepackage[utf8] {inputenc}
\usepackage[T2A]{fontenc}
\usepackage[english, russian] {babel}
\usepackage[usenames,dvipsnames]{xcolor}
\usepackage{listings,a4wide,longtable,amsmath,amsfonts,graphicx}
\usepackage{indentfirst}
\usepackage{bytefield}
\usepackage{multirow}
\usepackage{float}
\usepackage{caption}
\usepackage{subcaption}
\captionsetup{compatibility=false}
\usepackage{tabularx}
\usepackage[left=2cm,right=2cm,top=2cm,bottom=2cm,bindingoffset=0cm]{geometry}

\begin{document}
\setcounter{figure}{0}
\frenchspacing
\pagestyle{empty}
\begin{center}
                            Университет ИТМО    \\
                        Кафедра вычислительной техники

\vspace{\stretch{2}}

\end{center}
\vspace{\stretch{2}}
\begin{center}
				{\bf		Реферат \\
					по дисциплине:\\}
				{\it <<Метрология, стандартизация и сертификация>>\\}
				{\bf	на тему:\\ }
				{\it <<Правовые основы метрологии>>}
\end{center}
\vspace{\stretch{3}}
\begin{flushright}
                                    Студент:\\
                                    {\it Куклина Мария, гр. P3401}\\
                                    Преподаватель: \\
                                    {\it Муравьева-Витковская Л.А.}
\end{flushright}
\vspace{\stretch{4}}
\begin{center}
                             Санкт-Петербург, 2018
\end{center}
\newpage
\pagestyle{plain}
\setcounter{page}{2}
% toc
\tableofcontents

\newpage
\section{Введение}

       Метрология - наука об измерениях, методах и средствах обеспечения их
       единства и способах достижения требуемой точности \cite{micromake}.

       Метрологическая деятельность возникла и развивалась как деятельность
       прикладного характера, поэтому в значительной своей части она естественно
       участвует в общих рыночных отношениях, однако ее результаты должны отвечать
       особым требованиям «единства измерений»; в силу этого метрологическая
       деятельность является предметом правового регулирования, объектом воздействия
       права.

       Как известно, право - это система общеобязательных норм, на основе которых
       складываются определенные отношения - правовые отношения. Эти нормы
       устанавливаются государством и обеспечиваются его принудительной силой,
       т.е. осуществляется правовое регулирование метрологической деятельности \cite{metrob2}.
	   Именно ему посвящена данная работа. \\

	   В первом разделе приводится краткая историческая справка по историческим нормативным
	   актам, которые предшествовали рассмотренным в работе документам. Во втором
	   разделе происходит определение понятия "законодательная метрология" и обзор
	   законодательных основ метрологии и приводится перечень актов, которые
	   регулируют метрологическую деятельностью в Российской Федерации. В третьей
	   главе приводится детальное рассмотрение закона "Об обеспечении единства измерений". \\

	   Основными источниками данной работы являются федеральные законы \cite{federal},
	   технические нормативы по стандартизации и  учебники 
	   Крыловой \cite{krilova} и Клочкова \cite{micromake}. % ссылки на Крылову и законы 

\newpage
\section{История вопроса}

Метрология как область практической деятельности зародилась в
древности. Начальный этап становления метрологии характеризуется
использованием количественно неопределенных мер: частей человеческого
тела и условных единиц, связанных с физическими особенностями человека. В
Киевской Руси применялись в обиходе вершок, пядь, локоть, косая сажень.\\

Важнейшим метрологическим документом являлась «Двинская грамота» 1560
г. Ивана Грозного. В ней были регламентированы правила хранения и передачи
размера новой меры сыпучих веществ – осьмины. Ее медные экземпляры
рассылались по городам на хранение выборным людям – старостам. Образцовые
меры, с которых снимались первые копии, хранились централизованно в
приказах Московского государства, храмах и церквях. Таким образом, можно
говорить о начале создания при Иване Грозном государственной системы
обеспечения единства измерений и государственной метрологической службы.\\

Осуществление поставленной Петром I задачи «прорубить окно в Европу»,
повлекшее за собой чрезвычайное расширение культурных, научных,
производственных и торговых связей с Западом, отразилось на метрологии
как петровской, так и послепетровской эпохи.

Метрологической реформой Петра I к обращению в России были допущены
английские меры, получившие особенно широкое распространение на флоте
и в кораблестроении - футы и дюймы. Для облегчения вычислений были изданы
таблицы мер и соотношений между русскими и иностранными мерами. Начинают
выделяться некоторые метрологические центры. 

Упоминание метрологии было и в Наказе «О сборе в Московской
Большой таможне пошлин» (1698 г.): «за найденные непрямые, воровские весы
лавки опечатать, товары отобрать и семьей сослать».\\

В 1835 г. указом «О системе Российских мер и весов» были утверждены эталоны
длины и массы – платиновая сажень, равная семи английским футам, и
платиновый фунт, практически совпадающий с бронзовым золоченым фунтом 1727 г.\\

В 1875 г. представителями 17 государств (в том числе Россией) была подписана
Метрическая конвенция, которой предусматривалось изготовление
международных и национальных прототипов метра и килограмма и создание
международных метрологических учреждений.\\

14 сентября 1918 года Совнарком РСФСР издал декрет «О введении Международной
метрической системы мер и весов». Издание декрета знаменует «нормативный
этап» в развитии отечественной метрологии. С этого момента различные
установления в области метрологии вводятся нормативными актами.\\

В 1924 г. ЦИК и СНК принял постановление «О
признании заключенной в Париже 20 мая 1875 года Международной метрической
конвенции для обеспечения международного единства и усовершенствования
метрической системы, имеющей силу для СССР». В этом же году при Совете
Труда и Обороны создается комитет по стандартизации. Стандартизация
становится нормативно-правовой основой метрологической деятельности.\\

В 1960г. 11 Генеральная конференция по мерам и весам приняла новую систему
единиц, присвоив ей наименование «Международная система единиц». С 1981
г. постановлением Государственного комитета СССР по стандартам (ГОСТ 8.417-81)
в СССР установлено применение Международной системы единиц (СИ). В 1973 году
утверждена Государственная система обеспечения единства измерений
(ГСИ), регламентирующая все стороны метрологической деятельности по
обеспечению единства измерений в стране (с 2000 г. ГОСТ Р 8.000-00 ГСИ).\\

В 1993 году принят закон РФ «Об обеспечении единства измерений» и установлена
гражданско-правовая, административная, уголовная ответственность за
нарушение правовых норм и обязательных требований стандартов в области
единства измерений и метрологического обеспечения. В настоящее время
действует новая редакция этого закона от 2008 \cite{hist}.

\newpage
\section{Закондательная метрология}
     {\bf Закондательная метрология} -- раздел метрологии, предметом которого
     является установление обязательных технических и юридических требований по
     применению единиц физических величин, эталонов, методов и средств измерений,
     направленных на обеспечение единства и необходимости точности измерений в
     интересах общества\cite{rmg29}. \\

    Основополагающим этапом развития законодательной метрологии в Российской
    Федерации можно считать 1993 год, когда был принят Закон «Об обеспечении единства
    измерений», который впервые на высшем уровне установил основные нормы и
    правила управления метрологической деятельностью в стране.\\

    Головным институтом в системе Федерального агентства
    по техническому регулированию и метрологии России (Ростехрегулирование), является ВНИИМС --
    институт осуществляет исследования и разработки по правовым и методическим
    проблемам обеспечения единства измерений и деятельности метрологической
    службы России, выполняет функции информационного центра Ростехрегулирования России
    в области метрологии, участвует в международном сотрудничестве в области
    законодательной метрологии. Они проводят исследования и разработки в сфере
	государственного управления (регулирования) метрологической деятельностью в России;
    Исследования по совершенствованию деятельности ГМС и развитию метрологической инфраструктуры.\cite{wiki}

    \subsection{Законодательные акты}

    Значимость и ответственность измерений и измерительной информации
    обусловливают необходимость установления в законодательном порядке
    комплекса правовых и нормативных актов и положений. Законодательная база
    метрологии включает следующие основные документы.

	\begin{enumerate}
    	\item Закон РФ "Об обеспечении единства измерений"
    	\item РМГ 29-99. "Государственная система обеспечения единства измерений. Метрология. Основные термины и определения."
    	\item МИ 2247-93 ГСИ." Метрология. Основные термины и определения."
    	\item ГОСТ 8.417-2002 "ГСИ. Единицы физических величин."
    	\item ПР 50.2.006-94 "ГСИ. Поверка средств измерений. Организация и порядок проведения."
    	\item ПР 50.2.009-94 "ГСИ. Порядок проведения испытаний и утверждения типа средств измерения "
    	\item ПР 50.2.014-94 "ГСИ. Аккредитация метрологических служб юридических лиц на право поверки средств измерений."
    	\item МИ 2277-94 "ГСИ. Система сертификации средств измерений. Основные положения и порядок проведения работ."
    	\item ПР 50.2.002-94 "ГСИ. Порядок осуществления государственного метрологического надзора за выпуском, состоянием и применением средств измерений, аттестованными методиками выполнения измерений, эталонами и соблюдением метрологических правил и норм." \cite{micromake}
	\end{enumerate}

    Вся метрологическая деятельность в Российской Федерации основывается на
    конституционной норме (глава 3, статья 71, пункт p)\cite{const}, которая устанавливает, что в федеральном ведении находятся
    стандарты, эталоны, метрическая система и исчисление времени, и закрепляет
    централизованное руководство основными вопросами законодательной метрологии,
    такими, как эталоны и связанные с ними другие метрологические основы. В
    развитие этой конституционной нормы приняты законы "Об обеспечении единства
    измерений" и "О техническом регулировании", детализирующие основы метрологической
    деятельности. \cite{metrob1}


    Российская Федерация наделяет специальными функциями по обеспечению
    действия стандартов на своей территории федеральные органы исполнительной
    власти. Например, Правительство РФ определяет орган, уполномоченный на
    исполнение функций национального органа по стандартизации. Таковым является
    Федеральное агентство по техническому регулированию и метрологии,
	которое утверждает национальные стандарты,
    осуществляет межрегиональную и межотраслевую координацию деятельности
    в области обеспечения единства измерений, координацию проведения работ
    по аккредитации организаций, осуществляющих деятельность по оценке
    соответствия, и координацию деятельности по развитию системы кодирования
    технико-экономической и социальной информации, представляет Россию по
    вопросам стандартизации на международном уровне, а также выполняет ряд
    иных функций \cite{const}.


    В сферах, которые напрямую не контролируются государственными органами,
    действует Российская система калибровки, также направленная на обеспечение
    единства измерений. Система калибровки - совокупность субъектов деятельности и
    калибровочных работ, направленных на обеспечение единства измерений в сферах, не
    подлежащих государственному метрологическому контролю и надзору и действующих
    на основе установленных требований к организации и проведению калибровочных
    работ. Закон обеспечивает взаимодействие с международной и национальными
    системами измерений. Это позволяет достигнуть взаимного признания
    результатов испытаний, калибровки и сертификации, а также использовать
    мировой опыт и тенденции развития современной метрологии. Существуют и
    другие законодательные, акты и документы по стандартизации, относящиеся к
    законодательной базе метрологии. \cite{micromake}

\newpage
\section{Закон "Об обеспечении единства измерений" от 2008 года}

    Цели Закона состоят в следующем:
	\begin{itemize}
		\item установление правовых основ обеспечения единства измерений в РФ;
		\item защита прав и законных интересов граждан, общества и государства от
			  отрицательных последствий недостоверных результатов измерений;
		\item обеспечение потребности граждан, общества и государства в получении
			  объективных, достоверных и сопоставимых результатов измерений,
			 используемых в целях защиты жизни и здоровья граждан, охраны
			 окружающей среды, животного и растительного мира, обеспечения
			 обороны и безопасности государства, в том числе экономической безопасности;
		\item  содействие развитие экономики РФ и научно-техническому прогрессу.
	\end{itemize}

	Закон регулирует отношения, возникающие при выполнении измерений,
	установлении и соблюдении требований к измерениям, единицам величин,
	эталонам единиц величин, стандартным образцам, средствам измерений,
	применении стандартных образцов, средств измерений, методик измерений, а также
	при осуществлении деятельности по обеспечению единства измерений,
	предусмотренной законодательством РФ об обеспечении единства измерений,
	в том числе при выполнении работ и оказании услуг по обеспечению единства
	измерений. \cite{federal}

	\subsubsection{Основные положения}

	Основные сферы его приложения -- торговля, здравоохранение, защита
	окружающей среды, внешнеэкономическая деятельность. Однако Закон распространяется и
	на некоторые области производства в части калибровки средств измерений
	метрологическими службами юридических лиц с использованием эталонов,
	соподчиненных государственным эталонам единиц величин. Закон предоставляет
	право аккредитованным метрологическим службам юридических лиц выдавать
	сертификаты о калибровке от имени органов и организаций, которые их
	аккредитовали.

    Закон "Об обеспечении единства измерений" устанавливает и законодательно
    закрепляет основные понятия, принимаемые для целей Закона: единство
    измерений, средство измерений, эталон единицы величины, государственный
    эталон единицы величины, нормативные документы по обеспечению единства
    измерений, метрологическая служба, метрологический контроль и надзор,
    поверка и калибровка средств измерений, сертификат об утверждении типа
    средств измерений, аккредитация на право поверки средств измерений,
    сертификат о калибровке. В основу определений положена официальная
    терминология Международной организации законодательной метрологии
    (МОЗМ). \cite{krilova}
    Основные статьи Закона устанавливают:
	\begin{itemize}
        \item организационную структуру государственного управления обеспечением единства измерений;
        \item нормативные документы по обеспечению единства измерений;
        \item единицы величин и государственные эталоны единиц величин;
		\item требования к измерениям, единицам величин, эталонам единиц величин, стандартным образцам, средствам измерений;
		\item правила аккредитации в области обеспечения единства измерений;
        \item средства и методики измерений.\cite{federal}
	\end{itemize}

	\subsubsection{Формы государственного регулирования}

	Государственное регулирование в области обеспечения единства
    измерений осуществляется в следующих формах:
	\begin{itemize}
		\item утверждение типа стандартных образцов или типа средств измерений;
		\item проверка средств измерений;
		\item метрологическая экспертиза;
		\item государственный метрологический надзор;
		\item аттестация методик измерений;
		\item аккредитация юридических лиц и индивдуальных предпренимателей
			  на выполнение работ и оказание услуг в бласти обеспечения единства измерений. \cite{federal}
	\end{itemize}


	\subsubsection{Юридическая ответственность}

		Согласно главе 8 Закона, ответственность рассматривается отдельно
		для должностных и юридических лиц.

        Юридические лица, их руководители и работники, индивидуальные предприниматели,
        допустившие нарушения законодательства Российской Федерации об
        обеспечении единства измерений, необоснованно препятствующие
        осуществлению государственного метрологического надзора и
        (или) не исполняющие в установленный срок предписаний федеральных органов
        исполнительной власти, осуществляющих государственный метрологический
        надзор, об устранении выявленных нарушений, несут ответственность в
        соответствии с законодательством Российской Федерации. 
	 
        За нарушения законодательства Российской Федерации об обеспечении
        единства измерений должностные лица федеральных органов исполнительной
        власти, осуществляющих функции по выработке государственной политики
        и нормативно - правовому регулированию , оказанию государственных
        услуг, управлению государственным имуществом в области обеспечения
        единства измерений, а также федеральных органов исполнительной
        власти, осуществляющих государственный метрологический надзор, и
        подведомственных им организаций несут ответственность в соответствии
        с законодательством Российской Федерации. \cite{federal}

\newpage
\section{Заключение}

	В Российской Федерации метрологическая деятельность основывается на ряде
	правовых актов и нормативов. Главым законодательным актом является
	закон "Об обеспечении единства измерений", который устанавливает 
	основные нормы и правила управления метрологической деятельностью.\\

	Несмотря на кажущуюся полноценность, закон подвергся критике специалистов
	в области метрологии. В работе профессора НИТУ "МИСиС" Богомолова Ю.А. \cite{bogomol}
	говорится о том, что Федеральный закон не в полной мере отвечает принципам 
	Системы государства и права и положениям Конституции Российской Федерации.
	К закону, в частности, предявляются упрёки из-за внесённых изменений
    и отсутствия основполагающих тезисов:
	\begin{itemize}
		\item в целях закона отсутствует "обеспечение единства измерений";
		\item государственное управление закона не предполагает системного подхода
		      в управлении;
		\item ранее организованное "государственное управление деятельностью по обеспечению единства измерений" на
              основе единых общеобязательных норм (государственных требований) заменено "регулированием" на
              основе измерений по перечням Федеральных органов исполнительной власти (т.е. по отраслевым нормам);
		\item резко сокращён объём проверяемых систем измерений;
		\item и т.д. 
	\end{itemize} 

	Помимо этого, специалисты из многих областей (начиная от геологической отрасли и заканчивая областями
    использования атомной энергии) говорят о том, что в закон не были включены те или иные
    положения, которые для данной отрасли являются существенными.

	Таким образом, можно заключить, что в данной области стоит необходимость пересмотра и дополнения
	существующих нормативных актов, чтобы не допустить ошибок измерений и двойственность трактования законов,
 	которые могут привести к катастрофическим последствиям.
 
\newpage
\bibliographystyle{utf8gost705u}  %% стилевой файл для оформления по ГОСТу
\bibliography{bib}     %% имя библиографической базы (bib-файла) 

% metrob1 http://metrob.ru/html/metrology/NPosnovi.html
% metrob2 http://metrob.ru/html/zak/regulir.html
% [krilova] Стандартизация, сертификация и метрология  Г.Д. Крылова
% hist https://superinf.ru/view_helpstud.php?id=192

\end{document}
