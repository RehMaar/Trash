\documentclass[12pt, a4paper] {ncc}
\usepackage[utf8] {inputenc}
\usepackage[T2A]{fontenc}
\usepackage[english, russian] {babel}
\usepackage[usenames,dvipsnames]{xcolor}
\usepackage{listings,a4wide,longtable,amsmath,amsfonts,graphicx,tikz}
\usepackage{indentfirst}
\usepackage{bytefield}
\usepackage{multirow}
\usepackage{float}
\usepackage{caption}
\usepackage{subcaption}
\captionsetup{compatibility=false}
\usepackage{tabularx}

\usepackage[left=2cm,right=2cm,top=2cm,bottom=2cm,bindingoffset=0cm]{geometry}

\begin{document}
\setcounter{figure}{0}
\frenchspacing
\pagestyle{empty}
\begin{center}
							Университет ИТМО	\\
                        Кафедра вычислительной техники

\vspace{\stretch{2}}
                    Методы цифровой обработки сигналов
\end{center}
\vspace{\stretch{2}}
\begin{center}
                            Лабораторная работа №1\\
						{\bf Исследование метода подавления случайного шума
							 путём когерентного накопления сигнала}
\end{center}
\vspace{\stretch{3}}
\begin{flushright}
                                    Студент:\\
                                    {\it Куклина Мария, P3401}\\
									Преподаватель: \\
									{\it Тропченко А.А.}
\end{flushright}
\vspace{\stretch{4}}
\begin{center}
                             Санкт-Петербург, 2017
\end{center}
\newpage


\section{Цели работы}
    Определение возможностей метода когерентного накопления для случаев стационарного
    и квазистационарного сигнала.
\section{Задание}

    \begin{description}
        \item[Вид сигнала:] гармонический. 
        \item[Соотношение сигнал/шум:] $0.1$.
        \item[Число циклов накопления:] до $500$.
        \item[Пределы изменения соотношения сиигнал/шум:] $0.1-2$.
    \end{description}

\section{Для стационарного сигнала}
    
    \subsection{Зависимость SNR от числа накоплений}
        \begin{table}[H]
            \centering
            \begin{tabular} { |c|c| }
                \hline
                \textbf{M} & \textbf{SNR} \\ \hline
                    10     &  0.8805 \\ \hline
                    25     &  1.507 \\ \hline
                    50     &  2.046 \\ \hline
                    75     &  2.1812 \\ \hline
                    100    &  2.4077 \\ \hline
                    125    &  2.9294 \\ \hline
                    150    &  3.1868 \\ \hline
                    200    &  3.7074 \\ \hline
                    250    &  4.2957 \\ \hline
                    300    &  4.5854 \\ \hline
                    350    &  5.4545 \\ \hline
                    400    &  6.0561 \\ \hline
                    450    &  5.9225 \\ \hline
                    500    &  6.5576 \\ \hline
            \end{tabular}
            \caption{Отношение сигнал/шум в выходной смеси от длительности накопления}
        \end{table}

        \begin{figure}[H]
            \centering
            \includegraphics[scale=0.9,page=1]{stat_by_m.pdf}
            \caption{Зависимость SNR от числа циклов накоплений.}
        \end{figure}

    \subsection{Зависимость $\text{SNR}_{\text{out}}$ от $\text{SNR}_{\text{in}}$}
        \begin{table}[H]
            \begin{tabular} { |c|c| }
                \hline
					$\text{SNR}_\text{in}$ & $\text{SNR}_\text{out}$ \\ \hline
                    0.1  &   0.9298	\\ \hline
                    0.2  &   1.7689	\\ \hline
                    0.3  &   2.3864	\\ \hline
                    0.4  &   3.0902	\\ \hline
                    0.5  &   3.6152	\\ \hline
                    0.6  &   5.6947	\\ \hline
                    0.7  &   5.8467	\\ \hline
                    0.8  &   6.6303	\\ \hline
                    0.9  &   7.8079	\\ \hline
                    1    &   8.7068	\\ \hline
                    1.1  &   9.6044	\\ \hline
                    1.2  &   9.7766	\\ \hline
                    1.3  &   11.3096	\\ \hline
                    1.4  &   12.0391	\\ \hline
                    1.5  &   12.8742	\\ \hline
                    1.6  &   13.6251	\\ \hline
                    1.7  &   14.7871	\\ \hline
                    1.8  &   15.1528	\\ \hline
                    1.9  &   15.8292	\\ \hline
                    2    &   16.2992	\\ \hline
            \end{tabular}
            \begin{tabular} { |c|c| }
                \hline
					$\text{SNR}_\text{in}$ & $\text{SNR}_\text{out}$ \\ \hline
                        0.1   &  1.2707	\\ \hline
                        0.2   &  2.5998	\\ \hline
                        0.3   &  4.3951	\\ \hline
                        0.4   &  5.5769	\\ \hline
                        0.5   &  6.8013	\\ \hline
                        0.6   &  7.8619	\\ \hline
                        0.7   &  8.9443	\\ \hline
                        0.8   &  10.7575	\\ \hline
                        0.9   &  12.8911	\\ \hline
                        1     &  13.1404	\\ \hline
                        1.1   &  14.0451	\\ \hline
                        1.2   &  16.08	\\ \hline
                        1.3   &  18.1254	\\ \hline
                        1.4   &  18.5445	\\ \hline
                        1.5   &  21.1901	\\ \hline
                        1.6   &  21.5	\\ \hline
                        1.7   &  22.4446	\\ \hline
                        1.8   &  24.0947	\\ \hline
                        1.9   &  25.0542	\\ \hline
                        2     &  25.6024	\\ \hline
            \end{tabular}
            \begin{tabular} { |c|c| }
                \hline
					$\text{SNR}_\text{in}$ & $\text{SNR}_\text{out}$ \\ \hline
                        0.1  &   1.9678 \\ \hline
                        0.2  &   3.737 \\ \hline
                        0.3  &   5.959 \\ \hline
                        0.4  &   8.4162 \\ \hline
                        0.5  &   9.9779 \\ \hline
                        0.6  &   12.0321 \\ \hline
                        0.7  &   14.0263 \\ \hline
                        0.8  &   15.6813 \\ \hline
                        0.9  &   17.1535 \\ \hline
                        1    &   17.435 \\ \hline
                        1.1  &   21.8457 \\ \hline
                        1.2  &   21.9823 \\ \hline
                        1.3  &   24.8981 \\ \hline
                        1.4  &   24.5258 \\ \hline
                        1.5  &   26.4063 \\ \hline
                        1.6  &   28.4651 \\ \hline
                        1.7  &   30.8562 \\ \hline
                        1.8  &   32.0492 \\ \hline
                        1.9  &   32.2567 \\ \hline
                        2    &   33.5303 \\ \hline
            \end{tabular}
            \caption{Отношение сигнал/шум выхода от сигнал/шум на входе для фиксированного числа выборок (M = 10, 25, 50 соответственно)}
        \end{table}

        \begin{figure}[H]
            \centering
            \includegraphics[scale=0.9,page=1]{stat_by_snr.pdf}
            \caption{Зависимость SNR от числа циклов накоплений.}
        \end{figure}

\section{Для квазистационарного сигнала}
    \subsection{Зависимость SNR от числа накоплений}
        \begin{table}[H]
            \centering
            \begin{tabular} { |c|c| }
                \hline
                \textbf{M} & \textbf{SNR} \\ \hline
                    1   &   0.7797  \\ \hline
                    2   &   1.1275  \\ \hline
                    3   &   1.3821  \\ \hline
                    4   &   1.5945  \\ \hline
                    5   &   1.6124  \\ \hline
                    6   &   1.7063  \\ \hline
                    7   &   1.5947  \\ \hline
                    8   &   1.7035  \\ \hline
                    9   &   1.6614  \\ \hline
                    10  &   1.6728  \\ \hline
                    15  &   1.3546  \\ \hline
                    20  &   1.1629  \\ \hline
                    25  &   0.9798  \\ \hline
                    50  &   1.0088  \\ \hline
                    100 &   0.9968  \\ \hline
                    150 &   1.0063  \\ \hline
                    200 &   1.002   \\ \hline
                    250 &   0.998   \\ \hline
                    300 &   1.0092  \\ \hline
                    350 &   1.0017  \\ \hline
                    400 &   0.9916  \\ \hline
                    450 &   1.0034  \\ \hline
                    500 &   0.9981  \\ \hline
            \end{tabular}
            \caption{Отношение сигнал/шум в выходной смеси от длительности накопления}
        \end{table}

        \begin{figure}[H]
            \centering
            \includegraphics[scale=0.9,page=1]{nonstat_by_m.pdf}
            \caption{Зависимость SNR от числа циклов накоплений.}
        \end{figure}

    \subsection{Зависимость $\text{SNR}_{\text{out}}$ от $\text{SNR}_{\text{in}}$}
        \begin{table}[H]
            \begin{tabular} { |c|c| }
                \hline
					$\text{SNR}_\text{in}$ & $\text{SNR}_\text{out}$ \\ \hline
					0.1 &  1.1005 \\ \hline
					0.2 &  1.6626 \\ \hline
					0.3 &  1.718 \\ \hline
					0.4 &  1.9245 \\ \hline
					0.5 &  2.0495 \\ \hline
					0.6 &  2.2551 \\ \hline
					0.7 &  2.2947 \\ \hline
					0.8 &  2.3052 \\ \hline
					0.9 &  2.3028 \\ \hline
					1   &  2.3481 \\ \hline
					1.1 &  2.4099 \\ \hline
					1.2 &  2.341 \\ \hline
					1.3 &  2.4444 \\ \hline
					1.4 &  2.4108 \\ \hline
					1.5 &  2.4986 \\ \hline
					1.6 &  2.4083 \\ \hline
					1.7 &  2.4752 \\ \hline
					1.8 &  2.4575 \\ \hline
					1.9 &  2.5077 \\ \hline
					2   &  2.495 \\ \hline
			\end{tabular}
            \begin{tabular} { |c|c| }
                \hline
					$\text{SNR}_\text{in}$ & $\text{SNR}_\text{out}$ \\ \hline
						0.1 &  0.9702 \\ \hline
						0.2 &  0.9777 \\ \hline
						0.3 &  1.0252 \\ \hline
						0.4 &  1.0131 \\ \hline
						0.5 &  1.0344 \\ \hline
						0.6 &  1.0203 \\ \hline
						0.7 &  1.0157 \\ \hline
						0.8 &  1.0262 \\ \hline
						0.9 &  1.0324 \\ \hline
						1   &  1.0228 \\ \hline
						1.1 &  1.018 \\ \hline
						1.2 &  1.0181 \\ \hline
						1.3 &  1.0198 \\ \hline
						1.4 &  1.0186 \\ \hline
						1.5 &  1.0169 \\ \hline
						1.6 &  1.0254 \\ \hline
						1.7 &  1.0206 \\ \hline
						1.8 &  1.0181 \\ \hline
						1.9 &  1.0208 \\ \hline
						2   &  1.0202 \\ \hline
			\end{tabular}
            \begin{tabular} { |c|c| }
                \hline
					$\text{SNR}_\text{in}$ & $\text{SNR}_\text{out}$ \\ \hline
                            0.1 &  1.0202 \\ \hline
                            0.2 &  0.9996 \\ \hline
                            0.3 &  0.9823 \\ \hline
                            0.4 &  0.9947 \\ \hline
                            0.5 &  0.9909 \\ \hline
                            0.6 &  1.0005 \\ \hline
                            0.7 &  0.9948 \\ \hline
                            0.8 &  1.0087 \\ \hline
                            0.9 &  1.0024 \\ \hline
                            1   &  1.0007 \\ \hline
                            1.1 &  1.0022 \\ \hline
                            1.2 &  1.0017 \\ \hline
                            1.3 &  1.0015 \\ \hline
                            1.4 &  0.9969 \\ \hline
                            1.5 &  0.9993 \\ \hline
                            1.6 &  1.002 \\ \hline
                            1.7 &  0.9993 \\ \hline
                            1.8 &  1.0026 \\ \hline
                            1.9 &  0.9979 \\ \hline
                            2   &  1.0005 \\ \hline
			\end{tabular}
            \caption{Отношение сигнал/шум выхода от сигнал/шум на входе для фиксированного числа выборок (M = 10, 25, 50 соответственно)}
		\end{table}

        \begin{figure}[H]
            \centering
            \includegraphics[scale=0.9,page=1]{nonstat_by_snr.pdf}
            \caption{Зависимость SNR от числа циклов накоплений.}
        \end{figure}

\section*{Функциональная схема устройства}
        \begin{figure}[H]
            \centering
            \includegraphics[scale=0.9,page=1]{ffbd.pdf}
        \end{figure}

\section*{Вывод}

В результате выполнения лабораторной работы были сделаны следующие выводы.
\begin{itemize}
	\item Для стационарного сигнала:
		 \begin{itemize}
		  	\item Наблюдается, что зависимость выходного значения SNR (сигнал / шум) от
        		  числа накполений практически линейная. При большем числе накоплений
        		  SNR имеет большее значение.
			\item При фиксации значений накполения и вариации входного SNR наблюдается, что выходный SNR
				  тем больше, чем больше входноное значение, причём значения меньшие единицы (что означет
				  преобладание шума в сигнале) проявляются только при небольших значениях выборки.
		\end{itemize}
		Таким образом, при применении метода накопления для гармонического сигнала даёт ожидаемые
		результаты. 
	\item  Для квазистационарного сигнала:
		\begin{itemize}
			\item Для зависимости входного SNR от числа накоплений наблюдается, что
				  метод даёт результаты при значениях числа накоплений до 20. Далее
				 SNR имеет значения около единицы. 		
			\item Для небольшого фиксированного значения числа накоплений
				  выходное значение SNR незначительно растёт (в нашем случае до 2.5)
				  и после не меняется. Для оставшихся значений числа накоплений для 
				  всех входных значений SNR выход держится около единцы.
		\end{itemize}
    Таким образом для квазистационарного сигнала метод накполений применять не целесообразно,
	так как не наблюдается ожидаемый эффект.
\end{itemize}

\end{document}
