\documentclass[14pt, a4paper] {ncc}
\usepackage[utf8] {inputenc}
\usepackage[T2A]{fontenc}
\usepackage[english, russian] {babel}
\usepackage[usenames,dvipsnames]{xcolor}
\usepackage{listings,a4wide,longtable,amsmath,amsfonts,graphicx}
\usepackage{indentfirst}
\usepackage{bytefield}
\usepackage{multirow}
\usepackage{float}
\usepackage{caption}
\usepackage{subcaption}
\captionsetup{compatibility=false}
\usepackage{tabularx}
\usepackage{cite}
\usepackage[left=2cm,right=2cm,top=2cm,bottom=2cm,bindingoffset=0cm]{geometry}

\begin{document}

\frenchspacing
\pagestyle{empty}
\begin{center}
                            {\bf Университет ИТМО    \\}
\vspace{\stretch{2}}

\end{center}
\vspace{\stretch{2}}
\begin{center}

	%{\bf
			Эссе по дисциплине:\\ {\bf <<Имидж специалиста и корпоративная культура>> \\}
			на тему: \\
	%}
	{\it <<Стереотипы и предрассудки в межкультурной коммуникации>> \\ }
\end{center}
\vspace{\stretch{3}}
\begin{flushright}
                                    Студент:\\
                                    {\it Куклина М.Д., P3401}\\
                                    Преподаватель: \\
                                    {\it Копонева Е.М.}
\end{flushright}
\vspace{\stretch{4}}
\begin{center}
                             Санкт-Петербург, 2018
\end{center}
\newpage

% toc
\tableofcontents

\newpage
\section{Введение}
% Введение. Почему выбрала тему

	Межкультурные, международные связи играли и играют колоссальную роль в
	развитии истории и в развитии общества. Это остаётся актуальным и в нашем времени.

	Межкультурную коммуникацию следует рассматривать как совокупность разнообразных
	форм отношений и общения между индивидами и группами, принадлежащими к
	разным культурам. Большинство специалистов считает, что говорить о
	межкультурной коммуникации можно лишь в том случае, если люди
	представляют разные культуры и осознаю всё, не принадлежаещее к их
	культуры, как чужое. Отношения являются межкультурными, если
	их участники не прибегают к собственным традициям, обычаям,
	представлениям и способам поведения, а знакомятся с чужими
	правилами и нормами повседневного общения. \cite{Grush}
	Трудности межкультурного общения связаны с разницей в ожидании и
	предубеждениях, свойсвенных людям и отличающихся в разных культурах.
	Поэтому важно изучать стререотипы и предрассудки в межкультурных
	коммуникациях, так как они являются основой для начала знакомства
	с другими культурами.

\section{Стереотипы в межкультурной коммуникации}

	\subsection{Определение}

        В процессе интерпретации поведения носителя чужой культуры многое объясняется
        стереотипными представлениями каждой из сторон. Стереотипы — это устойчивые,
        обобщенные представления об образе жизни, обычаях, нравах, привычках, т.е. о
        системе этнокультурных свойств того или иного народа. Они основываются
        на многократно повторяющихся однообразных жизненных ситуациях, которые
        закрепляются в сознании человека в виде стандартных схем и моделей. Тот
        же механизм работает и при восприятии различных категорий людей, когда
        преувеличиваются сходные качества между ними и игнорируются различия. Иными
        словами, стереотипы помогают человеку дифференцировать и упрощать окружающий
        мир, навести в нем порядок.

        Реальным носителем стереотипов является группа, и поэтому именно в опыте группы
        следует искать корни стереотипа. Стереотипы отражают общественный опыт людей,
        общее и повторяющееся в их повседневной практике. Они формируются в результате
        совместной деятельности путем акцентирования сознания человека на тех или
        иных свойствах, качествах явлений окружающего мира, которые хорошо известны,
        видны или понятны по крайней мере большому числу людей. По своему содержанию
        стереотипы представляют собой концентрированное выражение этих свойств и
        качеств, наиболее схематично и понятно передающих их сущность.\cite{Sadoh}

	\subsection{Формирование стереотипов}
        Психологический механизм возникновения стереотипов основывается на
        принципе экономии усилий, свойственном для повседневного человеческого
        мышления. Данный принцип означает, что люди не стремятся реагировать на
        окружающие их явления каждый раз по-новому, а подводят их под имеющиеся
        у них категории. Постоянно меняющийся мир просто перегружает человека
        новой информацией и психологически вынуждает его классифицировать эту
        информацию в наиболее удобные и привычные модели, которые и получили название
        стереотипов. Реальным носителем стереотипов является группа, и поэтому именно
        в опыте группы следует искать корни стереотипа. Наиболее известны этнические
        стереотипы — устойчивые суждения о представителях одних национальных групп
        с точки зрения других. Например, стереотипные представления о вежливости и
        худобе англичан, об эксцентричности итальянцев, легкомысленности французов
        или «загадочной славянской душе». Стереотипы часто эмоционально окрашены
        симпатиями и антипатиями, в зависимости от которых, одно и то же поведение
        получает разную оценку. Те черты, которые у своего народа рассматриваются как
        проявление ума, у другого народа считается проявлением хитрости. Несмотря на
        обоснованность или необоснованность, истинность или ложность стереотипов,
        все они являются неотъемлемым элементом любой культуры и уже самим фактом
        своего существования оказывают воздействие на психологию и поведение людей,
        влияют на их сознание и межнациональные контакты.

		\begin{itemize}
    		\item Поскольку стереотипы представляют собой часть культуры, то
    		«привычку» думать о других группах определенным образом мы «всасываем
    		с молоком матери». Причем, это касается не только того, как выглядят
    		чужие группы в наших глазах, но и наших представлений о том, как выглядит
    		собственная группа в представлениях чужих групп.

			\item Во-вторых, стереотипы главным образом приобретаются в
			процессе общения с теми людьми, с которыми чаще всего приходится
			сталкиваться. Это — родители, друзья, сверстники, учителя и
			т.д. Если, например, дети слышат, когда их родители говорят, что
			«русские слишком простодушны» или что «цыганам в глаза смотреть
			нельзя — обманут», то они воспринимают стереотипы.

			\item В-третьих, стереотипы могут возникать через ограниченные личные
			контакты. Например, если нас на рынке обманул азербайджанский
			торговец, то мы можем заключить, что все азербайджанцы лживы. В этом
			случае мы приобретаем стереотип, исходя из ограниченной информации.

			\item И в-четвертых, особое место в образовании стереотипов
			занимают средства массовой информации. Возможности формирования
			стереотипов средствами массовой информации не ограничены как по
			своему масштабу, так и по своей силе. Для большинства людей пресса,
			радио и телевидение являются авторитетами. Мнение средств массовой
			коммуникации становится мнением людей, вытесняя из мышления их
			индивидуальные установки.\cite{Sadoh2}
		\end{itemize}

	\subsection{Роль стереотипов в межкультурной коммуникации}

    Стереотипы обладают целым рядом качеств: целостностью, ценностной окраской,
    устойчивостью, консерватизмом, эмоциональностью, рациональностью и
    др. Благодаря этим качествам стереотипы выполняют свои разнообразные
    функции. Из последних для процесса межкультурной коммуникации особое
    значение имеют следующие функции стереотипов:

	\begin{itemize}
    	\item передача относительно достоверной информации;
    	\item ориентирующая функция;
    	\item влияние на создание реальности.
	\end{itemize}
    Функция передачи относительно достоверной информации основана на процессах
    «глобального» обобщения, происходящих при наблюдении неординарного,
    бросающегося в глаза, необычного поведения и образа мыслей членов другой
    культурной группы. Попадая в чужую культуру, люди склонны к обобщению и
    упорядочиванию всего, что они видят. Уже с первых контактов с чужой культурой
    всегда начинается классификация новой информации и формируется относительно
    четкая модель этой культуры. Это достигается, как правило, путем упрощения
    и генерализации реальности, выделения наиболее характерных черт данной
    культуры. Например, в основе стереотипов практичности и пунктуальности немцев
    или гостеприимства и склонности к выпивке у русских во многих случаях лежат
    наблюдения за их действительным поведением.

    Ориентирующая функция заключается в том, что с помощью стереотипизации
    удается создать упрощенную матрицу окружающего мира, в ячейки которой,
    опираясь на стереотипы, «расставляются» определенные социальные группы. Такой
    прием позволяет довольно быстро дифференцировать людей по группам на основе
    стереотипных признаков, ожидая от них определенного поведения. Например, если
    спросить у любого человека, для членов какой этнической группы характерно
    предсказывать судьбу по линиям руки, то, скорее всего, у него возникнет образ
    цыганки. И, наоборот, увидев на улице цыганку, которая заговаривает с прохожими,
    довольно точно можно предположить, что она предлагает услуги гадалки. Если
    такие же действия будут исходить, например, от женщины, ничем не отличающейся
    по внешнему облику от членов вашей культурной группы, то такого предположения
    не возникнет.

    Функция влияния на создание реальности заключается в том, что с
    помощью стереотипов удается четко разграничить свою и чужую этнические
    группы. Стереотипизация позволяет дать оценочное сравнение чужой и своей
    групп и тем самым защитить традиции, взгляды, ценности своей группы. В связи
    с этим стереотипы являются своего рода защитным механизмом, служащим для
    сохранения позитивной идентичности собственной культурной группы. Примером
    этого могут служить стереотипы-поговорки или анекдоты, существующие в каждой
    культуре. «Незваный гость — хуже татарина», «Что русскому хорошо, то немцу —
    смерть». В этих поговорках можно увидеть вполне конкретный образ другой группы.\cite{Sadoh2}

\section{Предрассудки в межкультурной коммуникации}

	\subsection{Определение}
Психология рассматривает предрассудок как психологическую установку
предвзятого и враждебного отношения к чему-либо без достаточных на то оснований
или причин.

Главным фактором в возникновении предрассудков выступает неравенство в
социальных, экономических и культурных условиях жизни различных этнических
общностей. Этот же фактор определяет и такую распространенную форму
предрассудка, как ксенофобия - неприязнь ко всем иностранцам.

Предрассудки возникают как следствие неполного или искаженного понимания
объекта Возникая на основе ассоциации, воображения или предположения,
психологическая установка с искаженным информационным компонентом оказывает
тем не менее стойкое влияние на отношение людей к объекту.

В процессе межкультурной коммуникации следует различать стереотип и
предрассудок. Стереотип, как обобщенный, собирательный образ группы или ее
представителя, как правило, без ярко выраженных эмоциональных оценок, содержит
возможность позитивных суждений.

Для предрассудков характерно бездумное негативное отношение ко всем членам
группы или большей ее части. В практике человеческого общения объектом
предрассудков обычно становятся люди, резко отличающиеся от большинства
какими-либо чертами, отрицательно оцениваемыми другими людьми. Известные
формы предрассудков - это, например, расизм, сексизм, гомофобия,
дискриминация по возрастному признаку, полу.

Предрассудки, как и стереотипы, являются элементами культуры, поскольку
порождены общественными, а не биологическими причинами. Однако предрассудки
представляют собой устойчивые и широко распространенные элементы обыденной,
повседневной культуры, которые передаются их носителями из поколения в
поколение и сохраняются при помощи обычаев или нормативных актов. Чаще всего
предрассудки включены в культуру в виде нормативных заповедей, т.е. строгих
представлений - 'что и как должно быть', например, как следует относиться
к представителям тех или иных этнических или социокультурных групп.\cite{Sadoh}


    \subsection{Формирование предрассудков}
Механизм формирования предрассудков позволяет людям эмоционально реагировать
на человека, причинно не связанного с возникновением этой реакции. Этот особый
механизм чаще всего используется в культивации этноцентризма, стереотипов и
предрассудков и называется психологическим процессом перемещения

Суть этого процесса сводится к переносу выражения эмоций в иную ситуацию, где
это можно осуществить либо безопасно, либо с большой выгодой. Психологическое
перемещение используется людьми или бессознательно - в качестве защитного
механизма психики, или преднамеренно - в поисках 'козла отпущения', за
собственные неудачи или ошибки. При этом психологическое перемещение есть
атрибут не только индивидуальной психической деятельности, но и коллективных
психических процессов и способно охватывать значительные массы людей. Механизм
перемещения позволяет направить гнев и враждебность людей на объект, не имеющий
отношения к причинам этих эмоций.

Развиваясь на почве неполного или искаженного знания, предрассудки могут
возникать по поводу объектов самого различного рода - вещей и животных, людей
и их ассоциаций, идей и представлений. Самые распространенные - это этнические
предрассудки. Сохранению и широкому их распространению способствуют некоторые
социально-психологические причины, вытекающие из социально-экономических
условий жизни людей. Одна из них - попытка представителей доминирующей
этнической группы найти источник психического удовлетворения от чувства
мнимого превосходства, находясь при этом у подножия социальной лестницы из-за
сложного материального положения. Отсутствие действительного престижа среди
господствующей этнической группы компенсируется иллюзорным престижем от
сознания принадлежности к 'высшей расе'.

В зарубежной психологии существует довольно много теорий происхождения
предрассудков. Одна из них - теория фрустрации и агрессии. Ее суть заключается
в том, что в человеческой психике в силу определенных причин, вызванных
отрицательными эмоциями; создается состояние напряжения - фрустрация. Это
состояние требует разрядки, и ее объектом может стать лют бой человек. Когда
причины трудностей и невзгод видятся в какой-либо этнической группе, раздражение
направляется против этой группы, причем, как правило, против нее уже накопилось
враждебное предубеждение.

Согласно другой теории формирование предрассудков объясняется потребностью
людей определить свое положение по отношению к другим, основываясь на
превосходстве своей этнической группы и тем самым своего индивидуального Я
над другими. В процессе такого самоутверждения уничижаются достижения другой
группы и подчеркивается неприязненное отношение к ней. В данном случае, по
мнению Г. Тэджфела, можно говорить о социально-экономическом, культурном и
политическом контекстах межгрупповых отношений, которые, в свою очередь, связаны
с этнической и культурной идентификацией взаимодействующих групп. Причем
позитивная идентификация связывается в основном со своей культурной группой,
а к чужой культурной группе демонстрируется негативная идентификация или
даже открытая дискриминация.

Предрассудки усваиваются каждым отдельным индивидом в процессе социализации
и кристаллизуются под влиянием культурно-групповых норм и ценностей. Их
источником служит ближнее окружение человека, в первую очередь родители,
учителя, приятели. Таким образом, индивидуальные предрассудки в большинстве
случаев возникают не из личного опыта межкультурного общения, а в результате
своения сложившихся ранее предрассудков.\cite{Sadoh}

	\subsection{Роль предрассудков в межкультурной коммуникации}

 В повседневной жизни человека предрассудки играют весьма значительную роль.

Во-первых, наличие того или иного предрассудка серьезно искажает для его
носителя восприятие людей из других этнических или социокультурных групп -
он видит в них то, что хочет видеть, а не то, что есть на самом деле.

Во-вторых, в среде людей, зараженных предрассудками, возникает неосознанное
чувство тревоги и страха: они видят потенциальную угрозу, исходящую будто
бы от объектов, подвергнутых ими же дискриминации, что порождает еще больше
недоверия к ним.

В-третьих, предрассудки и основанные на них традиции и практики дискриминации,
сегрегации, ущемления гражданских прав в конечном счете искажают самооценку
объектов этих предрассудков. Значительному числу людей навязывается чувство
социальной неполноценности, и как реакция на это чувство возникает готовность
к утверждению личностной полноценности через межэтнические и межкультурные
конфликты.

Различного рода негативные формы взаимоотношений возникают в результате
влияния какого-то одного предрассудка или группы предрас судков. Все зависит
от того, к какому типу принадлежит соответствующий предрассудок. В психологии
принято выделять шесть основных типов предрассудков:

	\begin{itemize}
    	\item яркие необоснованные предрассудки открыто декларируют: члены чужой группы (по тем или иным признакам) хуже членов собственной группы;
    	\item символические предрассудки предполагают наличие негативных чувств к членам чужой группы, которые якобы угрожают культурным базовым ценностям собственной группы;
    	\item токенизский тип предрассудков выражается в предоставлении социальных преимуществ этническим или социокультурным группам в обществе, чтобы создать видимость справедливости. Предрассудки этого типа предполагают наличие негативных чувств по отношению к чужой группе, однако члены собственной группы не хотят себе в этом признаваться;
    	\item предрассудки типа 'длинной руки' подразумевают позитивное поведение при общении с членами чужой группы только в определенных обстоятельствах (например, случайное знакомство, формальные встречи). В ситуации более близкого контакта (например, соседство) демонстрируется недружелюбное поведение;
    	\item фактические антипатии как тип предрассудков предполагают наличие открыто негативного отношения к членам чужой группы в случае, если их поведение действительно не устраивает членов собственной группы;
    	\item предрассудки типа 'знакомое и незнакомое' подразумевают отказ от контактов с членами чужой группы, поскольку общение с чужаками якобы заставляет в той или иной мере испытывать неудобства Носители этих предрассудков предпочитают взаимодействовать с людьми собственной группы, поскольку это общение не вызывает у них нервных и эмоциональных переживаний.\cite{Sadoh}
	\end{itemize}
\newpage
\section{Заключение}

Стереотипы и предрассудки -- это важные механизмы общения, познания и выживания.
Несмотря на то, что в современном обществе всё чаще возникают тенденции, которые
намерены изгнать стереотипы и предрассудки из общества, я думаю, что это пустое
лицемерие. Как было описано выше, стереотипы возникают не на пустом месте.
Стереотип -- это отражение общей сущности группы. Для того, чтобы другие группы
приняли о данной другие взгляды, -- а именно этого сейчас пытаются добиться
многие социальные группы на западе (движения феминизма, Black Lifes Matter и т.д.), --
необходимо меняться изнутри. Всё остальное фарс. Касательно предрассудков, я думаю,
что это по-своему полезный механизм, обеспечивающий безопасность при столкновении
с новой культурой. Вредить предрассудки и стереотипы начинают тогда, когда при контакте с другими
культурами руководстуются только ими. Очень важно уметь учиться и уточнять модель
окружающего мира, если не для успешной междкультурной коммуникации, то хотя бы
для выживания.



%\addcontentsline{toc}{chapter}{Список литературы}
\bibliographystyle{utf8gost705u}  %% стилевой файл для оформления по ГОСТу
\bibliography{biblio}     %% имя библиографической базы (bib-файла)

\end{document}
