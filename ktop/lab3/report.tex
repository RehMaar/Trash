\documentclass[12pt, a4paper] {ncc}
\usepackage[utf8] {inputenc}
\usepackage[T2A]{fontenc}
\usepackage[english, russian] {babel}
\usepackage[usenames,dvipsnames]{xcolor}
\usepackage{listings,a4wide,longtable,amsmath,amsfonts,graphicx,tikz}
\usepackage{indentfirst,fancyvrb,pdfpages,tabularx}
\usepackage[rounded]{syntax}
\usepackage{calc}

\newcommand{\sel}[1]{$\fbox{#1}$}

\begin{document}
\setcounter{figure}{0}
\frenchspacing
\pagestyle{empty}
% ============================ ТИТУЛЬНЫЙ ЛИСТ ================================
\begin{center}
     Национальный исследовательский университет информационных технологий,
                              механики и оптики.\\
                       Кафедра вычислительной техники.\\
\end{center}
\vspace{\stretch{2}}
\begin{center}
                         {\bf Домашняя работа №3}\\
							по дисциплине: \\
            {\bf <<Конструкторско-техническое обеспечение производства ЭВМ>>}
                Графо-теоретический подход к синтезу топологии\\
                              {\sl Вариант 7}
\end{center}
\vspace{\stretch{3}}
\begin{flushright}
														 Студент:\\
                                                         {\it Куклина Мария, P3401}\\
														 Преподаватель:\\
														 {\it Поляков В.И.}\\
\end{flushright}
\vspace{\stretch{4}}
\begin{center}
                                      2018
\end{center}
% ======================== КОНЕЦ ТИТУЛЬНОГО ЛИСТА ============================
\newpage
\pagestyle{plain}
% ================================ ОТЧЁТ =====================================

\section{Исходные данные}

21 цепь, 13 модулей. 

\begin{tabular}{c|l l l}
Цепь & \multicolumn{3}{l}{Модуль/контакт} \\ \hline
1    & 4/14  & 10/11   & 7/10   \\
2    & 13/7  & 6/9  & \\
3    & 7/9 & 11/7 & \\
4    & 1/5   & 10/14  &       \\
5    & 5/12  & 8/9   &  3/5     \\
6    & 11/9  & 12/14  & 5/13 \\
7    & 13/2  & 12/12  & 2/5  \\
8    & 13/3  & 7/8   & 1/6  \\
9    & 13/9  & 12/9   & 8/10   \\
10   & 1/8 & 8/14   & 5/1   \\
11   & 13/10   & 13/8   &       \\
12   & 5/7 & 7/2  &   \\
13   & 13/5 & 7/13 &   \\
14   & 12/11  & 1/13 & 7/7  \\
15   & 9/14  & 13/6   &  12/8     \\
16   & 11/8 & 9/2  &   \\
17   & 12/5 & 13/1  & 5/11      \\
18   & 13/12   & 5/2   &   \\
19   & 13/14  & 3/4  &       \\
20   & 7/3 & 7/14   &       \\
21   & 13/13  & 4/12   & 4/5  \\
\end{tabular}

\section{Ход работы}

\subsection{Представление исходных данных}

\subsubsection{Матрица комплексов}

\begin{longtable}{c|c c c c c c c c c c c c c}
\hline
      & $e_{1}$ & $e_{2}$ & $e_{3}$ & $e_{4}$ & $e_{5}$ & $e_{6}$ & $e_{7}$ & $e_{8}$ & $e_{9}$ & $e_{10}$ & $e_{11}$ & $e_{12}$ & $e_{13}$\\
\hline    %1   2   3   4   5   6   7   8   9   10  11  12  13
$u_{1} $ & 0 & 0 & 0 & 1 & 0 & 0 & 1 & 0 & 0 & 1 & 0 & 0 & 0 \\
$u_{2} $ & 0 & 0 & 0 & 0 & 0 & 1 & 0 & 0 & 0 & 0 & 0 & 0 & 1 \\
$u_{3} $ & 0 & 0 & 0 & 0 & 0 & 0 & 1 & 0 & 0 & 0 & 1 & 0 & 0 \\
$u_{4} $ & 1 & 0 & 0 & 0 & 0 & 0 & 0 & 0 & 0 & 1 & 0 & 0 & 0 \\
$u_{5} $ & 0 & 0 & 1 & 0 & 1 & 0 & 0 & 1 & 0 & 0 & 0 & 0 & 0 \\
$u_{6} $ & 0 & 0 & 0 & 0 & 1 & 0 & 0 & 0 & 0 & 0 & 1 & 1 & 0 \\
$u_{7} $ & 0 & 1 & 0 & 0 & 0 & 0 & 0 & 0 & 0 & 0 & 0 & 1 & 1 \\
$u_{8} $ & 1 & 0 & 0 & 0 & 0 & 0 & 1 & 0 & 0 & 0 & 0 & 0 & 1 \\
$u_{9} $ & 0 & 0 & 0 & 0 & 0 & 0 & 0 & 1 & 0 & 0 & 0 & 1 & 1 \\
$u_{10}$ & 1 & 0 & 0 & 0 & 1 & 0 & 0 & 1 & 0 & 0 & 0 & 0 & 0 \\
$u_{11}$ & 0 & 0 & 0 & 0 & 0 & 0 & 0 & 0 & 0 & 0 & 0 & 0 & 1 \\
$u_{12}$ & 0 & 0 & 0 & 0 & 1 & 0 & 1 & 0 & 0 & 0 & 0 & 0 & 0 \\
$u_{13}$ & 0 & 0 & 0 & 0 & 0 & 0 & 1 & 0 & 0 & 0 & 0 & 0 & 1 \\
$u_{14}$ & 1 & 0 & 0 & 0 & 0 & 0 & 1 & 0 & 0 & 0 & 0 & 1 & 0 \\
$u_{15}$ & 0 & 0 & 0 & 0 & 0 & 0 & 0 & 0 & 1 & 0 & 0 & 1 & 1 \\
$u_{16}$ & 0 & 0 & 0 & 0 & 0 & 0 & 0 & 0 & 1 & 0 & 1 & 0 & 0 \\
$u_{17}$ & 0 & 0 & 0 & 0 & 1 & 0 & 0 & 0 & 0 & 0 & 0 & 1 & 1 \\
$u_{18}$ & 0 & 0 & 0 & 0 & 1 & 0 & 0 & 0 & 0 & 0 & 0 & 0 & 1 \\
$u_{19}$ & 0 & 0 & 1 & 0 & 0 & 0 & 0 & 0 & 0 & 0 & 0 & 0 & 1 \\
$u_{20}$ & 0 & 0 & 0 & 0 & 0 & 0 & 1 & 0 & 0 & 0 & 0 & 0 & 0 \\
$u_{21}$ & 0 & 0 & 0 & 1 & 0 & 0 & 0 & 0 & 0 & 0 & 0 & 0 & 1 \\
\end{longtable}

\subsubsection{Матрица соединений}

\begin{longtable}{|c|c|c|c|c|c|c|c|c|c|c|c|c|c|}
\hline
&$e_{1}$&$e_{2}$&$e_{3}$&$e_{4}$&$e_{5}$&$e_{6}$&$e_{7}$&$e_{8}$&$e_{9}$&$e_{10}$&$e_{11}$&$e_{12}$&$e_{13}$\\
\hline
$e_{1}$  & 0&0&0&0&1&0&2&1&0&1&0&1&1 \\ \hline 
$e_{2}$  & 0&0&0&0&0&0&0&0&0&0&0&1&1 \\ \hline 
$e_{3}$  & 0&0&0&0&1&0&0&1&0&0&0&0&1 \\ \hline 
$e_{4}$  & 0&0&0&0&0&0&1&0&0&1&0&0&1 \\ \hline 
$e_{5}$  & 1&0&1&0&0&0&1&2&0&0&1&2&2 \\ \hline 
$e_{6}$  & 0&0&0&0&0&0&0&0&0&0&0&0&1 \\ \hline 
$e_{7}$  & 2&0&0&1&1&0&0&0&0&1&1&1&2 \\ \hline 
$e_{8}$  & 1&0&1&0&2&0&0&0&0&0&0&1&1 \\ \hline 
$e_{9}$  & 0&0&0&0&0&0&0&0&0&0&1&1&1 \\ \hline 
$e_{10}$ & 1&0&0&1&0&0&1&0&0&0&0&0&0 \\ \hline 
$e_{11}$ & 0&0&0&0&1&0&1&0&1&0&0&1&0 \\ \hline 
$e_{12}$ & 1&1&0&0&2&0&1&1&1&0&1&0&4 \\ \hline 
$e_{13}$ & 1&1&1&1&2&1&2&1&1&0&0&4&0 \\ \hline 
\end{longtable}

\subsection{Поиск гамильтонова цикла}

\subsubsection{Поиск цикла алгоритмом Робертса-Флореса}

\begin{enumerate}
	\item Включаем в S начальную вершину $S = \{ e_1 \}$.
	\item Первая "возможная" вершина $e_5$. Включаем её в множество $S = \{ e_1, e_5 \}$.
	\item Так до конца: $S = \{ e_1, e_5, e_3, e_8, e_{12}, e_2, e_{13}, e_4, e_7, e_{10}\}$.
	\item У $e_{10}$ нет "возможных" вершин. Удаляем её: $S = \{ e_1, e_5, e_3, e_8, e_{12}, e_2, e_{13}, e_4, e_7\}$.
	\item У $e_7$ есть "возможная" вершина $e_{11}$: $S = \{ e_1, e_5, e_3, e_8, e_{12}, e_2, e_{13}, e_4, e_7, e_{11}\}$.
	\item Так до конца: $S = \{ e_1, e_5, e_3, e_8, e_{12}, e_2, e_{13}, e_4, e_7, e_{11}, e_9\}$.
	\item У $e_9$ нет "возможных" вершин: $S = \{ e_1, e_5, e_3, e_8, e_{12}, e_2, e_{13}, e_4, e_7, e_{11}\}$.
	\item Удаляем все вершины, у которых нет "возможных": $S = \{ e_1, e_5, e_3, e_8, e_{12}, e_2, e_{13}\}$. 
	\item Из анализа матрицы получаем, что в графе точно нет гамильтонова цикла, так как вершина $e_6$ связаны только с вершиной $e_13$.
		  Волевым решением удаляем вершину $e_6$.
	\item Получаем: $S = \{ e_1, e_5, e_3, e_8, e_{12}, e_2, e_{13}, e_9, e_{11}, e_7, e_4, e_{10}\}$.
	\item Есть ребро между $e_1$ и $e_{10}$, значит, гамильтонов цикл найден.
\end{enumerate}

Новая матрица соединений без $e_6$.
\begin{longtable}{|c|c|c|c|c|c|c|c|c|c|c|c|c|}
\hline
&$p_{1}$&$p_{2}$&$p_{3}$&$p_{4}$&$p_{5}$&$p_{6}$&$p_{7}$&$p_{8}$&$p_{9 }$&$p_{10}$&$p_{11}$&$p_{12}$\\
\hline
$p_{1}$  & 0&0&0&0&1&2&1&0&1&0&1&1 \\ \hline 
$p_{2}$  & 0&0&0&0&0&0&0&0&0&0&1&1 \\ \hline 
$p_{3}$  & 0&0&0&0&1&0&1&0&0&0&0&1 \\ \hline 
$p_{4}$  & 0&0&0&0&0&1&0&0&1&0&0&1 \\ \hline 
$p_{5}$  & 1&0&1&0&0&1&2&0&0&1&2&2 \\ \hline 
$p_{6}$  & 2&0&0&1&1&0&0&0&1&1&1&2 \\ \hline 
$p_{7}$  & 1&0&1&0&2&0&0&0&0&0&1&1 \\ \hline 
$p_{8}$  & 0&0&0&0&0&0&0&0&0&1&1&1 \\ \hline 
$p_{9 }$ & 1&0&0&1&0&1&0&0&0&0&0&0 \\ \hline 
$p_{10}$ & 0&0&0&0&1&1&0&1&0&0&1&0 \\ \hline 
$p_{11}$ & 1&1&0&0&2&1&1&1&0&1&0&4 \\ \hline 
$p_{12}$ & 1&1&1&1&2&2&1&1&0&0&4&0 \\ \hline 
\end{longtable}

\subsubsection{Перенумерация вершин}
Перенумеруем вершины таким образом, чтобы его вершины были упорядочены по и
позиции в гамильтоновом цикле.

\begin{tabular}{cc c c c c c c c c c c c}
Новая:  & $e_{1}$ & $e_{2}$ & $e_{3}$  & $e_{4}$ & $e_{5}$ & $e_6$ & $e_{7}$ & $e_{8}$ & $e_{9}$ & $e_{10}$ & $e_{11}$ & $e_{12}$ \\
Старая: & $e_{1}$ & $e_{5}$ & $e_{3}$  & $e_{8}$ & $e_{12}$& $e_{2}$ & $e_{13}$& $e_{9}$ & $e_{11}$& $e_{7}$  & $e_4$ & $e_{10}$ \\
\end{tabular}

%
%Новая матрица соединений:
%
\begin{longtable}{|c|c|c|c|c|c|c|c|c|c|c|c|c|c|}
\hline
      & $e_{1}$ & $e_{2}$ & $e_{3}$ & $e_{4}$ & $e_{5}$ & $e_{6}$ & $e_{7}$ & $e_{8}$ & $e_{9}$ & $e_{10}$ & $e_{11}$ & $e_{12}$ \\
\hline  %  1   2   3   4   5 6(7) 7(8) 8(9)  9(10) 10(11)11(12)12(13)      
$e_{1}$  & 0 &1  & 0  &1 & 1&0  &1   & 0   &0   & 2   & 0   & 1 \\ 
$e_{2}$  & 1 &0  & 1  &2 & 2&0  &2   & 0   &1   & 1   & 0   & 0 \\ 
$e_{3}$  & 0 &1  & 0  &1 & 0&0  &1   & 0   &0   & 0   & 0   & 0 \\ 
$e_{4}$  & 1 &2  & 1  &0 & 1&0  &1   & 0   &0   & 0   & 0   & 0 \\ 
$e_{5}$ & 1 &2  & 0  &1 & 0&1  &4   & 1   &1   & 1   & 0   & 0 \\ 
$e_{6}$  & 0 &0  & 0  &0 & 1&0  &1   & 0   &0   & 0   & 0   & 0 \\ 
$e_{7}$ & 1 &2  & 1  &1 & 4&1  &0   & 1   &0   & 2   & 1   & 0 \\ 
$e_{8}$  & 0 &0  & 0  &0 & 1&0  &1   & 0   &1   & 0   & 0   & 0 \\ 
$e_{9}$ & 0 &1  & 0  &0 & 1&0  &0   & 1   &0   & 1   & 0   & 0 \\ 
$e_{10}$  & 2 &1  & 0  &0 & 1&0  &2   & 0   &1   & 0   & 1   & 1 \\ 
$e_{11}$  & 0 &0  & 0  &0 & 0&0  &1   & 0   &0   & 1   & 0   & 1 \\ 
$e_{12}$  & 1 &0  & 0  &0 & 0&0  &0   & 0   &0   & 1   & 1   & 0 \\ 
\hline
\end{longtable}
%
%То, что вершины пронумерованы по гамильтонову циклу, видно по тому, что рёбра,
%смежные главной диагонали, не нулевые, как и вершины в левом нижнем и правом
%верхнем углах.

\begin{tikzpicture}
\foreach \a in {1,2,...,12}{
\draw (\a*360/12: 4cm) node[circle,draw,minimum width=2em] (p\a) {p\a};
}
\draw(p1) -- (p2) -- (p3) -- (p4) -- (p5) -- (p6) -- (p7) -- (p8) -- (p9)
  -- (p10) -- (p11) -- (p12) -- (p1);

\draw (p1) -- (p4);
\draw (p1) -- (p5);
\draw (p1) -- (p7);
\draw (p1) -- (p10);
\draw (p2) -- (p4);
\draw (p2) -- (p5);
\draw (p2) -- (p7);
\draw (p2) -- (p9);
\draw (p2) -- (p10);
\draw (p3) -- (p7);
\draw (p4) -- (p7);
\draw (p5) -- (p7);
\draw (p5) -- (p8);
\draw (p5) -- (p9);
\draw (p5) -- (p10);
\draw (p7) -- (p10);
\draw (p7) -- (p11);
\draw (p10) -- (p12);

\end{tikzpicture}

\subsection{Построение графа пересечений}

Граф смежности рёбер:

\begin{tikzpicture}
\draw (1*360/18: 6cm)  node[circle,draw,minimum width=3em] (1x4) {1,4};
\draw (2*360/18: 6cm)  node[circle,draw,minimum width=3em] (1x5) {1,5};
\draw (3*360/18: 6cm)  node[circle,draw,minimum width=3em] (1x7) {1,7};
\draw (4*360/18: 6cm)  node[circle,draw,minimum width=3em] (1x10) {1,10};
\draw (5*360/18: 6cm)  node[circle,draw,minimum width=3em] (2x4) {2,4};
\draw (6*360/18: 6cm)  node[circle,draw,minimum width=3em] (2x5) {2,5};
\draw (7*360/18: 6cm)  node[circle,draw,minimum width=3em] (2x7) {2,7};
\draw (8*360/18: 6cm)  node[circle,draw,minimum width=3em] (2x9) {2,9};
\draw (9*360/18: 6cm)  node[circle,draw,minimum width=3em] (2x10) {2,10};
\draw (10*360/18: 6cm) node[circle,draw,minimum width=3em] (3x7) {3,7};
\draw (11*360/18: 6cm) node[circle,draw,minimum width=3em] (4x7) {4,7};
\draw (12*360/18: 6cm) node[circle,draw,minimum width=3em] (5x7) {5,7};
\draw (13*360/18: 6cm) node[circle,draw,minimum width=3em] (5x8) {5,8};
\draw (14*360/18: 6cm) node[circle,draw,minimum width=3em] (5x9) {5,9};
\draw (15*360/18: 6cm) node[circle,draw,minimum width=3em] (5x10) {5,10};
\draw (16*360/18: 6cm) node[circle,draw,minimum width=3em] (7x10) {7,10};
\draw (17*360/18: 6cm) node[circle,draw,minimum width=3em] (7x11) {7,11};
\draw (18*360/18: 6cm) node[circle,draw,minimum width=3em] (10x12) {10,12};

\draw (2x4) -- (1x4);
\draw (2x5) -- (1x4);
\draw (2x7) -- (1x4);
\draw (2x9) -- (1x4);
\draw (2x10) -- (1x4);
\draw (3x7) -- (1x4);

\draw (2x4) -- (1x5);
\draw (2x7) -- (1x5);
\draw (2x9) -- (1x5);
\draw (2x10) -- (1x5);
\draw (3x7) -- (1x5);
\draw (4x7) -- (1x5);

\draw (2x9) -- (1x7);
\draw (2x10) -- (1x7);
\draw (5x8) -- (1x7);
\draw (5x9) -- (1x7);
\draw (5x10) -- (1x7);

\draw (3x7) -- (2x4);

\draw (3x7)  -- (2x5);
\draw (4x7)  -- (2x5);

\draw (5x8) -- (2x7);
\draw (5x9) -- (2x7);
\draw (5x10) -- (2x7);

\draw (5x10) -- (2x9);
\draw (7x10) -- (2x9);

\draw (7x11) -- (2x10);

\draw (5x8) -- (3x7);
\draw (5x9) -- (3x7);
\draw (5x10) -- (3x7);

\draw (5x8) -- (4x7);
\draw (5x9) -- (4x7);
\draw (5x10) -- (4x7);

\draw (7x11) -- (5x8);
\draw (7x10) -- (5x8);

\draw (7x11) -- (5x9);
\draw (7x10) -- (5x9);

\draw (7x10) -- (5x10);

\draw (7x11) -- (10x12);

\end{tikzpicture}

Матрица соединений этого графа:

{\scriptsize\begin{tabularx}{\textwidth}{|c| c|X X X X X X X X X X X X X X X X X X}
\hline
& & $u_{1,4}$ & $u_{1,5}$ & $u_{1,7}$ & $u_{1,10}$ & $u_{2,4}$ & $u_{2,5}$ & $u_{2,7}$ & $u_{2,9}$ & $u_{2,10}$ & $u_{3,7}$ & $u_{4,7}$ & $u_{5,7}$ & $u_{5,8}$ & $u_{5,9}$ & $u_{5,10}$ & $u_{7,10}$ & $u_{7,11}$ & $u_{10,12}$ \\ \hline
& & 1 & 2 & 3 & 4 & 5 & 6 & 7 & 8& 9 & 10 & 11 & 12 & 13 & 14 & 15 & 16 & 17 & 18 \\ \hline
				%1/4 1/5 1/7  1/10 2/4  2/5 2/7  2/9  2/10 3/7  4/7  5/7  5/8  5/9  5/10 7/10 7/11  10/12  
$u_{1,4}$   & 1  & 0 & 0 & 0  & 0  & 1  & 1 & 1  & 1  & 1  & 1  & 0  & 0  & 0  & 0  & 0  & 0  & 0   & 0 \\ \hline
$u_{1,5}$   & 2  & 0 & 0 & 0  & 0  & 1  & 0 & 1  & 1  & 1  & 1  & 1  & 0  & 0  & 0  & 0  & 0  & 0   & 0 \\ \hline
$u_{1,7}$   & 3  & 0 & 0 & 0  & 0  & 0  & 0 & 0  & 1  & 1  & 0  & 0  & 0  & 1  & 1  & 1  & 0  & 0   & 0 \\ \hline
$u_{1,10}$  & 4  & 0 & 0 & 0  & 0  & 0  & 0 & 0  & 0  & 0  & 0  & 0  & 0  & 0  & 0  & 0  & 0  & 1   & 0 \\ \hline
$u_{2,4}$   & 5  & 0 & 0 & 0  & 0  & 0  & 0 & 0  & 0  & 0  & 1  & 0  & 0  & 0  & 0  & 0  & 0  & 0   & 0 \\ \hline
$u_{2,5}$   & 6  & 1 & 0 & 0  & 0  & 0  & 0 & 0  & 0  & 0  & 1  & 1  & 0  & 0  & 0  & 0  & 0  & 0   & 0 \\ \hline
$u_{2,7}$   & 7  & 1 & 1 & 0  & 0  & 0  & 0 & 0  & 0  & 0  & 0  & 0  & 0  & 1  & 1  & 1  & 0  & 0   & 0 \\ \hline
$u_{2,9}$   & 8  & 1 & 1 & 1  & 0  & 0  & 0 & 0  & 0  & 0  & 0  & 0  & 0  & 0  & 0  & 1  & 1  & 0   & 0 \\ \hline
$u_{2,10}$  & 9  & 1 & 1 & 1  & 0  & 0  & 0 & 0  & 0  & 0  & 0  & 0  & 0  & 0  & 0  & 0  & 0  & 1   & 0 \\ \hline
$u_{3,7}$   &10  & 1 & 1 & 0  & 0  & 1  & 1 & 0  & 0  & 0  & 0  & 0  & 0  & 1  & 1  & 1  & 0  & 0   & 0 \\ \hline
$u_{4,7}$   &11  & 0 & 1 & 0  & 0  & 0  & 1 & 0  & 0  & 0  & 0  & 0  & 0  & 1  & 1  & 1  & 0  & 0   & 0 \\ \hline
$u_{5,7}$   &12  & 0 & 0 & 0  & 0  & 0  & 0 & 0  & 0  & 0  & 0  & 0  & 0  & 0  & 0  & 0  & 0  & 0   & 0 \\ \hline
$u_{5,8}$   &13  & 0 & 0 & 1  & 0  & 0  & 0 & 1  & 0  & 0  & 1  & 1  & 0  & 0  & 0  & 0  & 1  & 1   & 0 \\ \hline
$u_{5,9}$   &14  & 0 & 0 & 1  & 0  & 0  & 0 & 1  & 0  & 0  & 1  & 1  & 0  & 0  & 0  & 0  & 1  & 1   & 0 \\ \hline
$u_{5,10}$  &15  & 0 & 0 & 1  & 0  & 0  & 0 & 1  & 0  & 0  & 1  & 1  & 0  & 0  & 0  & 0  & 0  & 1   & 0 \\ \hline
$u_{7,10}$  &16  & 0 & 0 & 0  & 0  & 0  & 0 & 0  & 1  & 0  & 0  & 0  & 0  & 1  & 1  & 0  & 0  & 0   & 0 \\ \hline
$u_{7,11}$  &17  & 0 & 0 & 0  & 1  & 0  & 0 & 0  & 1  & 1  & 0  & 0  & 0  & 1  & 1  & 1  & 0  & 0   & 1 \\ \hline
$u_{10,12}$ & 18 & 0 & 0 & 0  & 0  & 0  & 0 & 0  & 0  & 0  & 0  & 0  & 0  & 0  & 0  & 0  & 0  & 1   & 0 \\ \hline
\end{tabularx}}

\subsection{Нахождение максимальных внутренне устойчивых подмножеств}

{\scriptsize\begin{tabularx}{\textwidth}{|c| c|X X X X X X X X X X X X X X X X X X}
\hline
& & $u_{1,4}$ & $u_{1,5}$ & $u_{1,7}$ & $u_{1,10}$ & $u_{2,4}$ & $u_{2,5}$ & $u_{2,7}$ & $u_{2,9}$ & $u_{2,10}$ & $u_{3,7}$ & $u_{4,7}$ & $u_{5,7}$ & $u_{5,8}$ & $u_{5,9}$ & $u_{5,10}$ & $u_{7,10}$ & $u_{7,11}$ & $u_{10,12}$ \\ \hline
& & 1 & 2 & 3 & 4 & 5 & 6 & 7 & 8& 9 & 10 & 11 & 12 & 13 & 14 & 15 & 16 & 17 & 18 \\ \hline
				%1/4 1/5 1/7  1/10 2/4  2/5 2/7  2/9  2/10 3/7  4/7  5/7  5/8  5/9  5/10 7/10 7/11  10/12  
$u_{1,4}$   & 1  & 1 & 0 & 0  & 0  & 1  & 1 & 1  & 1  & 1  & 1  & 0  & 0  & 0  & 0  & 0  & 0  & 0   & 0 \\ \hline
$u_{1,5}$   & 2  & 0 & 1 & 0  & 0  & 1  & 0 & 1  & 1  & 1  & 1  & 1  & 0  & 0  & 0  & 0  & 0  & 0   & 0 \\ \hline
$u_{1,7}$   & 3  & 0 & 0 & 1  & 0  & 0  & 0 & 0  & 1  & 1  & 0  & 0  & 0  & 1  & 1  & 1  & 0  & 0   & 0 \\ \hline
$u_{1,10}$  & 4  & 0 & 0 & 0  & 1  & 0  & 0 & 0  & 0  & 0  & 0  & 0  & 0  & 0  & 0  & 0  & 0  & 1   & 0 \\ \hline
$u_{2,4}$   & 5  & 0 & 0 & 0  & 0  & 1  & 0 & 0  & 0  & 0  & 1  & 0  & 0  & 0  & 0  & 0  & 0  & 0   & 0 \\ \hline
$u_{2,5}$   & 6  & 1 & 0 & 0  & 0  & 0  & 1 & 0  & 0  & 0  & 1  & 1  & 0  & 0  & 0  & 0  & 0  & 0   & 0 \\ \hline
$u_{2,7}$   & 7  & 1 & 1 & 0  & 0  & 0  & 0 & 1  & 0  & 0  & 0  & 0  & 0  & 1  & 1  & 1  & 0  & 0   & 0 \\ \hline
$u_{2,9}$   & 8  & 1 & 1 & 1  & 0  & 0  & 0 & 0  & 1  & 0  & 0  & 0  & 0  & 0  & 0  & 1  & 1  & 0   & 0 \\ \hline
$u_{2,10}$  & 9  & 1 & 1 & 1  & 0  & 0  & 0 & 0  & 0  & 1  & 0  & 0  & 0  & 0  & 0  & 0  & 0  & 1   & 0 \\ \hline
$u_{3,7}$   &10  & 1 & 1 & 0  & 0  & 1  & 1 & 0  & 0  & 0  & 1  & 0  & 0  & 1  & 1  & 1  & 0  & 0   & 0 \\ \hline
$u_{4,7}$   &11  & 0 & 1 & 0  & 0  & 0  & 1 & 0  & 0  & 0  & 0  & 1  & 0  & 1  & 1  & 1  & 0  & 0   & 0 \\ \hline
$u_{5,7}$   &12  & 0 & 0 & 0  & 0  & 0  & 0 & 0  & 0  & 0  & 0  & 0  & 1  & 0  & 0  & 0  & 0  & 0   & 0 \\ \hline
$u_{5,8}$   &13  & 0 & 0 & 1  & 0  & 0  & 0 & 1  & 0  & 0  & 1  & 1  & 0  & 1  & 0  & 0  & 1  & 1   & 0 \\ \hline
$u_{5,9}$   &14  & 0 & 0 & 1  & 0  & 0  & 0 & 1  & 0  & 0  & 1  & 1  & 0  & 0  & 1  & 0  & 1  & 1   & 0 \\ \hline
$u_{5,10}$  &15  & 0 & 0 & 1  & 0  & 0  & 0 & 1  & 0  & 0  & 1  & 1  & 0  & 0  & 0  & 1  & 0  & 1   & 0 \\ \hline
$u_{7,10}$  &16  & 0 & 0 & 0  & 0  & 0  & 0 & 0  & 1  & 0  & 0  & 0  & 0  & 1  & 1  & 0  & 1  & 0   & 0 \\ \hline
$u_{7,11}$  &17  & 0 & 0 & 0  & 1  & 0  & 0 & 0  & 1  & 1  & 0  & 0  & 0  & 1  & 1  & 1  & 0  & 1   & 1 \\ \hline
$u_{10,12}$ & 18 & 0 & 0 & 0  & 0  & 0  & 0 & 0  & 0  & 0  & 0  & 0  & 0  & 0  & 0  & 0  & 0  & 1   & 1 \\ \hline
\end{tabularx}}

Найдём семейство $\Psi$.

\begin{enumerate}
\item Нули в первой строке соответствуют элементам

$$J(j) = \{2, 3, 4, 11, 12, 13, 14, 15, 16, 17, 18\}$$

\item Для первого нулевого элемента $2$ составляем дизъюнкцию:
 		$$M_{1,2} = r_1 \lor r_2 = 110011111110000000.$$

\begin{enumerate}
\item В строке $M_{1,2}$ находим номера нулевых элементов: $$J'(j') = \{ 3, 4, 12, 13, 14, 15, 16, 17, 18\}.$$
    \begin{enumerate}
        \item Составляем дизъюнкцию для $m_3$: $$M_{1,2,3} = M_{1,2} \lor r_3 = 111011111110111000.$$
        \item Составляем дизъюнкцию для $m_4$: $$M_{1,2,3,4} = M_{1,2, 3} \lor r_4 = 111111111110111010.$$
        \item Составляем дизъюнкцию для $m_{12}$: $$M_{1,2,3,4,12} = M_{1,2, 3, 4} \lor r_{12} = 111111111111111010.$$
        \item Составляем дизъюнкцию для $m_{16}$: $$M_{1,2,3,4,12,16} = M_{1,2, 3, 4,12} \lor r_{16} = 111111111111111110.$$
        \item Составляем дизъюнкцию для $m_{18}$: $$M_{1,2,3,4,12,16,18} = M_{1,2, 3, 4,12,16} \lor r_{18} = 111111111111111111.$$
        \item В дизъюнкции все единицы, значит построено $$\psi_1 = \{ u_{1,4}, u_{1,5}, u_{1,7}, u_{1,10}, u_{5,7}, u_{7,10}, u_{10,12} \}$$.
    \end{enumerate}
\item Рассматриваем следующий элемент из $J'(j')$ -- $4$.
    \begin{enumerate}
		\item $M_{1,2,4} = M_{1,2} \lor r_4 = 110111111110000010$.
		\item $m_{12}$: $M_{1,2,4,12} = 110111111111000010$
		\item $m_{13}$: $M_{1,2,4,12,13} = 111111111111100110$
		\item $m_{14}$: $M_{1,2,4,12,13,14} = 111111111111110110$
		\item $m_{15}$: $M_{1,2,4,12,13,14,15} = 111111111111111110$
		\item $m_{18}$: $M_{1,2,4,12,13,14,15,18} = 111111111111111111$
        \item В дизъюнкции все единицы, значит построено $$\psi_2 = \{ u_{1,4}, u_{1,5}, u_{1,10}, u_{5,7}, u_{5,8}, u_{5,9}  u_{5,10}, u_{10,12}, \}$$.
	\end{enumerate}
\item Рассматриваем следующий элемент из $J'(j')$ -- $12$.
    \begin{enumerate}
		\item $M_{1,2,12} = M_{1,2} \lor r_{12} = 110011111111000000.$
		\item $m_{13}$: $M_{1, 2, 12,13} = 111011111111100110$.
		\item $m_{14}$: $M_{1, 2, 12,13,14} = 111011111111110110$.
		\item $m_{15}$: $M_{1, 2, 12,13,14,15} = 111011111111110110$.
		\item $1$ в $m_4$ можно получить только с помощью вершины 17, которая не войдёт в дизъюнкцию, значит, этот элемент закрыть не получится.
	\end{enumerate}
\item Рассматриваем следующий элемент из $J'(j')$ -- $13$.
    \begin{enumerate}
		\item $M_{1,2,13} = M_{1,2} \lor r_{13} = 111011111110100110.$
		\item Так как в $m_{17}$ ноль, то уже ничего не закроет $m_4$.
    \end{enumerate}
\item Элементы 14 и 15 также приводят к тому, что $m_4$ не закроется. Последующие элементы приведут к тому, что не закроектся $m_3$.
\end{enumerate}
\end{enumerate}

По той же логике проводим дальнейшие вычисления и получаем результат:
\begin{enumerate}
	\item $\psi_1 = \{ 1,2,3,4,12,16,18 \}$
	\item $\psi_2 = \{ 1,2,4,12,13,14,15,18 \}$
	\item $\psi_3 = \{ 1,3,4,11,12,16,18 \}$
	\item $\psi_4 = \{ 1,3,11,12,16,17 \}$
	\item $\psi_5 = \{ 2,3,4,6,12,16,18 \}$
	\item $\psi_6 = \{ 2,3,6,12,16,17 \}$
	\item $\psi_7 = \{ 2,4,6,12,13,14,15,18 \}$
	\item $\psi_8 = \{ 3,4,5,6,7,12,16,18 \}$
	\item $\psi_9 = \{ 3,4,7,10,11,12,16,18 \}$
	\item $\psi_{10}= \{ 3,5,6,7,12,16,17 \}$
	\item $\psi_{11}= \{ 3,5,7,11,12,16,17 \}$
	\item $\psi_{12}= \{ 3,7,10,11,12,16,17 \}$
	\item $\psi_{13}= \{ 4,5,6,7,8,9,12,18 \}$
	\item $\psi_{14}= \{ 4,5,7,8,9,11,12,18 \}$
	\item $\psi_{15}= \{ 4,7,8,9,10,11,12,18 \}$
	\item $\psi_{16}= \{ 4,7,9,10,11,12,16,18 \}$
	\item $\psi_{17}= \{ 7,8,10,11,12,17 \}$
\end{enumerate}

\subsection{Поиск максимального двудольного подграфа}

Высчитаем для каждой пары множеств из семейства $\Psi$ значение $|\psi_i| +
|\psi_j| - |\psi_i \cup \psi_j|$ и составим матрицу:

\begin{tabular}{c|c c c c c c c c c c c c c c c c c c}
\hline
& $\psi_1$ & $\psi_2$ & $\psi_3$ & $\psi_4$ & $\psi_5$ & $\psi_6$ & $\psi_7$ & $\psi_8$ & $\psi_9$ & $\psi_{10}$ & $\psi_{11}$ & $\psi_{12}$ & $\psi_{13}$ & $\psi_{14}$ & $\psi_{15}$ & $\psi_{16}$ & $\psi_{17}$\\ \hline
	$\psi_{1} $  & 0 &10&8 &9 &8 &9 &11&10&10&11&11&11&12&12&12&11&12 \\
	$\psi_{2} $  &   & 0&11&12&11&12&9 &13&13&\textbf{14}&\textbf{14}&\textbf{14}&13&13&13&13&13 \\
	$\psi_{3} $  &   &  & 0&8 &9 &10&12&10&9 &11&10&10&12&11&11&10&11 \\
	$\psi_{4} $  &   &  &  & 0&10&8 &13&11&10&9 &8 &8 &13&12&12&11&9  \\
	$\psi_{5} $  &   &  &  &  & 0&8 &10&9 &10&10&11&11&11&12&12&11&12 \\
	$\psi_{6} $  &   &  &  &  &  & 0&11&10&11&8 &9 &9 &12&13&13&12&10 \\
	$\psi_{7} $  &   &  &  &  &  &  &0 &12&13&13&\textbf{14}&\textbf{14}&12&13&13&13&13 \\
	$\psi_{8} $  &   &  &  &  &  &  &  & 0&10&9 &10&11&10&11&12&11&12 \\
	$\psi_{9} $  &   &  &  &  &  &  &  &  & 0&11&10&9 &12&11&10&9 &10 \\
	$\psi_{10} $ &   &  &  &  &  &  &  &  &  & 0&8 &9 &11&12&13&12&10 \\
	$\psi_{11} $ &   &  &  &  &  &  &  &  &  &  & 0&8 &12&11&12&11&9  \\
	$\psi_{12} $ &   &  &  &  &  &  &  &  &  &  &  & 0&13&12&11&10&8  \\
	$\psi_{13} $ &   &  &  &  &  &  &  &  &  &  &  &  & 0&9 &10&11&11 \\
	$\psi_{14} $ &   &  &  &  &  &  &  &  &  &  &  &  &  & 0&9 &10&10 \\
	$\psi_{15} $ &   &  &  &  &  &  &  &  &  &  &  &  &  &  & 0&9 &9  \\
	$\psi_{16} $ &   &  &  &  &  &  &  &  &  &  &  &  &  &  &  & 0&10 \\
	$\psi_{17} $ &   &  &  &  &  &  &  &  &  &  &  &  &  &  &  &  &0  \\
\end{tabular}


Возьмём множества
$$\psi_2 = \{ 1, 2, 4, 12, 13, 15, 18\} = \{ u_{1,4}, u_{1,5}, u_{1,10}, u_{5,7}, u_{5,8}, u_{5,10}, u_{10,12}\}$$
и $$\psi_{10} = \{ 3, 5, 6, 7, 12, 16, 17\} = \{ u_{1,7}, u_{2,4}, u_{2,5}, u_{2,7}, u_{5,7}, u_{7,10}, u_{7,11}\} $$
\begin{tikzpicture}
    \foreach \a in {1,2,...,12}{
        \draw (\a*360/12: 4cm) node[circle,draw,minimum width=2em] (p\a) {$e_{\a}$};
    }
\draw(p1) -- (p2) -- (p3) -- (p4) -- (p5) -- (p6) -- (p7) -- (p8) -- (p9)
  -- (p10) -- (p11) -- (p12) -- (p1);

\draw[red] (p1) to [bend right=60] (p4);
\draw[red] (p1) to [bend right=90,looseness=1.1]  (p5);
\draw[red] (p1) to [bend left=90]  (p10);
\draw[red] (p5) to [bend right=45]  (p7);
\draw[red] (p5) to [bend right=70]  (p8);
\draw[red] (p5) to [out=210, in=220,looseness=1.5]  (p10);
\draw[red] (p10) to [bend right=60]  (p12);

\draw[blue] (p1) -- (p7);
\draw[blue] (p2) -- (p4);
\draw[blue] (p2) -- (p5);
\draw[blue] (p2) -- (p7);
\draw[blue] (p5) -- (p7);
\draw[blue] (p7) -- (p10);
\draw[blue] (p7) -- (p11);

\end{tikzpicture}

$\psi_{10}$ -- синие рёбра внутри гамильтонова цикла. $\psi_2$ -- красные рёбра вне гамильтонова цикла.

Удалим из $\Psi_G$ рёбра, вошедшие в $\psi_2$ и $\psi_{10}$.
\begin{enumerate}
	\item $\psi_3'   = \{11 \}$
	\item $\psi_9'   = \{ 10,11 \}$
	\item $\psi_{12}' = \{ 10,11 \}$
	\item $\psi_{13}' = \{ 8,9 \}$
	\item $\psi_{14}' = \{ 8,9,11 \}$
	\item $\psi_{15}' = \{ 8,9,10,11 \}$
	\item $\psi_{16}' = \{ 9,10,11 \}$
	\item $\psi_{17}' = \{ 8,10,11 \}$
\end{enumerate}

Объединяем одинаковые множества.

\begin{tabular}{c|c c c c c c c c}
\hline % 3(1) 4(2) 6(1) 8(2) 9(3) 10(4) 11(4) 12(4) 13(5) 14(5) 15(5) 16(4) 17(5)
		   & $\psi_3'$ & $\psi_9'$ & $\psi_{12}'$& $\psi_{13}'$& $\psi_{14}'$& $\psi_{15}'$& $\psi_{16}'$& $\psi_{17}'$ \\
$\psi_3'   $  &0&2&2&3&3&\textbf{4}&3&3 \\
$\psi_9'   $  & &0&2&\textbf{4}&\textbf{4}&\textbf{4}&3&3 \\
$\psi_{12}' $ & & &0&\textbf{4}&\textbf{4}&\textbf{4}&3&3 \\
$\psi_{13}' $ & & & &0&3&\textbf{4}&\textbf{4}&\textbf{4} \\
$\psi_{14}' $ & & & & &0&\textbf{4}&\textbf{4}&\textbf{4} \\
$\psi_{15}' $ & & & & & &0&\textbf{4}&\textbf{4} \\
$\psi_{16}' $ & & & & & & &0&\textbf{4} \\
$\psi_{17}' $ & & & & & & & &0 \\ \hline
\end{tabular}

Выбираем $$\psi_3' = \{ 11 \} = \{ u_{4,7} \}$$ и $$\psi_{15}' = \{ 8, 9, 10, 11\} = \{ u_{2,9}, u_{2,10}, u_{3,7}, u_{4,7}\}$$.
\begin{tikzpicture}
    \foreach \a in {1,2,...,12}{
        \draw (\a*360/12: 4cm) node[circle,draw,minimum width=2em] (p\a) {$e_{\a}$};
    }
\draw(p1) -- (p2) -- (p3) -- (p4) -- (p5) -- (p6) -- (p7) -- (p8) -- (p9)
  -- (p10) -- (p11) -- (p12) -- (p1);

\draw[red] (p4) to [bend right=75] (p7);

\draw[blue] (p2) to (p9);
\draw[blue] (p2) to (p10);
\draw[blue] (p3) to (p7);
\draw[blue] (p4) to (p7);

\end{tikzpicture}

Все рёбра графа реализованы.

\subsection{Проверка изоморфизма графов}

Проведём проверку на изоморфизм исходный граф и граф, полученный
перенумеровыванием вершин после нахождения гамильтонова цикла.

\begin{tabular}{|c|c c|}
\hline
\bf Значение & $G_1$ & $G_2$ \\
\hline
Число вершин $m$         & 12 & 12 \\
Число рёбер $k$          & 60 & 60 \\
Компоненты связности $p$ & 1  & 1 \\
\hline
\end{tabular}

По основным инвариантам графы совпадают.

Список вершин и соответствующих рангов:

\begin{tabular}{|c|c c|}
\hline
\bf Ранг & $G_2$ & $G_1$ \\
\hline
9 & $e_7$                     & $p_{12}$ \\
8 & $e_5$                     & $p_{11}$ \\
7 & $e_2, e_{10}$             & $p_5, p_6$ \\
6 & $e_1$                     &  $p_1$\\
5 & $e_4$                     & $p_7$ \\
4 & $e_9$                     & $p_{10}$\\
3 & $e_3, e_8, e_{11}, e_{12}$& $p_3, p_4, p_8, p_9$\\
2 & $e_6$                     & $p_2$\\
\hline
\end{tabular}

Распределение вершин по рангам совпадает.

Из таблицы видно соответствие следующих вершин графа:

\begin{tabular}{|cc|}
\hline
$G_2$ & $G_1$ \\ \hline
$e_7$ & $p_{12}$ \\
$e_5$ & $p_{11}$ \\
$e_1$ & $p_1$ \\
$e_4$ & $p_7$ \\
$e_9$ & $p_{10}$ \\
$e_6$ & $p_2$ \\
\hline
\end{tabular}

Рассмотрим, с какими вершинами связаны оставшиеся вершины.

\begin{tabular}{|c c||c c|}
\hline
$G_2$    &                                          & $G_1$ & \\
$e_2$    & $e_1, e_3, e_4, e_5, e_7, e_9, e_{10}$   & $p_{5}$ & $p_1, p_3, p_6, p_7, p_{10}, p_{11}, p_{12}$ \\
$e_{10}$ & $e_1, e_2, e_5, e_7, e_9, e_{11}, e_{12}$& $p_{6}$ & $p_1, p_4, p_5, p_9, p_{10}, p_{11}, p_{12}$ \\
$e_3$    & $e_2, e_4, e_7$                          & $p_{3}$ & $p_5, p_7, p_{12}$ \\
$e_8$    & $e_5, e_7, e_9$                          & $p_{4}$ & $p_6, p_9, p_{12}$ \\ 
$e_{11}$ & $e_7, e_{10}, e_{12}$                    & $p_{8}$ & $p_{10}, p_{11}, p_{12}$ \\ 
$e_{12}$ & $e_1, e_{10}, e_{11}$                    & $p_{9}$ & $p_1, p_4, p_6$ \\
\hline
\end{tabular}

Наблюдаем, что 

Заменим известные нам вершины.

\begin{tabular}{|c c||c c|}
\hline
$G_2$    &                                          & $G_1$ & \\
$e_2$    & $e_1, e_3, e_4, e_5, e_7, e_9, e{10}$    & $p_{5}$ & $e_1, p_3, p_6, e_4, e_9, e_5, e_7$ \\
$e_{10}$ & $e_1, e_2, e_5, e_7, e_9, e_{11}, e_{12}$& $p_{6}$ & $e_1, p_4, p_5, p_9, e_9, e_5, e_7$ \\
$e_3$    & $e_2, e_4, e_7$                          & $p_{3}$ & $p_5, e_4, e_7$ \\
$e_8$    & $e_5, e_7, e_9$                          & $p_{4}$ & $p_6, p_9, e_7$ \\ 
$e_{11}$ & $e_7, e_{10}, e_{12}$                    & $p_{8}$ & $e_5, e_7, e_9$ \\ 
$e_{12}$ & $e_1, e_{10}, e_{11}$                    & $p_{9}$ & $e_1, p_4, p_6$ \\
\hline
\end{tabular}

Из анализа таблицы получаем:

\begin{tabular}{|cc|}
\hline
$G_2$ & $G_1$ \\ \hline
$e_8$ & $p_{8}$ \\
$e_3$ & $p_{3}$ \\
$e_2$ & $p_5$ \\
$e_{10}$ & $p_6$ \\
$e_{11}$ & $p_4$ \\
$e_{12}$ & $p_9$ \\
\hline
\end{tabular}

Наблюдаем, что существует однозначное соответствие между вершинам
 первого и второго графов, следовательно, графы изоморфны. 

\end{document}
