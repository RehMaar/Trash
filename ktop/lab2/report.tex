\documentclass[12pt, a4paper] {ncc}
\usepackage[utf8] {inputenc}
\usepackage[T2A]{fontenc}
\usepackage[english, russian] {babel}
\usepackage[usenames,dvipsnames]{xcolor}
\usepackage{listings,a4wide,longtable,amsmath,amsfonts,graphicx,tikz}
\usepackage{indentfirst,fancyvrb,pdfpages,tabularx}
\usepackage[rounded]{syntax}

\newcommand{\sel}[1]{$\fbox{#1}$}

\begin{document}
\setcounter{figure}{0}
\frenchspacing
\pagestyle{empty}
% ============================ ТИТУЛЬНЫЙ ЛИСТ ================================
\begin{center}
     Национальный исследовательский университет информационных технологий,
                              механики и оптики.\\
                       Кафедра вычислительной техники.\\
            Конструкторско-техническое обеспечение производства ЭВМ.
\end{center}
\vspace{\stretch{2}}
\begin{center}
                         {\bf Домашняя работа №2}\\
                Графовое представление электрических схем\\
                              {\sl Вариант 5}
\end{center}
\vspace{\stretch{3}}
\begin{flushright}
                                    Студент:\\
                                    {\it Куклина М.Д., P3401}\\
                                    Преподаватель: \\
                                    {\it Поляков В.И.}
\end{flushright}
\vspace{\stretch{4}}
\begin{center}
                                      Санкт-Петербург, 2018
\end{center}
% ======================== КОНЕЦ ТИТУЛЬНОГО ЛИСТА ============================
\newpage
\pagestyle{plain}
% ================================ ОТЧЁТ =====================================
\section{Представление исходных данных}

\subsection{Матрица комплексов}

Матрица в транспонированном виде.

\begin{tabular}{|c|c|c|c|c|c|c|c|c|c|c|c|c|c|c|c|c|c|}
\hline
      & $e_{1}$ & $e_{2}$ & $e_{3}$ & $e_{4}$ & $e_{5}$ & $e_{6}$ & $e_{7}$ & $e_{8}$ & $e_{9}$ & $e_{10}$ & $e_{11}$ & $e_{12}$ & $e_{13}$ & $e_{14}$ & $e_{15}$ & $e_{16}$ & $e_{17}$ \\
\hline   % 1 % 2 % 3 % 4 % 5 % 6 % 7 % 8 % 9 %10 %11 %12 %13 %14 %15 %16 %17
$u_{1} $ & 0 & 0 & 1 & 0 & 0 & 1 & 0 & 0 & 0 & 0 & 0 & 0 & 1 & 0 & 0 & 1 & 0 \\
$u_{2} $ & 0 & 0 & 0 & 1 & 0 & 0 & 0 & 0 & 0 & 0 & 0 & 0 & 0 & 0 & 1 & 0 & 0 \\
$u_{3} $ & 0 & 0 & 0 & 0 & 0 & 0 & 0 & 0 & 0 & 1 & 0 & 0 & 1 & 0 & 0 & 1 & 1 \\
$u_{4} $ & 0 & 0 & 0 & 0 & 0 & 0 & 1 & 0 & 0 & 0 & 0 & 0 & 0 & 0 & 0 & 0 & 0 \\
$u_{5} $ & 0 & 0 & 0 & 1 & 0 & 0 & 0 & 0 & 0 & 0 & 0 & 0 & 0 & 0 & 0 & 0 & 1 \\
$u_{6} $ & 0 & 0 & 0 & 1 & 0 & 0 & 0 & 0 & 0 & 0 & 0 & 1 & 0 & 0 & 0 & 0 & 1 \\
$u_{7} $ & 1 & 0 & 0 & 0 & 0 & 0 & 0 & 0 & 0 & 0 & 1 & 0 & 0 & 0 & 0 & 0 & 0 \\
$u_{8} $ & 0 & 0 & 0 & 0 & 0 & 1 & 0 & 0 & 0 & 0 & 0 & 0 & 1 & 0 & 0 & 0 & 0 \\
$u_{9} $ & 0 & 0 & 0 & 0 & 0 & 0 & 0 & 0 & 1 & 1 & 0 & 0 & 0 & 0 & 1 & 0 & 0 \\
$u_{10}$ & 0 & 0 & 0 & 0 & 1 & 0 & 0 & 0 & 0 & 0 & 0 & 0 & 0 & 0 & 0 & 0 & 1 \\
$u_{11}$ & 0 & 1 & 0 & 1 & 0 & 0 & 0 & 0 & 0 & 0 & 0 & 0 & 0 & 0 & 0 & 0 & 1 \\
$u_{12}$ & 1 & 0 & 0 & 0 & 0 & 1 & 0 & 0 & 0 & 0 & 0 & 0 & 0 & 0 & 0 & 1 & 1 \\
$u_{13}$ & 0 & 0 & 0 & 0 & 0 & 1 & 0 & 0 & 0 & 1 & 0 & 0 & 0 & 1 & 1 & 0 & 0 \\
$u_{14}$ & 0 & 0 & 0 & 0 & 0 & 0 & 0 & 0 & 0 & 0 & 0 & 0 & 0 & 1 & 0 & 1 & 1 \\
$u_{15}$ & 0 & 0 & 0 & 0 & 0 & 0 & 0 & 0 & 1 & 0 & 0 & 0 & 0 & 1 & 0 & 0 & 1 \\
$u_{16}$ & 0 & 0 & 0 & 0 & 0 & 0 & 0 & 0 & 0 & 0 & 0 & 0 & 1 & 0 & 0 & 1 & 0 \\
$u_{17}$ & 0 & 0 & 0 & 0 & 0 & 1 & 0 & 1 & 0 & 0 & 0 & 0 & 1 & 0 & 0 & 0 & 0 \\
$u_{18}$ & 0 & 1 & 0 & 0 & 0 & 1 & 0 & 0 & 0 & 0 & 0 & 0 & 0 & 0 & 0 & 0 & 0 \\
$u_{19}$ & 1 & 0 & 0 & 0 & 0 & 0 & 0 & 1 & 0 & 0 & 1 & 0 & 0 & 0 & 0 & 0 & 0 \\
$u_{20}$ & 0 & 0 & 0 & 0 & 0 & 0 & 0 & 0 & 1 & 1 & 1 & 0 & 0 & 0 & 1 & 0 & 0 \\
$u_{21}$ & 0 & 0 & 1 & 0 & 0 & 0 & 0 & 0 & 0 & 0 & 0 & 0 & 0 & 1 & 0 & 0 & 0 \\
$u_{22}$ & 0 & 1 & 0 & 1 & 0 & 0 & 0 & 0 & 0 & 0 & 0 & 1 & 1 & 0 & 0 & 0 & 0 \\
$u_{23}$ & 0 & 1 & 0 & 0 & 0 & 1 & 0 & 0 & 1 & 0 & 0 & 1 & 0 & 0 & 0 & 0 & 0 \\
$u_{24}$ & 1 & 1 & 0 & 0 & 1 & 0 & 0 & 0 & 0 & 0 & 0 & 0 & 0 & 0 & 0 & 1 & 0 \\
$u_{25}$ & 0 & 0 & 0 & 1 & 0 & 1 & 0 & 0 & 1 & 0 & 0 & 1 & 0 & 0 & 0 & 0 & 0 \\
$u_{26}$ & 0 & 0 & 1 & 0 & 0 & 0 & 0 & 0 & 0 & 0 & 0 & 0 & 0 & 1 & 1 & 0 & 0 \\
$u_{27}$ & 0 & 0 & 0 & 0 & 0 & 0 & 0 & 0 & 0 & 0 & 0 & 1 & 0 & 1 & 0 & 1 & 0 \\
$u_{28}$ & 0 & 1 & 0 & 0 & 0 & 0 & 0 & 0 & 0 & 0 & 0 & 1 & 1 & 0 & 1 & 0 & 0 \\
$u_{29}$ & 0 & 0 & 0 & 1 & 0 & 0 & 0 & 0 & 0 & 0 & 0 & 0 & 0 & 1 & 0 & 1 & 0 \\
$u_{30}$ & 0 & 0 & 0 & 0 & 0 & 0 & 0 & 0 & 0 & 0 & 0 & 0 & 1 & 0 & 1 & 0 & 0 \\
$u_{31}$ & 0 & 0 & 0 & 0 & 0 & 0 & 1 & 0 & 0 & 0 & 0 & 1 & 0 & 0 & 1 & 1 & 0 \\
$u_{32}$ & 0 & 1 & 0 & 0 & 1 & 0 & 0 & 0 & 0 & 0 & 1 & 0 & 0 & 0 & 0 & 0 & 0 \\
\hline
\end{tabular}

\subsection{Матрица соединений}

% Числом задаётся, какое количество связей существует между парой модулей. 
Из-за того, что граф неориентированный, достаточно указать верхнюю треугольную матрицу.

\begin{tabular}{|c|c|c|c|c|c|c|c|c|c|c|c|c|c|c|c|c|c|}
\hline
      & $e_{1}$ & $e_{2}$ & $e_{3}$ & $e_{4}$ & $e_{5}$ & $e_{6}$ & $e_{7}$ & $e_{8}$ & $e_{9}$ & $e_{10}$ & $e_{11}$ & $e_{12}$ & $e_{13}$ & $e_{14}$ & $e_{15}$ & $e_{16}$ & $e_{17}$ \\
\hline
$e_{1}$  & 0 & 1&0&0&1&1&0&1&0&0&2&0&0&0&0&2&1\\ \hline
$e_{2}$  &   & 0&0&2&2&2&0&0&1&0&1&3&2&0&1&1&1\\ \hline
$e_{3}$  &   &  &0&0&0&1&0&0&0&0&0&0&1&2&1&1&0\\ \hline
$e_{4}$  &   &  & &0&0&1&0&0&1&0&0&3&1&1&1&1&3\\ \hline
$e_{5}$  &   &  & & &0&0&0&0&0&0&1&0&0&0&0&1&1\\ \hline
$e_{6}$  &   &  & & & &0&0&1&2&1&0&2&3&1&1&2&1\\ \hline
$e_{7}$  &   &  & & & & &0&0&0&0&0&1&0&0&1&1&0\\ \hline
$e_{8}$  &   &  & & & & & &0&0&0&1&0&1&0&0&0&0\\ \hline
$e_{9}$  &   &  & & & & & & &0&2&1&2&0&1&2&0&1\\ \hline
$e_{10}$ &   &  & & & & & & & &0&1&0&1&1&3&1&1\\ \hline
$e_{11}$ &   &  & & & & & & & & &0&0&0&0&1&0&0\\ \hline
$e_{12}$ &   &  & & & & & & & & & &0&2&1&2&2&1\\ \hline
$e_{13}$ &   &  & & & & & & & & & & &0&0&2&3&1\\ \hline
$e_{14}$ &   &  & & & & & & & & & & & &0&2&3&2\\ \hline
$e_{15}$ &   &  & & & & & & & & & & & & &0&1&0\\ \hline
$e_{16}$ &   &  & & & & & & & & & & & & & &0&3\\ \hline
$e_{17}$ &   &  & & & & & & & & & & & & & & &0\\ \hline
\end{tabular}

\section{Раскраска графа методом упорядочивания вершин}

\begin{enumerate}

\item Подсчитываем число $r_i$ ненулевых элементов в каждом ряду $i$ в матрице
соединений:

\begin{center}
\begin{longtable}{|c c c c c c c c c c c c c c c c c|}
\hline
$e_{1}$ & $e_{2}$ & $e_{3}$ & $e_{4}$ & $e_{5}$ & $e_{6}$ & $e_{7}$ & $e_{8}$ & $e_{9}$ & $e_{10}$& $e_{11}$& $e_{12}$& $e_{13}$& $e_{14}$& $e_{15}$& $e_{16}$& $e_{17}$ \\
\hline
7 & 11 & 5 & 9 & 5 & 13 & 3 & 4 & 9 & 8 & 7 & 10 & 10 & 9 & 12 & 13 & 11 \\
\hline
\end{longtable}
\end{center}

\item Упорядочим вершины графа в порядке невозрастания $r_i$:

\begin{center}
\begin{longtable}{|c c c c c c c c c c c c c c c c c|}
\hline
$e_{6}$ & $e_{16}$  & $e_{15}$  & $e_{2}$  & $e_{17}$ & $e_{12}$  & $e_{13}$ & $e_{4}$ & $e_{9}$  & $e_{14}$ & $e_{10}$ & $e_{1}$ & $e_{11}$  & $e_{3}$ & $e_{5}$  & $e_{8}$  & $e_{7}$ \\
\hline
13 & 13 & 12 & 11 & 11 & 10 & 10 & 9 & 9 & 9 & 8 & 7 & 7 & 5 & 5 & 4 & 3 \\ \hline
\end{longtable}
\end{center}

% Stopped here

\item Просматривая последовательность слева направо, закрашиваем в некоторый
новый цвет все вершины, которые не смежны ещё окрашенным в этот цвет.

В данном случае это вершины 6 и 5, которые расскрашиывает в один цвет.

\item Удаляем окрашенные рёбра из таблицы.

\item Повторяем до тех пор, пока не останется неокрашенных вершин.
    \begin{enumerate}
        \item Следующий шаг.

\begin{tabular}{|c|c|c|c|c|c|c|c|c|c|c|c|c|c|c|c|c|}
\hline
      & $e_{1}$ & $e_{2}$ & $e_{3}$ & $e_{4}$ & $e_{7}$ & $e_{8}$ & $e_{9}$ & $e_{10}$ & $e_{11}$ & $e_{12}$ & $e_{13}$ & $e_{14}$ & $e_{15}$ & $e_{16}$ & $e_{17}$ & r \\
\hline
$e_{1}$  & 0 & 1 & 0 & 0 &  0 & 1 & 0 & 0 & 2 & 0 & 0 & 0 & 0 & 2 & 1 & 5 \\ \hline
$e_{2}$  &   & 0 & 0 & 2 &  0 & 0 & 1 & 0 & 1 & 3 & 2 & 0 & 1 & 1 & 1 & 9 \\ \hline
$e_{3}$  &   &   & 0 & 0 &  0 & 0 & 0 & 0 & 0 & 0 & 1 & 2 & 1 & 1 & 0 & 4 \\ \hline
$e_{4}$  &   &   &   & 0 &  0 & 0 & 1 & 0 & 0 & 3 & 1 & 1 & 1 & 1 & 3 & 8\\ \hline
$e_{7}$  &   &   &   &   &  0 & 0 & 0 & 0 & 0 & 1 & 0 & 0 & 1 & 1 & 0 & 3\\ \hline
$e_{8}$  &   &   &   &   &    & 0 & 0 & 0 & 1 & 0 & 1 & 0 & 0 & 0 & 0 & 3\\ \hline
$e_{9}$  &   &   &   &   &    &   & 0 & 2 & 1 & 2 & 0 & 1 & 2 & 0 & 1 & 8\\ \hline
$e_{10}$ &   &   &   &   &    &   &   & 0 & 1 & 0 & 1 & 1 & 3 & 1 & 1 & 7\\ \hline
$e_{11}$ &   &   &   &   &    &   &   &   & 0 & 0 & 0 & 0 & 1 & 0 & 0 & 6\\ \hline
$e_{12}$ &   &   &   &   &    &   &   &   &   & 0 & 2 & 1 & 2 & 2 & 1 & 9\\ \hline
$e_{13}$ &   &   &   &   &    &   &   &   &   &   & 0 & 0 & 2 & 3 & 1 & 9\\ \hline
$e_{14}$ &   &   &   &   &    &   &   &   &   &   &   & 0 & 2 & 3 & 2 & 8\\ \hline
$e_{15}$ &   &   &   &   &    &   &   &   &   &   &   &   & 0 & 1 & 0 & 11 \\ \hline
$e_{16}$ &   &   &   &   &    &   &   &   &   &   &   &   &   & 0 & 3 & 11 \\ \hline
$e_{17}$ &   &   &   &   &    &   &   &   &   &   &   &   &   &   & 0 & 9\\ \hline
\end{tabular}

\begin{center}
\begin{longtable}{|c c c c c c c c c c c c c c c|}
\hline
$e_{15}$ & $e_{16}$ & $e_{2}$ & $e_{12}$ & $e_{13}$ & $e_{17}$ & $e_{4}$ & $e_{9}$ & $e_{14}$ & $e_{10}$ & $e_{11}$ & $e_{1}$ & $e_{3}$ & $e_{7}$ & $e_{8}$ \\
\hline
11 & 11 & 9 & 9 & 9 & 9 & 8 & 8 & 8 & 7 & 6 & 5 & 4 & 3 & 3\\
\hline
\end{longtable}
\end{center}

Вершины 15, 1.

        \item Следующий шаг.

\begin{tabular}{|c|c|c|c|c|c|c|c|c|c|c|c|c|c|c|c|c|}
\hline
      & $e_{2}$ & $e_{3}$ & $e_{4}$ & $e_{7}$ & $e_{8}$ & $e_{9}$ & $e_{10}$ & $e_{11}$ & $e_{12}$ & $e_{13}$ & $e_{14}$ & $e_{16}$ & $e_{17}$ & r \\
\hline
$e_{2}$  &   0 & 0 & 2 &  0 & 0 & 1 & 0 & 1 & 3 & 2 & 0 & 1 & 1 & 7 \\ \hline
$e_{3}$  &     & 0 & 0 &  0 & 0 & 0 & 0 & 0 & 0 & 1 & 2 & 1 & 0 & 3 \\ \hline
$e_{4}$  &     &   & 0 &  0 & 0 & 1 & 0 & 0 & 3 & 1 & 1 & 1 & 3 & 7\\ \hline
$e_{7}$  &     &   &   &  0 & 0 & 0 & 0 & 0 & 1 & 0 & 0 & 1 & 0 & 2\\ \hline
$e_{8}$  &     &   &   &    & 0 & 0 & 0 & 1 & 0 & 1 & 0 & 0 & 0 & 2\\ \hline
$e_{9}$  &     &   &   &    &   & 0 & 2 & 1 & 2 & 0 & 1 & 0 & 1 & 5\\ \hline
$e_{10}$ &     &   &   &    &   &   & 0 & 1 & 0 & 1 & 1 & 1 & 1 & 6\\ \hline
$e_{11}$ &     &   &   &    &   &   &   & 0 & 0 & 0 & 0 & 0 & 0 & 4\\ \hline
$e_{12}$ &     &   &   &    &   &   &   &   & 0 & 2 & 1 & 2 & 1 & 8\\ \hline
$e_{13}$ &     &   &   &    &   &   &   &   &   & 0 & 0 & 3 & 1 & 8\\ \hline
$e_{14}$ &     &   &   &    &   &   &   &   &   &   & 0 & 3 & 2 & 7\\ \hline
$e_{16}$ &     &   &   &    &   &   &   &   &   &   &   & 0 & 3 & 9\\ \hline
$e_{17}$ &     &   &   &    &   &   &   &   &   &   &   &   & 0 & 8\\ \hline
\end{tabular}

\begin{center}
\begin{longtable}{|c c c c c c c c c c c c c |}
\hline
$e_{16}$ & $e_{12}$ & $e_{13}$ & $e_{17}$ & $e_{2}$ & $e_{4}$ & $e_{14}$ & $e_{10}$ & $e_{9}$ & $e_{11}$ & $e_{3}$ & $e_{7}$ & $e_{8}$ \\
\hline
9 & 8 & 8 & 8 & 7 & 7 & 7 & 6 & 5 & 4 & 3 & 2 & 2\\
\hline
\end{longtable}
\end{center}

Вершины 16, 9, 8.

        \item Следующий шаг.

\begin{tabular}{|c|c|c|c|c|c|c|c|c|c|c|c|c|c|}
\hline
      & $e_{2}$ & $e_{3}$ & $e_{4}$ & $e_{7}$ & $e_{10}$ & $e_{11}$ & $e_{12}$ & $e_{13}$ & $e_{14}$ & $e_{17}$ & r \\
\hline
$e_{2}$  &   0 & 0 & 2 &  0 & 0 & 1 & 3 & 2 & 0 & 1 & 5 \\ \hline
$e_{3}$  &     & 0 & 0 &  0 & 0 & 0 & 0 & 1 & 2 & 0 & 2 \\ \hline
$e_{4}$  &     &   & 0 &  0 & 0 & 0 & 3 & 1 & 1 & 3 & 5\\ \hline
$e_{7}$  &     &   &   &  0 & 0 & 0 & 1 & 0 & 0 & 0 & 1\\ \hline
$e_{10}$ &     &   &   &    & 0 & 1 & 0 & 1 & 1 & 1 & 5\\ \hline
$e_{11}$ &     &   &   &    &   & 0 & 0 & 0 & 0 & 0 & 2\\ \hline
$e_{12}$ &     &   &   &    &   &   & 0 & 2 & 1 & 1 & 6\\ \hline
$e_{13}$ &     &   &   &    &   &   &   & 0 & 0 & 1 & 6\\ \hline
$e_{14}$ &     &   &   &    &   &   &   &   & 0 & 2 & 5\\ \hline
$e_{17}$ &     &   &   &    &   &   &   &   &   & 0 & 6\\ \hline
\end{tabular}

\begin{center}
\begin{longtable}{|c c c c c c c c c c|}
\hline
$e_{12}$ & $e_{13}$ & $e_{17}$ & $e_{2}$ & $e_{4}$ & $e_{10}$ & $e_{14}$ & $e_{3}$ & $e_{11}$ & $e_{7}$ \\
\hline
6 & 6 & 6 & 5 & 5 & 5 & 5 & 2 & 2 & 1 \\
\hline
\end{longtable}
\end{center}

Вершины 12, 10, 3, 7.
        \item Следующий шаг.

\begin{tabular}{|c|c|c|c|c|c|c|c|c|c|}
\hline
      & $e_{2}$ & $e_{4}$ & $e_{11}$ & $e_{13}$ & $e_{14}$ & $e_{17}$ & r \\
\hline
$e_{2}$  &   0 &  2 & 1 & 2 & 0 & 1 & 4 \\ \hline
$e_{4}$  &     &  0 & 0 & 1 & 1 & 3 & 4\\ \hline
$e_{11}$ &     &    & 0 & 0 & 0 & 0 & 1\\ \hline
$e_{13}$ &     &    &   & 0 & 0 & 1 & 3\\ \hline
$e_{14}$ &     &    &   &   & 0 & 2 & 2\\ \hline
$e_{17}$ &     &    &   &   &   & 0 & 4\\ \hline
\end{tabular}

\begin{center}
\begin{longtable}{|c c c c c c|}
\hline
$e_{2}$ & $e_{4}$ & $e_{17}$ & $e_{13}$ & $e_{11}$ & $e_{14}$ \\
\hline
4 & 4 & 4 & 3 & 2 & 1 \\
\hline
\end{longtable}
\end{center}

Вершины 2, 14.

        \item Следующий шаг.

\begin{tabular}{|c|c|c|c|c|c|c|c|}
\hline
      & $e_{4}$ & $e_{11}$ & $e_{13}$ & $e_{17}$ & r \\
\hline
$e_{4}$  & 0 & 0 & 1 & 3 & 2\\ \hline
$e_{11}$ &   & 0 & 0 & 0 & 0\\ \hline
$e_{13}$ &   &   & 0 & 1 & 2\\ \hline
$e_{17}$ &   &   &   & 0 & 2\\ \hline
\end{tabular}

\begin{center}
\begin{longtable}{|c c c c|}
\hline
$e_{4}$ & $e_{13}$ & $e_{17}$ & $e_{11}$ \\
\hline
2 & 2 & 2 & 0\\
\hline
\end{longtable}
\end{center}

Вершины 4, 11.

        \item Следующий шаг.

\begin{tabular}{|c|c|c|c|c|c|c|c|}
\hline
      & $e_{13}$ & $e_{17}$ & r \\
\hline
$e_{13}$ & 0 & 1 & 1\\ \hline
$e_{17}$ &   & 0 & 1\\ \hline
\end{tabular}

Вершина 13.
    \item Следующий шаг.
Вершина 17.
\end{enumerate}

Таким образом, получаем следующую раскраску.
\begin{description}
    \item[Цвет 1:] 15, 1.
    \item[Цвет 2:] 16, 9, 8.
    \item[Цвет 3:] 12, 10, 3, 7.
    \item[Цвет 4:] 2, 14.
    \item[Цвет 5:] 4, 11.
    \item[Цвет 6:] 13.
    \item[Цвет 7:] 17.
\end{description}

\end{enumerate}

\section{Размещение элементов методом обратного размещения}

Зададим поверхность:

\tikzstyle{p} = [draw, thick, minimum width=3em, minimum height=3em,
  node distance=4em]
\begin{tikzpicture}
\node[p]                 (p1)  {$p_{1}$};
\node[p, right of = p1]  (p2)  {$p_{2}$};
\node[p, right of = p2]  (p3)  {$p_{3}$};
\node[p, right of = p3]  (p4)  {$p_{4}$};
\node[p, right of = p4]  (p5)  {$p_{5}$};
\node[p, right of = p5]  (p6)  {$p_{6}$};
\node[p, below of = p1]  (p7)  {$p_{7}$};
\node[p, right of = p7]  (p8)  {$p_{8}$};
\node[p, right of = p8]  (p9)  {$p_{9}$};
\node[p, right of = p9]  (p10) {$p_{10}$};
\node[p, right of = p10] (p11) {$p_{11}$};
\node[p, right of = p11] (p12) {$p_{12}$};
\node[p, below of = p7]  (p13) {$p_{13}$};
\node[p, right of = p13] (p14) {$p_{14}$};
\node[p, right of = p14] (p15) {$p_{15}$};
\node[p, right of = p15] (p16) {$p_{16}$};
\node[p, right of = p16] (p17) {$p_{17}$};
\end{tikzpicture}

Матрица $D$ расстояний между позициями для размещения.

\begin{tabular}{c|c|c|c|c|c|c|c|c|c|c|c|c|c|c|c|c|c}
\hline
      & $p_{1}$ & $p_{2}$ & $p_{3}$ & $p_{4}$ & $p_{5}$ & $p_{6}$ & $p_{7}$ & $p_{8}$ & $p_{9}$ & $p_{10}$ & $p_{11}$ & $p_{12}$ & $p_{13}$ & $p_{14}$ & $p_{15}$ & $p_{16}$ & $p_{17}$ \\
\hline
$p_{1}$  & 0 & 1 & 2 & 3 & 4 & 5 & 1 & 2 & 3 & 4 & 5 & 6 & 2 & 3 & 4 & 5 & 6 \\ \hline
$p_{2}$  & 1 & 0 & 1 & 2 & 3 & 4 & 2 & 1 & 2 & 3 & 4 & 5 & 3 & 2 & 3 & 4 & 5 \\ \hline
$p_{3}$  & 2 & 1 & 0 & 1 & 2 & 3 & 3 & 2 & 1 & 2 & 3 & 4 & 4 & 3 & 2 & 3 & 4 \\ \hline
$p_{4}$  & 3 & 2 & 1 & 0 & 1 & 2 & 4 & 3 & 2 & 1 & 2 & 3 & 5 & 4 & 3 & 2 & 3 \\ \hline
$p_{5}$  & 4 & 3 & 2 & 1 & 0 & 1 & 5 & 4 & 3 & 2 & 1 & 2 & 6 & 5 & 4 & 3 & 2 \\ \hline
$p_{6}$  & 5 & 4 & 3 & 2 & 1 & 0 & 6 & 5 & 4 & 3 & 2 & 1 & 7 & 6 & 5 & 4 & 3 \\ \hline
$p_{7}$  & 1 & 2 & 3 & 4 & 5 & 6 & 0 & 1 & 2 & 3 & 4 & 5 & 1 & 2 & 3 & 4 & 5 \\ \hline
$p_{8}$  & 2 & 1 & 2 & 3 & 4 & 5 & 1 & 0 & 1 & 2 & 3 & 4 & 2 & 1 & 2 & 3 & 4 \\ \hline
$p_{9}$  & 3 & 2 & 1 & 2 & 3 & 4 & 2 & 1 & 0 & 1 & 2 & 3 & 3 & 2 & 1 & 2 & 3 \\ \hline
$p_{10}$ & 4 & 3 & 2 & 1 & 2 & 3 & 3 & 2 & 1 & 0 & 1 & 2 & 4 & 3 & 2 & 1 & 2 \\ \hline
$p_{11}$ & 5 & 4 & 3 & 2 & 1 & 2 & 4 & 3 & 2 & 1 & 0 & 1 & 5 & 4 & 3 & 2 & 1 \\ \hline
$p_{12}$ & 6 & 5 & 4 & 3 & 2 & 1 & 5 & 4 & 3 & 2 & 1 & 0 & 6 & 5 & 4 & 3 & 2 \\ \hline
$p_{13}$ & 2 & 3 & 4 & 5 & 6 & 7 & 1 & 2 & 3 & 4 & 5 & 6 & 0 & 1 & 2 & 3 & 4 \\ \hline
$p_{14}$ & 3 & 2 & 3 & 4 & 5 & 6 & 2 & 1 & 2 & 3 & 4 & 5 & 1 & 0 & 1 & 2 & 3 \\ \hline
$p_{15}$ & 4 & 3 & 2 & 3 & 4 & 5 & 3 & 2 & 1 & 2 & 3 & 4 & 2 & 1 & 0 & 1 & 2 \\ \hline
$p_{16}$ & 5 & 4 & 3 & 2 & 3 & 4 & 4 & 3 & 2 & 1 & 2 & 3 & 3 & 2 & 1 & 0 & 1 \\ \hline
$p_{17}$ & 6 & 5 & 4 & 3 & 2 & 3 & 5 & 4 & 3 & 2 & 1 & 2 & 4 & 3 & 2 & 1 & 0 \\ \hline
\end{tabular}


Подсчитываем число $r_i$ ненулевых элементов в каждом ряду $i$ в матрице
соединений:
\begin{longtable}{|c c c c c c c c c c c c c c c c c|}
\hline
$e_{1}$ & $e_{2}$ & $e_{3}$ & $e_{4}$ & $e_{5}$ & $e_{6}$ & $e_{7}$ & $e_{8}$ & $e_{9}$ & $e_{10}$& $e_{11}$& $e_{12}$& $e_{13}$& $e_{14}$& $e_{15}$& $e_{16}$& $e_{17}$ \\
\hline
7 & 11 & 5 & 9 & 5 & 13 & 3 & 4 & 9 & 8 & 7 & 10 & 10 & 9 & 12 & 13 & 11 \\
\hline
\end{longtable}
Порядок позиций по неубыванию суммы величин $D$. 

\begin{center}
\begin{longtable}{|c c c c c c c c c c c c c c c c c|}
\hline
$p_{1}$ & $p_{2}$ & $p_{3}$ & $p_{4}$ & $p_{5}$ & $p_{6}$ & $p_{7}$ & $p_{8}$ & $p_{9}$ & $p_{10}$& $p_{11}$& $p_{12}$& $p_{13}$& $p_{14}$& $p_{15}$& $p_{16}$& $p_{17}$ \\
\hline
63 & 51 & 45 & 45 & 51 & 63 & 57 & 45 & 39 & 39 & 45 & 57 & 63 & 51 & 45 & 45 & 51 \\
\hline
\end{longtable}
\end{center}

Упорядочиваем:

\begin{center}
\begin{longtable}{|c c c c c c c c c c c c c c c c c|}
\hline
$e_{6}$ & $e_{16}$  & $e_{15}$  & $e_{2}$  & $e_{17}$ & $e_{12}$  & $e_{13}$ & $e_{4}$ & $e_{9}$  & $e_{14}$ & $e_{10}$ & $e_{1}$ & $e_{11}$  & $e_{3}$ & $e_{5}$  & $e_{8}$  & $e_{7}$ \\
\hline
13 & 13 & 12 & 11 & 11 & 10 & 10 & 9 & 9 & 9 & 8 & 7 & 7 & 5 & 5 & 4 & 3 \\ \hline
\end{longtable}
\end{center}

\begin{longtable}{|c c c c c c c c c c c c c c c c c|}
\hline
$p_{10}$   & $p_{9}$  & $p_{11}$   & $p_{15}$  & $p_{16}$   & $p_{3}$   & $p_{4}$   & $p_{8}$  & $p_{14}$  & $p_{17}$  & $p_{2}$   & $p_{5}$   & $p_{12}$   & $p_{7}$   & $p_{1}$  & $p_{13}$  & $p_{6}$  \\
\hline
39 & 39 & 45 & 45 & 45 & 45 & 45 & 45 & 51 & 51 & 51 & 51 & 57 & 57 & 63 & 63 & 63 \\
\hline
\end{longtable}

Таким образом, искомое размещение:

\begin{longtable}{|c|c|c|c|c|c|c|c|c|c|c|c|c|c|c|c|c|}
\hline
39 & 39 & 45 & 45 & 45 & 45 & 45 & 45 & 51 & 51 & 51 & 51 & 57 & 57 & 63 & 63 & 63 \\
$p_{10}$ & $p_{9}$  & $p_{11}$ & $p_{15}$ & $p_{16}$ & $p_{3}$  & $p_{4}$  & $p_{8}$ & $p_{14}$ & $p_{17}$ & $p_{2}$  & $p_{5}$ & $p_{12}$ & $p_{7}$  & $p_{1}$ & $p_{13}$ & $p_{6}$ \\
$e_{6}$  & $e_{16}$ & $e_{15}$ & $e_{2}$  & $e_{17}$ & $e_{12}$ & $e_{13}$ & $e_{4}$ & $e_{9}$  & $e_{14}$ & $e_{10}$ & $e_{1}$ & $e_{11}$  & $e_{3}$ & $e_{5}$ & $e_{8}$  & $e_{7}$ \\
13 & 13 & 12 & 11 & 11 & 10 & 10 & 9 & 9 & 9 & 8 & 7 & 7 & 5 & 5 & 4 & 3 \\ 
\hline
\end{longtable}

Функционал данного размещения: $F = \frac 1 2 \sum_i \sum_j d_{ij} r_{ij} = 338$. \\


На рисунке не представлены линии соединений, так как они слишком сильно
засорили бы изображение. 

\tikzstyle{p} = [draw, thick, minimum width=3em, minimum height=3em,
  node distance=4em]
\begin{center}
    \begin{tikzpicture}
        \node[p]                 (p1)  {$e_{5}$};
        \node[p, right of = p1]  (p2)  {$e_{10}$};
        \node[p, right of = p2]  (p3)  {$e_{12}$};
        \node[p, right of = p3]  (p4)  {$e_{13}$};
        \node[p, right of = p4]  (p5)  {$e_{1}$};
        \node[p, right of = p5]  (p6)  {$e_{7}$};
        \node[p, below of = p1]  (p7)  {$e_{3}$};
        \node[p, right of = p7]  (p8)  {$e_{4}$};
        \node[p, right of = p8]  (p9)  {$e_{16}$};
        \node[p, right of = p9]  (p10) {$e_{6}$};
        \node[p, right of = p10] (p11) {$e_{15}$};
        \node[p, right of = p11] (p12) {$e_{11}$};
        \node[p, below of = p7]  (p13) {$e_{8}$};
        \node[p, right of = p13] (p14) {$e_{9}$};
        \node[p, right of = p14] (p15) {$e_{2}$};
        \node[p, right of = p15] (p16) {$e_{17}$};
        \node[p, right of = p16] (p17) {$e_{14}$};
    \end{tikzpicture}
\end{center}

\section{Поиск кратчайших путей}

Матрица весов: $c_{ij} = r_{ij} d_{ij}$: \\
\begin{tabular}{|c|c|c|c|c|c|c|c|c|c|c|c|c|c|c|c|c|c|}
\hline
      & $c_{1}$ & $c_{2}$ & $c_{3}$ & $c_{4}$ & $c_{5}$ & $c_{6}$ & $c_{7}$ & $c_{8}$ & $c_{9}$ & $c_{10}$ & $c_{11}$ & $c_{12}$ & $c_{13}$ & $c_{14}$ & $c_{15}$ & $c_{16}$ & $c_{17}$ \\
\hline
$c_{1}$   &   & 1 &   &   & 4 & 5 &   & 2 &   &   & 10 &   &   &   &   & 10 & 6 \\ \hline
$c_{2}$   & 1 &   &   & 4 & 6 & 8 &   &   & 2 &   & 4 & 15 & 6 &   & 3 & 4 & 5 \\ \hline
$c_{3}$   &   &   &   &   &   & 3 &   &   &   &   &   &   & 4 & 6 & 2 & 3 &   \\ \hline
$c_{4}$   &   & 4 &   &   &   & 2 &   &   & 2 &   &   & 9 & 5 & 4 & 3 & 2 & 9 \\ \hline
$c_{5}$   & 4 & 6 &   &   &   &   &   &   &   &   & 1 &   &   &   &   & 3 & 2 \\ \hline
$c_{6}$   & 5 & 8 & 3 & 2 &   &   &   & 5 & 8 & 3 &   & 2 & 21 & 6 & 5 & 8 & 3 \\ \hline
$c_{7}$   &   &   &   &   &   &   &   &   &   &   &   & 5 &   &   & 3 & 4 &   \\ \hline
$c_{8}$   & 2 &   &   &   &   & 5 &   &   &   &   & 3 &   & 2 &   &   &   &   \\ \hline
$c_{9}$   &   & 2 &   & 2 &   & 8 &   &   &   & 2 & 2 & 6 &   & 2 & 2 &   & 3 \\ \hline
$c_{10}$  &   &   &   &   &   & 3 &   &   & 2 &   & 1 &   & 4 & 3 & 6 & 1 & 2 \\ \hline
$c_{11}$  & 10 & 4 &   &   & 1 &   &   & 3 & 2 & 1 &   &   &   &   & 3 &   &   \\ \hline
$c_{12}$  &   & 15 &   & 9 &   & 2 & 5 &   & 6 &   &   &   & 12 & 5 & 8 & 6 & 2 \\ \hline
$c_{13}$  &   & 6 & 4 & 5 &   & 21 &   & 2 &   & 4 &   & 12 &   &   & 4 & 9 & 4 \\ \hline
$c_{14}$  &   &   & 6 & 4 &   & 6 &   &   & 2 & 3 &   & 5 &   &   & 2 & 6 & 6 \\ \hline
$c_{15}$  &   & 3 & 2 & 3 &   & 5 & 3 &   & 2 & 6 & 3 & 8 & 4 & 2 &   & 1 &   \\ \hline
$c_{16}$  & 10 & 4 & 3 & 2 & 3 & 8 & 4 &   &   & 1 &   & 6 & 9 & 6 & 1 &   & 3 \\ \hline
$c_{17}$  & 6 & 5 &   & 9 & 2 & 3 &   &   & 3 & 2 &   & 2 & 4 & 6 &   & 3 &   \\ \hline
\end{tabular}

Найдём кратчайшие пути от вершины $1$ до всех остальных.

\newcommand{\z}[0]{$\infty$}

%\begin{tabular}{|c|c|c|c|c|c|c|c|c|c|c|c|c|c|c|c|c|c|}
%\hline
%   Шаг   & 0 & 1 & 2 & 3 & 4 & & & & & & & & & & & &\\
%\hline
%$e_{1}$  &$0^+$&      &      &      &     &     &     &     &     &     &     &      &     &     &     &     &  \\ \hline
%$e_{2}$  & \z  & $1^+$&      &      &     &     &     &     &     &     &     &      &     &     &     &     &  \\ \hline
%$e_{3}$  & \z  & \z   & 4    & 4    & 4   &  4  & 4   &  4  & 4   & 4   & 3   &$2^+$ &     &     &     &     &  \\ \hline
%$e_{4}$  & \z  & \z   & 5    & 5    & 5   &$2^+$&     &     &     &     &     &      &     &     &     &     &  \\ \hline
%$e_{5}$  & \z  & 4    & 4    & 4    & 4   &  4  & 4   &  4  &$1^+$&     &     &      &     &     &     &     &  \\ \hline
%$e_{6}$  & \z  & 5    & 5    & 5    & 5   &  5  & 5   &  3  & 3   &  3  & 3   &  3   &  3  &  3  & 3   &$2^+$   &  \\ \hline
%$e_{7}$  & \z  & \z   & \z   & \z   & \z  & \z  & \z  & \z  &\z   & \z  & 4   &  3   &  3  &  3  & 3   & 3   &$3^+$  \\ \hline
%$e_{8}$  & \z  & 2    & $2^+$&      &     &     &     &     &     &     &     &      &     &     &     &     &  \\ \hline
%$e_{9}$  & \z  & \z   &  2   & 2    &$2^+$&     &     &     &     &     &     &      &     &     &     &     &  \\ \hline
%$e_{10}$ & \z  & \z   & \z   & \z   & 4   &  2  &$2^+$&     &     &     &     &      &     &     &     &     &  \\ \hline
%$e_{11}$ & \z  & 10   & 4    & 3    &  3  &  2  & 2   &$1^+$&     &     &     &      &     &     &     &     &  \\ \hline
%$e_{12}$ & \z  & \z   & 15   & 15   & 12  &  6  & 6   &  6  & 6   &  6  & 6   &  6   &  6  &  5  &$2^+$&     &  \\ \hline
%$e_{13}$ & \z  & \z   & 6    & $2^+$&     &     &     &     &     &     &     &      &     &     &     &     &  \\ \hline
%$e_{14}$ & \z  & \z   & \z   & \z   &\z   & 2   & 4   &  3  & 3   &  3  & 3   &  2   &$2^+$&     &     &     &  \\ \hline
%$e_{15}$ & \z  & \z   & 3    & 3    & 3   & 2   & 2   &  2  & 2   &  2  &$1^+$&      &     &     &     &     &  \\ \hline
%$e_{16}$ & \z  & 10   & 4    & 4    & 4   & 4   & 2   &  1  & 1   &$1^+$&     &      &     &     &     &     &  \\ \hline
%$e_{17}$ & \z  & 6    & 5    & 5    & 4   & 3   & 3   &  2  & 2   &  2  & 2   &  2   &  2  &$2^+$&     &     &  \\ \hline
%\end{tabular}
\begin{enumerate}
    \item $l(c_1) = 0^+; l(c_i) =$ \z  для всех $ i \neq 1, p = c_1$. 

\begin{tabular} {|c|c|}
    \hline
    $c_{1}$  &$0^+$\\ \hline
    $c_{2}$  & \z  \\ \hline
    $c_{3}$  & \z  \\ \hline
    $c_{4}$  & \z  \\ \hline
    $c_{5}$  & \z  \\ \hline
    $c_{6}$  & \z  \\ \hline
    $c_{7}$  & \z  \\ \hline
    $c_{8}$  & \z  \\ \hline
    $c_{9}$  & \z  \\ \hline
    $c_{10}$ & \z  \\ \hline
    $c_{11}$ & \z  \\ \hline
    $c_{12}$ & \z  \\ \hline
    $c_{13}$ & \z  \\ \hline
    $c_{14}$ & \z  \\ \hline
    $c_{15}$ & \z  \\ \hline
    $c_{16}$ & \z  \\ \hline
    $c_{17}$ & \z  \\ \hline 
\end{tabular}

    \item $\Gamma p = \{ c_2, c_5, c_6, c_8, c_11, c_16, c_17\}$. Уточним временные пометки.  
        $$l(c_2)    = min(\infty, 0^+ + 1) = 1$$
        $$l(c_5)    = min(\infty, 0^+ + 4) = 4$$
        $$l(c_6)    = min(\infty, 0^+ + 5) = 5$$
        $$l(c_8)    = min(\infty, 0^+ + 2) = 2$$
        $$l(c_{11}) = min(\infty, 0^+ + 10) = 10$$
        $$l(c_{16}) = min(\infty, 0^+ + 10) = 10$$
        $$l(c_{17}) = min(\infty, 0^+ + 6) = 6$$
        $$l(c^*) = min(l(c_i)) = l(c_2) = 1$$
\begin{tabular} {|c|c|c|}
    \hline
    $c_{1}$  &$0^+$&      \\ \hline
    $c_{2}$  & \z  &$1^+$ \\ \hline
    $c_{3}$  & \z  & \z   \\ \hline
    $c_{4}$  & \z  & \z   \\ \hline
    $c_{5}$  & \z  & 4    \\ \hline
    $c_{6}$  & \z  & 5    \\ \hline
    $c_{7}$  & \z  & \z   \\ \hline
    $c_{8}$  & \z  & 2    \\ \hline
    $c_{9}$  & \z  & \z   \\ \hline
    $c_{10}$ & \z  & \z   \\ \hline
    $c_{11}$ & \z  & 10   \\ \hline
    $c_{12}$ & \z  & \z   \\ \hline
    $c_{13}$ & \z  & \z   \\ \hline
    $c_{14}$ & \z  & \z   \\ \hline
    $c_{15}$ & \z  & \z   \\ \hline
    $c_{16}$ & \z  & 10   \\ \hline
    $c_{17}$ & \z  & 6    \\ \hline 
\end{tabular}

    \item $\Gamma p = \{c_4, c_5, c_6, c_9, c_{11}, c_{12}, c_{13}, c_{15}, c_{16}, c_{17}\}$.
        $$l(c_4) = min(\infty, 1^+ + 4) = 5$$
        $$l(c_5) = min(4, 1^+ + 6) = 4$$
        $$l(c_6) = min(5, 1^+ + 8) = 5$$
        $$l(c_9) = min(\infty, 1^+ + 2) = 3$$
        $$l(c_{11}) = min(10, 1^+ + 4) = 5$$
        $$l(c_{12}) = min(\infty, 1^+ + 15) = 16$$
        $$l(c_{13}) = min(\infty, 1^+ + 6) = 7$$
        $$l(c_{15}) = min(\infty, 1^+ + 3) = 4$$
        $$l(c_{16}) = min(10, 1^+ + 4) = 5$$
        $$l(c_{17}) = min(6, 1^+ + 5) = 6$$
        $$l(min(c_i)) = l(c_9) = 3$$

\begin{tabular} {|c|c|c|c|}
    \hline
    $c_{1}$  &$0^+$&      &     \\ \hline
    $c_{2}$  & \z  &$1^+$ &     \\ \hline
    $c_{3}$  & \z  & \z   & \z  \\ \hline
    $c_{4}$  & \z  & \z   & 5   \\ \hline
    $c_{5}$  & \z  & 4    & 4   \\ \hline
    $c_{6}$  & \z  & 5    & 5   \\ \hline
    $c_{7}$  & \z  & \z   & \z  \\ \hline
    $c_{8}$  & \z  & 2    & \z  \\ \hline
    $c_{9}$  & \z  & \z   &$3^+$\\ \hline
    $c_{10}$ & \z  & \z   & \z  \\ \hline
    $c_{11}$ & \z  & 10   &  5  \\ \hline
    $c_{12}$ & \z  & \z   & 16  \\ \hline
    $c_{13}$ & \z  & \z   & 7   \\ \hline
    $c_{14}$ & \z  & \z   & \z  \\ \hline
    $c_{15}$ & \z  & \z   & 4   \\ \hline
    $c_{16}$ & \z  & 10   & 5   \\ \hline
    $c_{17}$ & \z  & 6    & 6   \\ \hline 
\end{tabular}

    \item $\Gamma p = \{ c_4, c_6, c_{10}, c_{11}, c_{12}, c_{14}, c_{15}, c_{17}\}$.
        $$l(c_4)    = min(5, 3^+ + 2) = 5 $$
        $$l(c_6)    = min(5, 3^+ + 8) = 5 $$
        $$l(c_{10}) = min(\infty, 3^+ + 2) = 5 $$
        $$l(c_{11}) = min(5, 3^+ + 2) = 5 $$
        $$l(c_{12}) = min(16, 3^+ + ) = 9 $$
        $$l(c_{14}) = min(\infty, 3^+ + ) = 5 $$
        $$l(c_{15}) = min(4, 3^+ + 2) = 4 $$
        $$l(c_{17}) = min(6, 3^+ + 3) = 6 $$
\begin{tabular} {|c|c|c|c|c|}
    \hline
    $c_{1}$  &$0^+$&      &     &    \\ \hline
    $c_{2}$  & \z  &$1^+$ &     &    \\ \hline
    $c_{3}$  & \z  & \z   & \z  &\z   \\ \hline
    $c_{4}$  & \z  & \z   & 5   & 5    \\ \hline
    $c_{5}$  & \z  & 4    & 4   & 4   \\ \hline
    $c_{6}$  & \z  & 5    & 5   & 5  \\ \hline
    $c_{7}$  & \z  & \z   & \z  & \z   \\ \hline
    $c_{8}$  & \z  & 2    & \z  & \z   \\ \hline
    $c_{9}$  & \z  & \z   &$3^+$&     \\ \hline
    $c_{10}$ & \z  & \z   & \z  & 5   \\ \hline
    $c_{11}$ & \z  & 10   &  5  & 5   \\ \hline
    $c_{12}$ & \z  & \z   & 16  & 9   \\ \hline
    $c_{13}$ & \z  & \z   & 7   & 7   \\ \hline
    $c_{14}$ & \z  & \z   & \z  & 5   \\ \hline
    $c_{15}$ & \z  & \z   & 4   &$4^+$   \\ \hline
    $c_{16}$ & \z  & 10   & 5   & 5   \\ \hline
    $c_{17}$ & \z  & 6    & 6   & 6   \\ \hline 
\end{tabular}

    \item $\Gamma p = \{ c_3, c_4, c_6, c_7, c_{10}, c_{11}, c_{12}, c_{13}, c_{14}, c_{16}\}$
        $$l(c_3) = min(\infty, 4^+ + 2) = 6$$
        $$l(c_4) = min(5, 4^+ + 3) = 5$$
        $$l(c_6) = min(5, 4^+ + 5) = 5$$
        $$l(c_7) = min(\infty, 4^+ + 3) = 7$$
        $$l(c_{10}) = min(5, 4^+ + 6) = 5$$
        $$l(c_{11}) = min(5, 4^+ + 3) = 5$$
        $$l(c_{12}) = min(9, 4^+ + 8) = 9$$
        $$l(c_{13}) = min(7, 4^+ + 4) = 7$$
        $$l(c_{14}) = min(5, 4^+ + 2) = 5$$
        $$l(c_{16}) = min(5, 4^+ + 1) = 5$$

\begin{tabular} {|c|c|c|c|c|c|}
    \hline
    $c_{1}$  &$0^+$&      &     &     &        \\ \hline
    $c_{2}$  & \z  &$1^+$ &     &     &        \\ \hline
    $c_{3}$  & \z  & \z   & \z  &\z   & 6      \\ \hline
    $c_{4}$  & \z  & \z   & 5   & 5   & 5      \\ \hline
    $c_{5}$  & \z  & 4    & 4   & 4   &$4^+$      \\ \hline
    $c_{6}$  & \z  & 5    & 5   & 5   & 5      \\ \hline
    $c_{7}$  & \z  & \z   & \z  & \z  & 7      \\ \hline
    $c_{8}$  & \z  & 2    & \z  & \z  & \z     \\ \hline
    $c_{9}$  & \z  & \z   &$3^+$&     &        \\ \hline
    $c_{10}$ & \z  & \z   & \z  & 5   & 5      \\ \hline
    $c_{11}$ & \z  & 10   &  5  & 5   & 5      \\ \hline
    $c_{12}$ & \z  & \z   & 16  & 9   & 9      \\ \hline
    $c_{13}$ & \z  & \z   & 7   & 7   & 7      \\ \hline
    $c_{14}$ & \z  & \z   & \z  & 5   & 5      \\ \hline
    $c_{15}$ & \z  & \z   & 4   &$4^+$&        \\ \hline
    $c_{16}$ & \z  & 10   & 5   & 5   & 5      \\ \hline
    $c_{17}$ & \z  & 6    & 6   & 6   & 6      \\ \hline 
\end{tabular}

    \item $\Gamma p = \{ c_{11}, c_{16}, c_{17}\}$
        $$l(c_{11}) = min(5, 4^+ + 1) = 5$$
        $$l(c_{16}) = min(5, 4^+ + 3) = 5$$
        $$l(c_{17}) = min(6, 4^+ + 2) = 6$$
\begin{tabular} {|c|c|c|c|c|c|c|}
    \hline
    $c_{1}$  &$0^+$&      &     &     &     &       \\ \hline
    $c_{2}$  & \z  &$1^+$ &     &     &     &       \\ \hline
    $c_{3}$  & \z  & \z   & \z  &\z   & 6   &  6    \\ \hline
    $c_{4}$  & \z  & \z   & 5   & 5   & 5   &$5^+$    \\ \hline
    $c_{5}$  & \z  & 4    & 4   & 4   &$4^+$&          \\ \hline
    $c_{6}$  & \z  & 5    & 5   & 5   & 5   &  5    \\ \hline
    $c_{7}$  & \z  & \z   & \z  & \z  & 7   &  7    \\ \hline
    $c_{8}$  & \z  & 2    & \z  & \z  & \z  &  \z   \\ \hline
    $c_{9}$  & \z  & \z   &$3^+$&     &     &       \\ \hline
    $c_{10}$ & \z  & \z   & \z  & 5   & 5   &  5    \\ \hline
    $c_{11}$ & \z  & 10   &  5  & 5   & 5   &  5    \\ \hline
    $c_{12}$ & \z  & \z   & 16  & 9   & 9   &  9    \\ \hline
    $c_{13}$ & \z  & \z   & 7   & 7   & 7   &  7    \\ \hline
    $c_{14}$ & \z  & \z   & \z  & 5   & 5   &  5    \\ \hline
    $c_{15}$ & \z  & \z   & 4   &$4^+$&     &       \\ \hline
    $c_{16}$ & \z  & 10   & 5   & 5   & 5   &  5    \\ \hline
    $c_{17}$ & \z  & 6    & 6   & 6   & 6   &  6    \\ \hline 
\end{tabular}

    \item $\Gamma p = \{ c_6, c_{12}, c_{13}, c_{14}, c_{16}, c_{17}\}$
        $$l(c_6) = min(5, 5^+ + 2) = 5$$
        $$l(c_{12}) = min(9, 5^+ + 9) = 9$$
        $$l(c_{13}) = min(7, 5^+ + 5) = 7$$
        $$l(c_{14}) = min(5, 5^+ + 4) = 5$$
        $$l(c_{16}) = min(5, 5^+ + 2) = 5$$
        $$l(c_{17}) = min(6, 5^+ + 9) = 6$$

\begin{tabular} {|c|c|c|c|c|c|c|c|}
    \hline
    $c_{1}$  &$0^+$&      &     &     &     &     &      \\ \hline
    $c_{2}$  & \z  &$1^+$ &     &     &     &     &      \\ \hline
    $c_{3}$  & \z  & \z   & \z  &\z   & 6   &  6  &  6   \\ \hline
    $c_{4}$  & \z  & \z   & 5   & 5   & 5   &$5^+$&        \\ \hline
    $c_{5}$  & \z  & 4    & 4   & 4   &$4^+$&     &         \\ \hline
    $c_{6}$  & \z  & 5    & 5   & 5   & 5   &  5  &$5^+$     \\ \hline
    $c_{7}$  & \z  & \z   & \z  & \z  & 7   &  7  &  7   \\ \hline
    $c_{8}$  & \z  & 2    & \z  & \z  & \z  &  \z & \z   \\ \hline
    $c_{9}$  & \z  & \z   &$3^+$&     &     &     &      \\ \hline
    $c_{10}$ & \z  & \z   & \z  & 5   & 5   &  5  &  5   \\ \hline
    $c_{11}$ & \z  & 10   &  5  & 5   & 5   &  5  &  5   \\ \hline
    $c_{12}$ & \z  & \z   & 16  & 9   & 9   &  9  &  9   \\ \hline
    $c_{13}$ & \z  & \z   & 7   & 7   & 7   &  7  &  7   \\ \hline
    $c_{14}$ & \z  & \z   & \z  & 5   & 5   &  5  &  5   \\ \hline
    $c_{15}$ & \z  & \z   & 4   &$4^+$&     &     &      \\ \hline
    $c_{16}$ & \z  & 10   & 5   & 5   & 5   &  5  &  5   \\ \hline
    $c_{17}$ & \z  & 6    & 6   & 6   & 6   &  6  &  6   \\ \hline 
\end{tabular}

    \item $\Gamma p = \{ c_3, c_8, c_{10}, c_{12}, c_{13}, c_{14}, c_{16}, c_{17}\}$
        $$l(c_3) = min(6, 5^+ + 8) = 6$$
        $$l(c_8) = min(\infty, 5^+ + 5) = 10$$
        $$l(c_{10}) = min(5, 5^+ + 3) = 5$$
        $$l(c_{12}) = min(9, 5^+ + 2) = 7$$
        $$l(c_{13}) = min(7, 5^+ + 21) = 7$$
        $$l(c_{14}) = min(5, 5^+ + 6) = 5$$
        $$l(c_{16}) = min(5, 5^+ + 8) = 5$$
        $$l(c_{17}) = min(6, 5^+ + 3) = 6$$
\begin{tabular} {|c|c|c|c|c|c|c|c|c|}
    \hline
    $c_{1}$  &$0^+$&      &     &     &     &     &     &         \\ \hline
    $c_{2}$  & \z  &$1^+$ &     &     &     &     &     &   	 \\ \hline
    $c_{3}$  & \z  & \z   & \z  &\z   & 6   &  6  &  6  &  6	 \\ \hline
    $c_{4}$  & \z  & \z   & 5   & 5   & 5   &$5^+$&     &   	   \\ \hline
    $c_{5}$  & \z  & 4    & 4   & 4   &$4^+$&     &     &   	    \\ \hline
    $c_{6}$  & \z  & 5    & 5   & 5   & 5   &  5  &$5^+$&   	    \\ \hline
    $c_{7}$  & \z  & \z   & \z  & \z  & 7   &  7  &  7  &  7	 \\ \hline
    $c_{8}$  & \z  & 2    & \z  & \z  & \z  &  \z & \z  &  10	 \\ \hline
    $c_{9}$  & \z  & \z   &$3^+$&     &     &     &     &   	 \\ \hline
    $c_{10}$ & \z  & \z   & \z  & 5   & 5   &  5  &  5  &$5^+$	 \\ \hline
    $c_{11}$ & \z  & 10   &  5  & 5   & 5   &  5  &  5  &  5	 \\ \hline
    $c_{12}$ & \z  & \z   & 16  & 9   & 9   &  9  &  9  &  7	 \\ \hline
    $c_{13}$ & \z  & \z   & 7   & 7   & 7   &  7  &  7  &  7	 \\ \hline
    $c_{14}$ & \z  & \z   & \z  & 5   & 5   &  5  &  5  &  5	 \\ \hline
    $c_{15}$ & \z  & \z   & 4   &$4^+$&     &     &     &   	 \\ \hline
    $c_{16}$ & \z  & 10   & 5   & 5   & 5   &  5  &  5  &  5	 \\ \hline
    $c_{17}$ & \z  & 6    & 6   & 6   & 6   &  6  &  6  &  6	 \\ \hline 
\end{tabular}
	\item $\Gamma p = \{ c_{11}, c_{13}, c_{14}, c_{16}, c_{17}\}$
	$$l(c_{11}) = min(5, 5^+ + 1) = 5$$
	$$l(c_{13}) = min(7, 5^+ + 4) = 7$$
	$$l(c_{14}) = min(5, 5^+ + 3) = 5$$
	$$l(c_{16}) = min(5, 5^+ + 1) = 5$$
	$$l(c_{17}) = min(6, 5^+ + 2) = 6$$
\begin{tabular} {|c|c|c|c|c|c|c|c|c|c|c|}
    \hline
    $c_{1}$  &$0^+$&      &     &     &     &     &     &      &       \\ \hline
    $c_{2}$  & \z  &$1^+$ &     &     &     &     &     &      &       \\ \hline
    $c_{3}$  & \z  & \z   & \z  &\z   & 6   &  6  &  6  &  6   & 6     \\ \hline
    $c_{4}$  & \z  & \z   & 5   & 5   & 5   &$5^+$&     &      &       \\ \hline
    $c_{5}$  & \z  & 4    & 4   & 4   &$4^+$&     &     &      &       \\ \hline
    $c_{6}$  & \z  & 5    & 5   & 5   & 5   &  5  &$5^+$&      &       \\ \hline
    $c_{7}$  & \z  & \z   & \z  & \z  & 7   &  7  &  7  &  7   & 7     \\ \hline
    $c_{8}$  & \z  & 2    & \z  & \z  & \z  &  \z & \z  &  10  & 10    \\ \hline
    $c_{9}$  & \z  & \z   &$3^+$&     &     &     &     &      &       \\ \hline
    $c_{10}$ & \z  & \z   & \z  & 5   & 5   &  5  &  5  &$5^+$ &       \\ \hline
    $c_{11}$ & \z  & 10   &  5  & 5   & 5   &  5  &  5  &  5   &$5^+$  \\ \hline
    $c_{12}$ & \z  & \z   & 16  & 9   & 9   &  9  &  9  &  7   & 7     \\ \hline
    $c_{13}$ & \z  & \z   & 7   & 7   & 7   &  7  &  7  &  7   & 7     \\ \hline
    $c_{14}$ & \z  & \z   & \z  & 5   & 5   &  5  &  5  &  5   & 5     \\ \hline
    $c_{15}$ & \z  & \z   & 4   &$4^+$&     &     &     &      &       \\ \hline
    $c_{16}$ & \z  & 10   & 5   & 5   & 5   &  5  &  5  &  5   & 5     \\ \hline
    $c_{17}$ & \z  & 6    & 6   & 6   & 6   &  6  &  6  &  6   & 6     \\ \hline 
\end{tabular}
	\item $\Gamma p = \{ c_8\}$
	$$l(c_8) = min(10, 5^+ + 3) = 8$$
\begin{tabular} {|c|c|c|c|c|c|c|c|c|c|c|c|}
    \hline
    $c_{1}$  &$0^+$&      &     &     &     &     &     &      &     &      \\ \hline
    $c_{2}$  & \z  &$1^+$ &     &     &     &     &     &      &     &      \\ \hline
    $c_{3}$  & \z  & \z   & \z  &\z   & 6   &  6  &  6  &  6   & 6   & 6    \\ \hline
    $c_{4}$  & \z  & \z   & 5   & 5   & 5   &$5^+$&     &      &     &      \\ \hline
    $c_{5}$  & \z  & 4    & 4   & 4   &$4^+$&     &     &      &     &      \\ \hline
    $c_{6}$  & \z  & 5    & 5   & 5   & 5   &  5  &$5^+$&      &     &      \\ \hline
    $c_{7}$  & \z  & \z   & \z  & \z  & 7   &  7  &  7  &  7   & 7   & 7    \\ \hline
    $c_{8}$  & \z  & 2    & \z  & \z  & \z  &  \z & \z  &  10  & 10  & 8    \\ \hline
    $c_{9}$  & \z  & \z   &$3^+$&     &     &     &     &      &     &      \\ \hline
    $c_{10}$ & \z  & \z   & \z  & 5   & 5   &  5  &  5  &$5^+$ &     &      \\ \hline
    $c_{11}$ & \z  & 10   &  5  & 5   & 5   &  5  &  5  &  5   &$5^+$&      \\ \hline
    $c_{12}$ & \z  & \z   & 16  & 9   & 9   &  9  &  9  &  7   & 7   & 7    \\ \hline
    $c_{13}$ & \z  & \z   & 7   & 7   & 7   &  7  &  7  &  7   & 7   & 7    \\ \hline
    $c_{14}$ & \z  & \z   & \z  & 5   & 5   &  5  &  5  &  5   & 5   &$5^+$ \\ \hline
    $c_{15}$ & \z  & \z   & 4   &$4^+$&     &     &     &      &     &      \\ \hline
    $c_{16}$ & \z  & 10   & 5   & 5   & 5   &  5  &  5  &  5   & 5   & 5    \\ \hline
    $c_{17}$ & \z  & 6    & 6   & 6   & 6   &  6  &  6  &  6   & 6   & 6    \\ \hline 
\end{tabular}

	\item $\Gamma p = \{ c_3, c_{12}, c_{16}, c_{17}\}$
		$$l(c_3) = min(6, 5^+ + 6) = 6$$
		$$l(c_{12} = min(7, 5^+ + 5) = 7$$
		$$l(c_{16} = min(5, 5^+ + 6) = 5$$
		$$l(c_{17} = min(6, 5^+ + 6) = 6$$

\begin{tabular} {|c|c|c|c|c|c|c|c|c|c|c|c|c|}
    \hline
    $c_{1}$  &$0^+$&      &     &     &     &     &     &      &     &     &       \\ \hline
    $c_{2}$  & \z  &$1^+$ &     &     &     &     &     &      &     &     &       \\ \hline
    $c_{3}$  & \z  & \z   & \z  &\z   & 6   &  6  &  6  &  6   & 6   & 6   & 6     \\ \hline
    $c_{4}$  & \z  & \z   & 5   & 5   & 5   &$5^+$&     &      &     &     &       \\ \hline
    $c_{5}$  & \z  & 4    & 4   & 4   &$4^+$&     &     &      &     &     &       \\ \hline
    $c_{6}$  & \z  & 5    & 5   & 5   & 5   &  5  &$5^+$&      &     &     &       \\ \hline
    $c_{7}$  & \z  & \z   & \z  & \z  & 7   &  7  &  7  &  7   & 7   & 7   & 7     \\ \hline
    $c_{8}$  & \z  & 2    & \z  & \z  & \z  &  \z & \z  &  10  & 10  & 8   & 8     \\ \hline
    $c_{9}$  & \z  & \z   &$3^+$&     &     &     &     &      &     &     &       \\ \hline
    $c_{10}$ & \z  & \z   & \z  & 5   & 5   &  5  &  5  &$5^+$ &     &     &       \\ \hline
    $c_{11}$ & \z  & 10   &  5  & 5   & 5   &  5  &  5  &  5   &$5^+$&     &       \\ \hline
    $c_{12}$ & \z  & \z   & 16  & 9   & 9   &  9  &  9  &  7   & 7   & 7   & 7     \\ \hline
    $c_{13}$ & \z  & \z   & 7   & 7   & 7   &  7  &  7  &  7   & 7   & 7   & 7     \\ \hline
    $c_{14}$ & \z  & \z   & \z  & 5   & 5   &  5  &  5  &  5   & 5   &$5^+$&       \\ \hline
    $c_{15}$ & \z  & \z   & 4   &$4^+$&     &     &     &      &     &     &       \\ \hline
    $c_{16}$ & \z  & 10   & 5   & 5   & 5   &  5  &  5  &  5   & 5   & 5   &$5^+$     \\ \hline
    $c_{17}$ & \z  & 6    & 6   & 6   & 6   &  6  &  6  &  6   & 6   & 6   & 6     \\ \hline 
\end{tabular}

	\item $\Gamma p = \{c_3, c_7, c_{12}, c_{13}, c_{17}\}$
	$$l(c_3)    = min(6, 5^+ + 3) = 6$$
	$$l(c_7)    = min(7, 5^+ + 4) = 7$$
	$$l(c_{12}) = min(7, 5^+ + 6) = 7$$
	$$l(c_{13}) = min(7, 5^+ + 9) = 7$$
	$$l(c_{17}) = min(6, 5^+ + 3) = 6$$
\begin{tabular} {|c|c|c|c|c|c|c|c|c|c|c|c|c|}
    \hline
    $c_{1}$  &$0^+$&      &     &     &     &     &     &      &     &     &     &       \\ \hline
    $c_{2}$  & \z  &$1^+$ &     &     &     &     &     &      &     &     &     &       \\ \hline
    $c_{3}$  & \z  & \z   & \z  &\z   & 6   &  6  &  6  &  6   & 6   & 6   & 6   &$6^+$     \\ \hline
    $c_{4}$  & \z  & \z   & 5   & 5   & 5   &$5^+$&     &      &     &     &     &       \\ \hline
    $c_{5}$  & \z  & 4    & 4   & 4   &$4^+$&     &     &      &     &     &     &       \\ \hline
    $c_{6}$  & \z  & 5    & 5   & 5   & 5   &  5  &$5^+$&      &     &     &     &       \\ \hline
    $c_{7}$  & \z  & \z   & \z  & \z  & 7   &  7  &  7  &  7   & 7   & 7   & 7   & 7     \\ \hline
    $c_{8}$  & \z  & 2    & \z  & \z  & \z  &  \z & \z  &  10  & 10  & 8   & 8   & 8     \\ \hline
    $c_{9}$  & \z  & \z   &$3^+$&     &     &     &     &      &     &     &     &       \\ \hline
    $c_{10}$ & \z  & \z   & \z  & 5   & 5   &  5  &  5  &$5^+$ &     &     &     &       \\ \hline
    $c_{11}$ & \z  & 10   &  5  & 5   & 5   &  5  &  5  &  5   &$5^+$&     &     &       \\ \hline
    $c_{12}$ & \z  & \z   & 16  & 9   & 9   &  9  &  9  &  7   & 7   & 7   & 7   & 7     \\ \hline
    $c_{13}$ & \z  & \z   & 7   & 7   & 7   &  7  &  7  &  7   & 7   & 7   & 7   & 7     \\ \hline
    $c_{14}$ & \z  & \z   & \z  & 5   & 5   &  5  &  5  &  5   & 5   &$5^+$&     &       \\ \hline
    $c_{15}$ & \z  & \z   & 4   &$4^+$&     &     &     &      &     &     &     &       \\ \hline
    $c_{16}$ & \z  & 10   & 5   & 5   & 5   &  5  &  5  &  5   & 5   & 5   &$5^+$&          \\ \hline
    $c_{17}$ & \z  & 6    & 6   & 6   & 6   &  6  &  6  &  6   & 6   & 6   & 6   & 6     \\ \hline 
\end{tabular}
%
%	\item Далее значения уж точно не будут меняться, посему сократим таблицу, обозначив все оставшиеся вершины постоянными.
%
%\begin{tabular} {|c|c|c|c|c|c|c|c|c|c|c|c|c|}
%    \hline
%    $c_{1}$  &$0^+$&      &     &     &     &     &     &      &     &     &     &       \\ \hline
%    $c_{2}$  & \z  &$1^+$ &     &     &     &     &     &      &     &     &     &       \\ \hline
%    $c_{3}$  & \z  & \z   & \z  &\z   & 6   &  6  &  6  &  6   & 6   & 6   & 6   &$6^+$     \\ \hline
%    $c_{4}$  & \z  & \z   & 5   & 5   & 5   &$5^+$&     &      &     &     &     &       \\ \hline
%    $c_{5}$  & \z  & 4    & 4   & 4   &$4^+$&     &     &      &     &     &     &       \\ \hline
%    $c_{6}$  & \z  & 5    & 5   & 5   & 5   &  5  &$5^+$&      &     &     &     &       \\ \hline
%    $c_{7}$  & \z  & \z   & \z  & \z  & 7   &  7  &  7  &  7   & 7   & 7   & 7   &$7^+$     \\ \hline
%    $c_{8}$  & \z  & 2    & \z  & \z  & \z  &  \z & \z  &  10  & 10  & 8   & 8   &$8^+$     \\ \hline
%    $c_{9}$  & \z  & \z   &$3^+$&     &     &     &     &      &     &     &     &       \\ \hline
%    $c_{10}$ & \z  & \z   & \z  & 5   & 5   &  5  &  5  &$5^+$ &     &     &     &       \\ \hline
%    $c_{11}$ & \z  & 10   &  5  & 5   & 5   &  5  &  5  &  5   &$5^+$&     &     &       \\ \hline
%    $c_{12}$ & \z  & \z   & 16  & 9   & 9   &  9  &  9  &  7   & 7   & 7   & 7   & $7^+$     \\ \hline
%    $c_{13}$ & \z  & \z   & 7   & 7   & 7   &  7  &  7  &  7   & 7   & 7   & 7   & $7^+$     \\ \hline
%    $c_{14}$ & \z  & \z   & \z  & 5   & 5   &  5  &  5  &  5   & 5   &$5^+$&     &       \\ \hline
%    $c_{15}$ & \z  & \z   & 4   &$4^+$&     &     &     &      &     &     &     &       \\ \hline
%    $c_{16}$ & \z  & 10   & 5   & 5   & 5   &  5  &  5  &  5   & 5   & 5   &$5^+$&          \\ \hline
%    $c_{17}$ & \z  & 6    & 6   & 6   & 6   &  6  &  6  &  6   & 6   & 6   & 6   & $6^+$     \\ \hline 
%\end{tabular}

	\item $\Gamma p = \{ c_{13}\}$.
	$$l(c_{13}) = max(7, 6^+ + 4) = 7 $$

\begin{tabular} {|c|c|c|c|c|c|c|c|c|c|c|c|c|c|}
    \hline
    $c_{1}$  &$0^+$&      &     &     &     &     &     &      &     &     &     &     &  	       \\ \hline
    $c_{2}$  & \z  &$1^+$ &     &     &     &     &     &      &     &     &     &     &  	       \\ \hline
    $c_{3}$  & \z  & \z   & \z  &\z   & 6   &  6  &  6  &  6   & 6   & 6   & 6   &$6^+$&  	       \\ \hline
    $c_{4}$  & \z  & \z   & 5   & 5   & 5   &$5^+$&     &      &     &     &     &     &  	       \\ \hline
    $c_{5}$  & \z  & 4    & 4   & 4   &$4^+$&     &     &      &     &     &     &     &  	       \\ \hline
    $c_{6}$  & \z  & 5    & 5   & 5   & 5   &  5  &$5^+$&      &     &     &     &     &  	       \\ \hline
    $c_{7}$  & \z  & \z   & \z  & \z  & 7   &  7  &  7  &  7   & 7   & 7   & 7   &7    & 7	       \\ \hline
    $c_{8}$  & \z  & 2    & \z  & \z  & \z  &  \z & \z  &  10  & 10  & 8   & 8   &8    & 8	       \\ \hline
    $c_{9}$  & \z  & \z   &$3^+$&     &     &     &     &      &     &     &     &     &  	       \\ \hline
    $c_{10}$ & \z  & \z   & \z  & 5   & 5   &  5  &  5  &$5^+$ &     &     &     &     &  	       \\ \hline
    $c_{11}$ & \z  & 10   &  5  & 5   & 5   &  5  &  5  &  5   &$5^+$&     &     &     &  	       \\ \hline
    $c_{12}$ & \z  & \z   & 16  & 9   & 9   &  9  &  9  &  7   & 7   & 7   & 7   & 7   & 7	       \\ \hline
    $c_{13}$ & \z  & \z   & 7   & 7   & 7   &  7  &  7  &  7   & 7   & 7   & 7   & 7   & 7   	   \\ \hline
    $c_{14}$ & \z  & \z   & \z  & 5   & 5   &  5  &  5  &  5   & 5   &$5^+$&     &     &  	       \\ \hline
    $c_{15}$ & \z  & \z   & 4   &$4^+$&     &     &     &      &     &     &     &     &  	       \\ \hline
    $c_{16}$ & \z  & 10   & 5   & 5   & 5   &  5  &  5  &  5   & 5   & 5   &$5^+$&     &  	       \\ \hline
    $c_{17}$ & \z  & 6    & 6   & 6   & 6   &  6  &  6  &  6   & 6   & 6   & 6   & 6   &$6^+$	   \\ \hline 
\end{tabular}

	\item $\Gamma p = \{ c{12}, c_{13}\}$
	$$l(c_{12}) = max(7, 6^+ + 2) = 7 $$
	$$l(c_{13}) = max(7, 6^+ + 4) = 7 $$
\begin{tabular} {|c|c|c|c|c|c|c|c|c|c|c|c|c|c|c|}
    \hline
    $c_{1}$  &$0^+$&      &     &     &     &     &     &      &     &     &     &     &  	 &         \\ \hline
    $c_{2}$  & \z  &$1^+$ &     &     &     &     &     &      &     &     &     &     &  	 &         \\ \hline
    $c_{3}$  & \z  & \z   & \z  &\z   & 6   &  6  &  6  &  6   & 6   & 6   & 6   &$6^+$&  	 &         \\ \hline
    $c_{4}$  & \z  & \z   & 5   & 5   & 5   &$5^+$&     &      &     &     &     &     &  	 &         \\ \hline
    $c_{5}$  & \z  & 4    & 4   & 4   &$4^+$&     &     &      &     &     &     &     &  	 &         \\ \hline
    $c_{6}$  & \z  & 5    & 5   & 5   & 5   &  5  &$5^+$&      &     &     &     &     &  	 &         \\ \hline
    $c_{7}$  & \z  & \z   & \z  & \z  & 7   &  7  &  7  &  7   & 7   & 7   & 7   &7    & 7	 & $7^+$      \\ \hline
    $c_{8}$  & \z  & 2    & \z  & \z  & \z  &  \z & \z  &  10  & 10  & 8   & 8   &8    & 8	 &  8      \\ \hline
    $c_{9}$  & \z  & \z   &$3^+$&     &     &     &     &      &     &     &     &     &  	 &         \\ \hline
    $c_{10}$ & \z  & \z   & \z  & 5   & 5   &  5  &  5  &$5^+$ &     &     &     &     &  	 &         \\ \hline
    $c_{11}$ & \z  & 10   &  5  & 5   & 5   &  5  &  5  &  5   &$5^+$&     &     &     &  	 &         \\ \hline
    $c_{12}$ & \z  & \z   & 16  & 9   & 9   &  9  &  9  &  7   & 7   & 7   & 7   & 7   & 7	 &  7      \\ \hline
    $c_{13}$ & \z  & \z   & 7   & 7   & 7   &  7  &  7  &  7   & 7   & 7   & 7   & 7   & 7   &	7     \\ \hline
    $c_{14}$ & \z  & \z   & \z  & 5   & 5   &  5  &  5  &  5   & 5   &$5^+$&     &     &  	 &         \\ \hline
    $c_{15}$ & \z  & \z   & 4   &$4^+$&     &     &     &      &     &     &     &     &  	 &         \\ \hline
    $c_{16}$ & \z  & 10   & 5   & 5   & 5   &  5  &  5  &  5   & 5   & 5   &$5^+$&     &  	 &         \\ \hline
    $c_{17}$ & \z  & 6    & 6   & 6   & 6   &  6  &  6  &  6   & 6   & 6   & 6   & 6   &$6^+$&	      \\ \hline 
\end{tabular}

	\item $\Gamma p = \{ c{12}\}$
	$$l(c_{12}) = max(7, 7^+ + 5) = 7 $$
\begin{tabular} {|c|c|c|c|c|c|c|c|c|c|c|c|c|c|c|c|}
    \hline
    $c_{1}$  &$0^+$&      &     &     &     &     &     &      &     &     &     &     &     &      &         \\ \hline
    $c_{2}$  & \z  &$1^+$ &     &     &     &     &     &      &     &     &     &     &     &      &         \\ \hline
    $c_{3}$  & \z  & \z   & \z  &\z   & 6   &  6  &  6  &  6   & 6   & 6   & 6   &$6^+$&     &      &         \\ \hline
    $c_{4}$  & \z  & \z   & 5   & 5   & 5   &$5^+$&     &      &     &     &     &     &     &      &         \\ \hline
    $c_{5}$  & \z  & 4    & 4   & 4   &$4^+$&     &     &      &     &     &     &     &     &      &         \\ \hline
    $c_{6}$  & \z  & 5    & 5   & 5   & 5   &  5  &$5^+$&      &     &     &     &     &     &      &         \\ \hline
    $c_{7}$  & \z  & \z   & \z  & \z  & 7   &  7  &  7  &  7   & 7   & 7   & 7   &7    & 7   & $7^+$&            \\ \hline
    $c_{8}$  & \z  & 2    & \z  & \z  & \z  &  \z & \z  &  10  & 10  & 8   & 8   &8    & 8   &  8   & 8       \\ \hline
    $c_{9}$  & \z  & \z   &$3^+$&     &     &     &     &      &     &     &     &     &     &      &         \\ \hline
    $c_{10}$ & \z  & \z   & \z  & 5   & 5   &  5  &  5  &$5^+$ &     &     &     &     &     &      &         \\ \hline
    $c_{11}$ & \z  & 10   &  5  & 5   & 5   &  5  &  5  &  5   &$5^+$&     &     &     &     &      &         \\ \hline
    $c_{12}$ & \z  & \z   & 16  & 9   & 9   &  9  &  9  &  7   & 7   & 7   & 7   & 7   & 7   &  7   &$7^+$        \\ \hline
    $c_{13}$ & \z  & \z   & 7   & 7   & 7   &  7  &  7  &  7   & 7   & 7   & 7   & 7   & 7   &  7   & 7      \\ \hline
    $c_{14}$ & \z  & \z   & \z  & 5   & 5   &  5  &  5  &  5   & 5   &$5^+$&     &     &     &      &         \\ \hline
    $c_{15}$ & \z  & \z   & 4   &$4^+$&     &     &     &      &     &     &     &     &     &      &         \\ \hline
    $c_{16}$ & \z  & 10   & 5   & 5   & 5   &  5  &  5  &  5   & 5   & 5   &$5^+$&     &     &      &         \\ \hline
    $c_{17}$ & \z  & 6    & 6   & 6   & 6   &  6  &  6  &  6   & 6   & 6   & 6   & 6   &$6^+$&      &        \\ \hline 
\end{tabular}

	\item $\Gamma p = \{ c_{13}\}$
	$$l(c_{13}) = max(7, 7^+ + 12) = 7 $$
\begin{tabular} {|c|c|c|c|c|c|c|c|c|c|c|c|c|c|c|c|c|}
    \hline
    $c_{1}$  &$0^+$&      &     &     &     &     &     &      &     &     &     &     &     &      &     &    \\ \hline
    $c_{2}$  & \z  &$1^+$ &     &     &     &     &     &      &     &     &     &     &     &      &     &    \\ \hline
    $c_{3}$  & \z  & \z   & \z  &\z   & 6   &  6  &  6  &  6   & 6   & 6   & 6   &$6^+$&     &      &     &    \\ \hline
    $c_{4}$  & \z  & \z   & 5   & 5   & 5   &$5^+$&     &      &     &     &     &     &     &      &     &    \\ \hline
    $c_{5}$  & \z  & 4    & 4   & 4   &$4^+$&     &     &      &     &     &     &     &     &      &     &    \\ \hline
    $c_{6}$  & \z  & 5    & 5   & 5   & 5   &  5  &$5^+$&      &     &     &     &     &     &      &     &    \\ \hline
    $c_{7}$  & \z  & \z   & \z  & \z  & 7   &  7  &  7  &  7   & 7   & 7   & 7   &7    & 7   & $7^+$&     &       \\ \hline
    $c_{8}$  & \z  & 2    & \z  & \z  & \z  &  \z & \z  &  10  & 10  & 8   & 8   &8    & 8   &  8   & 8   & 8  \\ \hline
    $c_{9}$  & \z  & \z   &$3^+$&     &     &     &     &      &     &     &     &     &     &      &     &    \\ \hline
    $c_{10}$ & \z  & \z   & \z  & 5   & 5   &  5  &  5  &$5^+$ &     &     &     &     &     &      &     &    \\ \hline
    $c_{11}$ & \z  & 10   &  5  & 5   & 5   &  5  &  5  &  5   &$5^+$&     &     &     &     &      &     &    \\ \hline
    $c_{12}$ & \z  & \z   & 16  & 9   & 9   &  9  &  9  &  7   & 7   & 7   & 7   & 7   & 7   &  7   &$7^+$&        \\ \hline
    $c_{13}$ & \z  & \z   & 7   & 7   & 7   &  7  &  7  &  7   & 7   & 7   & 7   & 7   & 7   &  7   & 7   &$7^+$   \\ \hline
    $c_{14}$ & \z  & \z   & \z  & 5   & 5   &  5  &  5  &  5   & 5   &$5^+$&     &     &     &      &     &    \\ \hline
    $c_{15}$ & \z  & \z   & 4   &$4^+$&     &     &     &      &     &     &     &     &     &      &     &    \\ \hline
    $c_{16}$ & \z  & 10   & 5   & 5   & 5   &  5  &  5  &  5   & 5   & 5   &$5^+$&     &     &      &     &    \\ \hline
    $c_{17}$ & \z  & 6    & 6   & 6   & 6   &  6  &  6  &  6   & 6   & 6   & 6   & 6   &$6^+$&      &     &   \\ \hline 
\end{tabular}

	\item $\Gamma p = \{ c_8\}$
	$$l(c_{13}) = max(8, 7^+ + 2) = 8 $$
\begin{tabular} {|c|c|c|c|c|c|c|c|c|c|c|c|c|c|c|c|c|c|c|}
    \hline
    $c_{1}$  &$0^+$&      &     &     &     &     &     &      &     &     &     &     &     &      &     &    & \\ \hline
    $c_{2}$  & \z  &$1^+$ &     &     &     &     &     &      &     &     &     &     &     &      &     &    & \\ \hline
    $c_{3}$  & \z  & \z   & \z  &\z   & 6   &  6  &  6  &  6   & 6   & 6   & 6   &$6^+$&     &      &     &    & \\ \hline
    $c_{4}$  & \z  & \z   & 5   & 5   & 5   &$5^+$&     &      &     &     &     &     &     &      &     &    & \\ \hline
    $c_{5}$  & \z  & 4    & 4   & 4   &$4^+$&     &     &      &     &     &     &     &     &      &     &    & \\ \hline
    $c_{6}$  & \z  & 5    & 5   & 5   & 5   &  5  &$5^+$&      &     &     &     &     &     &      &     &    & \\ \hline
    $c_{7}$  & \z  & \z   & \z  & \z  & 7   &  7  &  7  &  7   & 7   & 7   & 7   &7    & 7   & $7^+$&     &       & \\ \hline
    $c_{8}$  & \z  & 2    & \z  & \z  & \z  &  \z & \z  &  10  & 10  & 8   & 8   &8    & 8   &  8   & 8   & 8  & $8^+$\\ \hline
    $c_{9}$  & \z  & \z   &$3^+$&     &     &     &     &      &     &     &     &     &     &      &     &    & \\ \hline
    $c_{10}$ & \z  & \z   & \z  & 5   & 5   &  5  &  5  &$5^+$ &     &     &     &     &     &      &     &    & \\ \hline
    $c_{11}$ & \z  & 10   &  5  & 5   & 5   &  5  &  5  &  5   &$5^+$&     &     &     &     &      &     &    & \\ \hline
    $c_{12}$ & \z  & \z   & 16  & 9   & 9   &  9  &  9  &  7   & 7   & 7   & 7   & 7   & 7   &  7   &$7^+$&        & \\ \hline
    $c_{13}$ & \z  & \z   & 7   & 7   & 7   &  7  &  7  &  7   & 7   & 7   & 7   & 7   & 7   &  7   & 7   &$7^+$   & \\ \hline
    $c_{14}$ & \z  & \z   & \z  & 5   & 5   &  5  &  5  &  5   & 5   &$5^+$&     &     &     &      &     &    & \\ \hline
    $c_{15}$ & \z  & \z   & 4   &$4^+$&     &     &     &      &     &     &     &     &     &      &     &    & \\ \hline
    $c_{16}$ & \z  & 10   & 5   & 5   & 5   &  5  &  5  &  5   & 5   & 5   &$5^+$&     &     &      &     &    & \\ \hline
    $c_{17}$ & \z  & 6    & 6   & 6   & 6   &  6  &  6  &  6   & 6   & 6   & 6   & 6   &$6^+$&      &     &   & \\ \hline 
\end{tabular}

\end{enumerate}

\section{Поиск пропускной способности алгоритмом Франка-Фриша}

Возьмём за граф с пропускной способностью граф с весами, представленный в виде
матрица из пунтка 4.

Найдём пропускную способность, принимая за исток $s$ вершину $e_1$, а за сток
$t$~--- вершину $e_7$.

Обнаружим разрез $(e_1, X \setminus e_1)$. Его максимальная пропускная
способность равна $10$.

Объединим все вершины, между которыми есть ребро весом $\ge 10$. Объединяются
множества $(e_1, e_{11}, e_{16})$, $(e_2, e_{12}, e_{13}, e_{6})$.

Получаем граф (так как разрез определяет по максимальной ПС, оставляем лишь наибольшие значения):

\begin{tabular}{|c|c|c|c|c|c|c|c|c|c|c|c|c|}
\hline
 & $e_{1,11,16}$ & $e_{2,6,12,13}$ & $e_{3}$ & $e_{4}$ & $e_{5}$ & $e_{7}$ & $e_{8}$ & $e_{9}$ & $e_{10}$ & $e_{14}$ & $e_{15}$ & $e_{17}$ \\
\hline %         1   2   3   4   5   7   8   9  10  14  15  17
$e_{1,11,16}$  & 0 & 9 & 3 & 2 & 4 & 4 & 3 & 2 & 1 & 6 & 3 & 6 \\ \hline
$e_{2,6,12,13}$&   & 0 & 4 & 9 & 6 & 5 & 5 & 8 & 4 & 6 & 8 & 5 \\ \hline
$e_{3}$        &   &   & 0 & 0 & 0 & 0 & 0 & 0 & 0 & 6 & 2 & 0 \\ \hline
$e_{4}$        &   &   &   & 0 & 0 & 0 & 0 & 2 & 0 & 4 & 3 & 9 \\ \hline
$e_{5}$        &   &   &   &   & 0 & 0 & 0 & 0 & 0 & 0 & 0 & 2 \\ \hline
$e_{7}$        &   &   &   &   &   & 0 & 0 & 0 & 0 & 0 & 3 & 0 \\ \hline
$e_{8}$        &   &   &   &   &   &   & 0 & 0 & 0 & 0 & 0 & 0 \\ \hline
$e_{9}$        &   &   &   &   &   &   &   & 0 & 2 & 2 & 2 & 3 \\ \hline
$e_{10}$       &   &   &   &   &   &   &   &   & 0 & 3 & 6 & 2 \\ \hline
$e_{14}$       &   &   &   &   &   &   &   &   &   & 0 & 2 & 6 \\ \hline
$e_{15}$       &   &   &   &   &   &   &   &   &   &   & 0 & 0 \\ \hline
$e_{17}$       &   &   &   &   &   &   &   &   &   &   &   & 0 \\ \hline
\end{tabular}


Теперь максимальная ПС -- 6. Пусть
$$s = (e_{1, 11,16}, e_{14}, e_{17}, e_{2,6,12,13}, e_3, e_4, e_5, e_9, e_{15}, e_{10})$$
$$t = e_7$$

\begin{tabular}{|c|c|c|c|c|}
\hline
         & s & t & $e_8$ \\ \hline
  s      & 0 & 5 &   5   \\ \hline
  t      &   & 0 &   0   \\ \hline
$e_8$    &   &   &   0   \\ \hline
\end{tabular}

Теперь максимальная ПС -- 5. Далее получаем, что s и t в одном множестве,
следовательно, максимальная ПС равна 5. Граф приведён ниже. Из-за большого
количество узлов и дуг, обозначим на графе только те вершины, которые
задействованы в пути. \\

\begin{tikzpicture}
	\node[circle, draw] (e1) at (1, 1) {$e_1$};
	\node[circle, draw] (e16) at (3, 1) {$e_{16}$};
	\node[circle, draw] (e12) at (5, 1) {$e_{12}$};
	\node[circle, draw] (e7) at (7, 1) {$e_7$};

	\draw[->] (e1) -- node[above] {10} (e16);
	\draw[->] (e16) -- node[above] {6} (e12);
	\draw[->] (e12) -- node[above] {5} (e7);
\end{tikzpicture}

	Учитывая, что максимальное возможное значение для того, чтобы достигнуть $e_7$,
	это 5, большее значение ПС уж точно не достигнуть.


\end{document}
