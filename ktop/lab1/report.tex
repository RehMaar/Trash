\documentclass[12pt, a4paper] {ncc}
\usepackage[utf8] {inputenc}
\usepackage[T2A]{fontenc}
\usepackage[english, russian] {babel}
\usepackage[usenames,dvipsnames]{xcolor}
\usepackage{listings,a4wide,longtable,amsmath,amsfonts,graphicx}
\usepackage{indentfirst}
\usepackage{bytefield}
\usepackage{multirow}
\usepackage{float}
\usepackage{caption}
\usepackage{subcaption}
\captionsetup{compatibility=false}
\usepackage{tabularx,tikz}
\usetikzlibrary{patterns}

\usepackage[left=2cm,right=2cm,top=2cm,bottom=2cm,bindingoffset=0cm]{geometry}

\begin{document}
\setcounter{figure}{0}
\frenchspacing
\pagestyle{empty}


\begin{center}
                            Университет ИТМО    \\
                        Кафедра вычислительной техники

\vspace{\stretch{2}}

\end{center}
\vspace{\stretch{2}}
\begin{center}
						Домашняя работа № 1 \\
				по дисциплине: \\
	<<Конструкторско-технологическое обеспечение производства ЭВМ>> \\
                    Вариант: 2, Схема 3.
\end{center}
\vspace{\stretch{3}}
\begin{flushright}
                                    Студент:\\
                                    {\it Куклина М.Д., P3401}\\
                                    Преподаватель: \\
                                    {\it Поляков В.И.}
\end{flushright}
\vspace{\stretch{4}}
\begin{center}
                             Санкт-Петербург, 2018
\end{center}
\newpage

\section{Постановка задачи}

	\includegraphics[scale=0.5]{./task_scheme.png}

	$R_1, R_5 = 20 \text{кОм} \pm 10 \%, 0.01 \text{Вт}$

	$R_2, R_6 = 4.5 \text{кОм} \pm 10 \%, 0.05 \text{Вт}$

	$R_3, R_4 = 2.1 \text{кОм} \pm 10 \%, 0.03 \text{Вт}$

	$C_1, C_2 = 800 \text{пФ}$

\section{Расчёт тонкоплёночных резисторов}

	\subsection{Расчёт оптимального удельного поверхностного сопротивления}

			$\rho_{\square \text{опт}} = \sqrt { \dfrac {\sum\limits_{i = 1}^{n} R_i} {\sum\limits_{i = 1}^{n} R_i^{-1}}} $

			$\rho_{\square \text{опт}} \approx 5961.7 \approx 6000 (\dfrac {\text{Ом}} {\square})$

	\subsection{Выбор материала резистивной плёнки}

    	\begin{description}
    		\item[Материал:] Кермет К-50С.
    		\item[$\rho_{\square \text{опт}}$, Ом/$\square$:] $1000 - 10000$.
    		\item[Диапазн значения сопротивления, Ом:] $100-100000$.
    		\item[Удельная мощность рассеяия $W_0$, Вт/см$^2$:] $2$.
    	\end{description}

		Тип контактной площадки.
		\begin{description}
			\item[Материал слоя:] Алюминий А-99.
			\item[Толщина слоя:] 0.3 -- 0.6.
			\item[$\rho_{\square}$:] 0.03 - 0.06.
			\item[Рекомендуемый способ контактирования:] Сварка.
		\end{description}

	\subsection{Коэффициент формы каждой резистора}
    	 Расчётная формула:
    	\begin{center}
    		$k_{\phi i} = \dfrac {R_i} {\rho_{\square}}, R_i - \text{номинал i-ого резистора}$ \\
    	\end{center}

    	\begin{tabular}{|c|c|}
    		\hline
    		i & $k_{\phi i}$ \\ \hline
    		1 &  3.35\\ \hline
    		2 &  0.75 \\ \hline
    		3 &  0.35\\ \hline
    		4 &  0.35 \\ \hline
    		5 &  3.35 \\ \hline
    		6 &  0.75 \\ \hline
    	\end{tabular}

	\subsection{Ширина резисторов}

    	Расчётное значение ширины каждого регистра:
        \begin{center}
        	$b \le max \{ b_{\text{точн}}, b_W\}$
        \end{center}
		Где $b_{\text{точн}}$ определяется заданной точностью изготовления

		\[ b_{\text{опт}} =
			\begin{cases}
				0.2 мм, \Delta R = \pm 20 \% \\
				0.3 мм, \Delta R = \pm 10 \%
			\end{cases}
		\]

		В данном случае $b_{\text{точн}} = 0.3$.

		$b_W$ -- значение ширины, обеспечеивающее необходимую мощность рассения
		\begin{center}
			$b_W = \sqrt {\dfrac {\rho_{\square} W} {R W_0}}$,
		\end{center}
		$W_0$ -- удельная мощность рассения плёнки, $W$ -- мощность, рассеиваемая
		на резисторе.

		\begin{tabular}{|c|c|c|c|}
			\hline
			$R$   & $b_\text{точн}$ мм & $b_W$ мм & $b$ мм \\ \hline
			$R_1$ &  0.3               & 0.4      &  0.4   \\ \hline
			$R_2$ &  0.3               & 1.8      &  1.8   \\ \hline
			$R_3$ &  0.3               & 2.0      &  2.0   \\ \hline
			$R_4$ &  0.3               & 2.0      &  2.0   \\ \hline
			$R_5$ &  0.3               & 0.4      &  0.4   \\ \hline
			$R_6$ &  0.3               & 1.8      &  1.8   \\ \hline
		\end{tabular}

	\subsection{Определение длины резисторов}
		Величина перекрытия плёночных слоёв $\delta = 0.2 \text{см}$.

		Расчётное значение $l_{\text{расч}}$ для каждого резистора
		\begin{center}
			$l_{\text{расч}} = \dfrac {R} {\rho_{\square}} b = k_{\phi} b $.
		\end{center}

		Погрешность, вызванная округлением.
		\begin{center}
			$\Delta R' = \dfrac {\left| R - R'\right|} {R} 100 \%$

			$R' = \dfrac {l' \rho_{\square}} {b}, l' \approx l$
		\end{center}

		\begin{tabular}{|c|c|c|}
			\hline
			R   & $l_{\text{расч}}$ & $\Delta R', \%$ \\ \hline
		  $R_1$ & 1.34 & 2.5 \\ \hline
		  $R_2$ & 1.35 & 3.7 \\ \hline
		  $R_3$ & 0.7  & 0 \\ \hline
		  $R_4$ & 0.7  & 0 \\ \hline
		  $R_5$ & 1.34 & 2.5 \\ \hline
		  $R_6$ & 1.35 & 3.7 \\ \hline
		\end{tabular}

		Значение погрешности удовлетворительное.

	\subsection{Определение формы резисторов}
    	\begin{tabular}{|c|c|c|c|c|}
    		\hline
    		i & $k_{\phi i}$ & $l_{\text{расч}}$ & $b$ мм & Сравнения \\ \hline
    		1 &  3.35        & 1.34              &  0.4   & $1 < k_\phi < 10, l > b$   \\ \hline 
    		2 &  0.75        & 1.35              &  1.8   & $0.1 < k_\phi < l, l < b$  \\ \hline 
    		3 &  0.35        & 0.7               &  2.0   & $0.1 < k < l, l < b$ \\ \hline 
    		4 &  0.35        & 0.7               &  2.0   & $0.1 < k < l, l < b$ \\ \hline 
    		5 &  3.35        & 1.34              &  0.4   &  $1 < k_\phi < 10, l > b$   \\ \hline 
    		6 &  0.75        & 1.35              &  1.8   & $0.1 < k_\phi < l, l < b$  \\ \hline 
    	\end{tabular}

		Из таблицы видно, что для всех резисторов выполняются условия,
		при которых их рекомендуется выполнять прямоугольной формы.

\section{Расчёт тонкоплёночных конденсаторов}
	В качестве материала для тонкоплёночных конденсаторов был
	выбран следующий материал.
	\begin{description}
		\item[Материал:] Моноокись кремния.
		\item[Материал обкладок:] Алюминий 99.
		\item[Удельная ёмкость $C_0, \frac{\text{пФ}}{\text{см}^2}$:] $(5 - 10) \dot 10^3$.
		\item[Рабочее напряжение, В:] $60-30$.
		\item[Диэлектрическая проницаемость $\epsilon$:] $5-6$.
	\end{description}
	Таким образом, при $C_0 = 1000$ площадь конденсаторов равняется:\\

	$S = \dfrac{C} {C_0} = \dfrac {510} {5000} = 0.102 (\text{см}^2)$.\\

	$a = 5.1 (\text{мм}), b = 2 (\text{мм})$.
	
\newpage
\section{Слои}
	\begin{tabular}{|c|c|c|}
		\hline
		Номер слоя & Тип слоя 			& Материал \\ \hline
		  	1	   & Резистивный 		& Кермет К-50С \\ \hline
			2      & Проводящий			& Алюминий А99 \\ \hline
			3 	   & Диэлектрический	& Моноокись кремния \\ \hline
		    4 	   & Проводящий			& Алюминий А99 \\ \hline
			5	   & Защитный 			& Моноокись германия \\ \hline 
	\end{tabular}

\section{Определение требуемой площади}

	Площадь под плёночные конденсаторы: $S_C = 0.202(\text{см}^2)$.
	Что не превышает максимальной площади в 2 $\text{см}^2$.\\

    Площадь резисторов.\\

	\begin{tabular}{|c|c|c|c|}
		\hline
            R   & $l_{\text{расч}}$, мм & $b$, мм & S, $\text{мм}^2$ \\ \hline
          $R_1$ & 1.34              	&  0.4    & 0.536 \\ \hline
          $R_2$ & 1.35              	&  1.8    & 2.43  \\ \hline
          $R_3$ & 0.7               	&  2.0    & 1.4  \\ \hline
          $R_4$ & 0.7               	&  2.0    & 1.4  \\ \hline
          $R_5$ & 1.34              	&  0.4    & 0.536  \\ \hline
          $R_6$ & 1.35              	&  1.8    & 2.43 \\ \hline
	\end{tabular}

	Суммарная площадь $S_R = 8.732 (\text{мм}^2)$.\\

	Площадь навесных элементов (всегда 2 элемента VT1 и VT2): $S_{\text{НЭ}} = 1 * 2 = 2 (\text{см}^2)$

\tikzstyle{capac}=[pattern=north east lines, preaction={fill,white}]
\tikzstyle{backl}=[pattern=north west lines, preaction={fill, white}]
\tikzstyle{resis}=[preaction={fill, white}, pattern=dots, pattern color=gray]
\tikzstyle{iarea}=[pattern=north east lines, preaction={fill, white},
                   pattern color=black]
\tikzstyle{dielc}=[dotted, fill=white]
\tikzstyle{exter}=[fill=white]

	Обозначеня на схеме.
	\begin{enumerate}
		\item Резистивный слой.

			\begin{tikzpicture}
				\draw[resis] (1,0) rectangle (0,1);
			\end{tikzpicture}

		\item Проводящий.

			\begin{tikzpicture}
				\draw[backl] (1,0) rectangle (0,1);
			\end{tikzpicture}
		\item Диэлектрический. 

			\begin{tikzpicture}
				\draw[dielc] (1,0) rectangle (0,1);
			\end{tikzpicture}
		\item Проводящий.

			\begin{tikzpicture}
				\draw[capac] (1,0) rectangle (0,1);
			\end{tikzpicture}
		\item Защитный. 

			\begin{tikzpicture}
				\draw[thick,dashed] (1,0) rectangle (0,1);
			\end{tikzpicture}
	\end{enumerate}

\newpage
\section{Схема}



\begin{center}
    \begin{tikzpicture}[pattern color=gray]
        \draw[very thin, lightgray, step=0.1] (1, 0) grid (12, 12);
        \draw[thin, gray, step=1] (1, 0) grid (12, 12);
        \draw (1, 0) rectangle (12, 12);

		% First (left) part
        \draw[dielc] (1.8, 4.6) rectangle (4.6, 10.5);
        \draw[backl]  (4.2, 4.8) --   (2, 4.8) |-
					  (4.4, 10.3) --(4.4, 3.5)
					%--(4.4, 3.5) |- (4.2, 2.7)
				    --(4.4, 1) 	 -- (4.2, 1)
					--(4.2, 1.8) --   (3.9, 1.8)
					--(3.9, 4) -- (4.2, 4) -- cycle;


        \draw[capac] (4.2, 10.1) |- (2.2, 5.0) |-
					 (4, 10.1) |- (4.2, 11.1) -- cycle;
        \draw(3.3,7.5) node {$C_1$};
		% C output
        \draw[iarea] (3.9, 11.0) rectangle (4.3, 11.4);

		% R1
        \draw[resis] (4.4, 3.8) rectangle (4.8, 2.46) node[pos=.5] {\scriptsize $R_1$};
		\draw[backl] (4.8, 4) rectangle (5, 1);
				   %--(4.8, 1) -- (4.6, 1) -- cycle;

		% R1 output
        \draw[iarea] (4.6, 0.6) rectangle (5.0, 1);
		% Middle out
        \draw[iarea] (4.2, 0.6) rectangle (4.6, 1);
		% R2
        \draw[resis] (2.55, 2) rectangle (3.9, 3.8) node[pos=.5] {\scriptsize $R_2$};
        \draw[backl] (2.35, 1) rectangle (2.55, 4);
		% R2 out
        \draw[iarea] (2.3, 1) rectangle (2.7, 0.6);

		% KP for VT1
        \draw[iarea] (5.0, 1) rectangle (5.4, 0.6);

		%% KP for VT1 out and VT2
		\draw[iarea] (5.9, 1) rectangle (6.3, 0.6) ;
		\draw[iarea] (6.3, 1) rectangle (6.7, 0.6) ;
		\draw[iarea] (6.7, 1) rectangle (7.1, 0.6) ;

		% Middle part
		% R3 out
        \draw[iarea] (5.4, 11.0) rectangle (5.8, 11.4);
		% of R3
		\draw[backl] (5.5, 11) rectangle (5.7, 8);

		% R3
        \draw[resis] (6.4, 10.4) rectangle (5.7, 8.4) node[pos=.5] {\scriptsize $R_3$};

		% middle
		\draw[backl] (6.4, 11) rectangle (6.6, 8.2);

		% R4
        \draw[resis] (7.3, 10.4) rectangle (6.6, 8.4) node[pos=.5] {\scriptsize $R_4$};
		% of R4
		\draw[backl] (7.3, 11) rectangle (7.5, 8);

		% outs
        \draw[iarea] (6.3, 11.0) rectangle (6.7, 11.4);
        \draw[iarea] (7.2, 11.0) rectangle (7.6, 11.4);

		% KP left
        \draw[iarea] (5.4, 7.6) rectangle (5.8, 8);
		% right
        \draw[iarea] (7.2, 7.6) rectangle (7.6, 8);

        \draw[exter] (5.2, 4.6) rectangle (6.2, 5.6) node[pos=.5] {$VT_1$};
        \draw (5.3, 4.7) to[bend left=10] (5.3, 0.7);
        \draw (5.8, 5.5) to[bend right=15] (5.5, 7.7);
		\draw (5.8, 4.7) to[bend left=5] (6.1, 0.7);


		% Second (right) part
        \draw[dielc] (8.4, 4.6) rectangle (11.2, 10.5);
		\draw[backl] (8.6, 4.8) |- (11.0, 10.3)
				  -- (11.0, 4.8)--   (8.8, 4.8)
				  -- (8.8, 4) --     (9.1, 4)
				  -- (9.1, 1.8) --   (8.8, 1.8)
				  -- (8.8, 1) --     (8.6, 1) -- cycle;

		\draw[capac] (8.8, 5) --    (10.8, 5)
				   --(10.8, 10.1) --(9.0, 10.1)
				   --(9.0, 11) -- (8.8, 11)-- cycle;

        \draw(10.1,7.5) node {$C_2$};

		% C output
        \draw[iarea] (9.1, 11.0) rectangle (8.7, 11.4);

		% R5
        \draw[resis] (8.2, 3.8) rectangle (8.6, 2.46) node[pos=.5] {\scriptsize $R_5$};
		\draw[backl] (8.0, 4) rectangle (8.2, 1);

		% R6
        \draw[resis] (9.1, 2) rectangle (10.45, 3.8) node[pos=.5] {\scriptsize $R_6$};
        \draw[backl] (10.45, 1) rectangle (10.65, 4);

		%% R2 out
        \draw[iarea] (10.4, 1) rectangle (10.8, 0.6);
		%% KP for VT1 out and VT2
		\draw[iarea] (7.6, 1) rectangle (8.0, 0.6);
		\draw[iarea] (8.0, 1) rectangle (8.4, 0.6);
		\draw[iarea] (8.4, 1) rectangle (8.8,   0.6);

        \draw[exter] (6.8, 4.6) rectangle (7.8,5.6) node[pos=.5] {$VT_2$};
        \draw (6.9, 4.7) to[bend right=15] (6.9, 0.7);
        \draw (7.4, 5.4) to[bend left=15] (7.4, 7.7);
		\draw (7.6, 4.7) to[bend right=10] (7.8, 0.7);

		\draw[thick, dashed] (5, 1.8) --    (2.4, 1.8) --
							 (2.4, 4.3) --  (1.6, 4.3) --
							 (1.6, 10.7) -- (11.4, 10.7) --
							 (11.4 , 4.3) --(10.7, 4.3) --
						     (10.7, 1.8) -- (8,   1.8) --
							 (8, 8.2)   --  (5, 8.2)  -- cycle ;

    \end{tikzpicture}
\end{center}

\end{document}
