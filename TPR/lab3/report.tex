\documentclass[12pt, a4paper] {ncc}
\usepackage[utf8] {inputenc}
\usepackage[T2A]{fontenc}
\usepackage[english, russian] {babel}
\usepackage[usenames,dvipsnames]{xcolor}
\usepackage{listings,a4wide,longtable,amsmath,amsfonts,graphicx,tikz}
\usepackage{indentfirst}
\usepackage{bytefield}
\usepackage{multirow}
\usepackage{float}
\usepackage{caption}
\usepackage{subcaption}
\captionsetup{compatibility=false}
\usepackage{tabularx}

\usetikzlibrary{positioning}

\usepackage[left=2cm,right=2cm,top=2cm,bottom=2cm,bindingoffset=0cm]{geometry}

\begin{document}
\setcounter{figure}{0}
\frenchspacing
\pagestyle{empty}
\begin{center}
                            Университет ИТМО    \\
                        Кафедра вычислительной техники

\vspace{\stretch{2}}

\end{center}
\vspace{\stretch{2}}
\begin{center}
			Лабораторная работа № 3\\
		по дисциплине:\\
	<<Теория принятия решений>>
\end{center}
\vspace{\stretch{3}}
\begin{flushright}
                                    Студент:\\
                                    {\it Куклина М.Д., P3401}\\
                                    Преподаватель: \\
                                    {\it Богатырев В.А. }
\end{flushright}
\vspace{\stretch{4}}
\begin{center}
                             Санкт-Петербург, 2018
\end{center}
\newpage
\section{Оптимизация в условиях неопределённости}

В рамках прошлой работы мы определили формулы для поэлементного резервирования.
Требуется найти такие $(n_K, n_P, n_M)$, что достигает оптимального значения
критерий

$$u\left(\vec n\right) \cdot C\left(\vec n\right) \rightarrow \min$$

\subsection{Описание неопределённого потока}

Зададим вектор вероятностей интенсивностей входного потока так:

$$b =
\left[\begin{aligned}
b_0 \\
b_1 \\
b_2 \\
b_3 \\
\end{aligned}\right]
= 
\left[\begin{aligned}
0.2 \\
0.4 \\
0.3 \\
0.1 \\
\end{aligned}\right]$$

\subsection{Оптимизация по критерию Байеса}

Расчёт по методу Байеса заключается в минимизации матожидания критерия:

$$\sum_i b_i \left(u(\lambda_i) \cdot C(\lambda_i)\right) \rightarrow \min$$

Реализуя эту функцию в GNU Octave, получаем результат: минимальное значение она
принимает в тривиальном случае

$$(n_K, n_M, n_P) = (1, 1, 1)$$

\subsection{Оптимизация при среднем значении интенсивности}

Среднее значение интенсивности определяется как $\sum_i b_i \lambda_i$.
Подсталяя значения, имеем результат:

$$\overline\lambda = 0.48\lambda_0 \approx 0.0209$$

Рассчитывая минимальное значение критерия при заданной интенсивности, имеем
тот же результат:

\subsection{Оптимизация при максимальной интенсивности}

Максимальной интенсивностью считаем $\lambda_0$. Тогда необходимо найти
оптимальное решение по критерию

$$u(\lambda_0) \cdot C(\lambda_0) \rightarrow \min$$

И снова результат

$$(n_K, n_M, n_P) = (1, 1, 1)$$

\subsection{Оптимизация по минимуму потерь}

При оптимизации по точке $i$ имеем критерий
$u(\lambda_i) \cdot C(\lambda_i) \rightarrow \min$

Однако обнаружилось, что во всех этих точках оказывается оптимальным одно и то
же решение: $(n_K, n_M, n_P) = (1, 1, 1)$

\subsection{Графики}

Единственным решением, которое появилось в рамках выполнения данной лабораторной
работы, оказалось $(n_K, n_M, n_P) = (1, 1, 1)$. Построим график зависимости
среднего времени ожидания для данного случая. Он определяется формулой:

$$w(\lambda) = \dfrac{1}{\mu_K - \lambda} +
              \dfrac{1}{\mu_P - \frac \lambda 3} +
              \dfrac{1}{\mu_M - \frac \lambda 3} - b_P - b_K - b_M$$

После подстановки констант и некоторых преобразований получаем

$$w(\lambda) = \dfrac{39}{3 - 13\lambda} + \dfrac{5}{1 - 5\lambda} +
               \dfrac{69}{3 - 23\lambda} - 41$$

\begin{tikzpicture}[domain=0:1/23,yscale=1,xscale=300,samples=400]
\draw[->,thick] (0, 0) node[below] {$0$} -- (1/23, 0) node[below] {$\lambda$};
\draw[->,thick] (0, 0) -- (0, 16) node[left] {$w(\lambda)$};
\draw[very thin] (0, 0) grid[ystep=.5,xstep=1/600] (1/23, 15.99);
    \foreach \i in {0.010,0.020,0.030,0.040} {
        \draw (\i,1/10) -- (\i,-1/10) node[below] {$\i$};
    }
    \foreach \i in {1, 2, ..., 15} {
        \draw (1/3000,\i) -- (-1/3000,\i) node[left] {$\i$};
    }
\draw[color=red] plot (\x,{39/(3 - 13*\x) + 5/(1 - 5 * \x) +
    69 / (3 - 23 * \x) - 41});
\end{tikzpicture}

\subsection{Сравнение методов оптимизации}

Так как все методы оптимизации дали один и тот же результат, сложно их
сопоставить. Однако интуитивно понятно, что метод Байеса более точный, поскольку
учитывает вероятности тех или иных нагрузок. В то же время метод, заключающийся
в выборе наихудшего случая и оптимизации под него, весьма пессимистичный, но
может быть полезен, когда ожидается, что система часто будет оперировать в
режиме наибольшей нагрузки. Наконец, метод поиска решений путём взятия среднего
значения интенсивности позволяет оптимально адресовать случаи, когда ожидаются
лишь малые флунктуации относительно матожидания, то есть при распределении
нагрузки, обладающим малым среднеквадратическим отклонением.

\subsection{Выводы}

Обнаружилось, что рассматриваемая система ведёт себя оптимально во всех
рассмотренных случаях даже без резервирования, если в качестве критерия
оптимизации принимать $C(\lambda) \cdot u(\lambda) \rightarrow \min$. Это
обусловлено малым приростом производительности при добавлении коммутаторов и
высокой стоимостью устройств памяти и обработки при высоком их числе, что
приводит к тому, что их резервирование требует сразу большого количества денег.


\end{document}
