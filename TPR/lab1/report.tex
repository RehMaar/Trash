\documentclass[12pt, a4paper] {ncc}
\usepackage[utf8] {inputenc}
\usepackage[T2A]{fontenc}
\usepackage[english, russian] {babel}
\usepackage[usenames,dvipsnames]{xcolor}
\usepackage{listings,a4wide,longtable,amsmath,amsfonts,graphicx,tikz}
\usepackage{indentfirst}
\usepackage{bytefield}
\usepackage{multirow}
\usepackage{float}
\usepackage{caption}
\usepackage{subcaption}
\captionsetup{compatibility=false}
\usepackage{tabularx}

\usetikzlibrary{positioning}

\usepackage[left=2cm,right=2cm,top=2cm,bottom=2cm,bindingoffset=0cm]{geometry}

\begin{document}
\setcounter{figure}{0}
\frenchspacing
\pagestyle{empty}
\begin{center}
                            Университет ИТМО    \\
                        Кафедра вычислительной техники

\vspace{\stretch{2}}

\end{center}
\vspace{\stretch{2}}
\begin{center}
			Лабораторная работа № 1\\
		по дисциплине:\\
	<<Теория принятия решений>>
\end{center}
\vspace{\stretch{3}}
\begin{flushright}
                                    Студент:\\
                                    {\it Куклина М.Д., P3401}\\
                                    Преподаватель: \\
                                    {\it Богатырев В.А. }
\end{flushright}
\vspace{\stretch{4}}
\begin{center}
                             Санкт-Петербург, 2018
\end{center}
\newpage

\section{Задание}
	Стоимость блока обработки $C = 5$, системы хранения $C = 6$ и коммутатора $C = 4$.

	Среднее время обслуживания запроса в блоке обработки $V = 4$, систем ханения $V = 2$ и коммутаторов $V = 3$.

	Вероятности работоспособности (коэффициент готовности) блока обработки $p = 0.95$,
	системы хранения $p = 0.95$ и коммутатора: $p = 0.96$.

	Интенсивность входного потока: $\lambda = 0.9 \lambda_0$, $\lambda_0$ -- рассчитываемое
	значение предельной интенсивности входного потока, выдерживаемой конфигурацией.

	Структуры: 1, 2, 14, 19, 16.\\

\section{Структура 1}

	\includegraphics[scale=0.5]{./1.png}
    \subsection{Оценка надёжности структуры}

		Для данной системы нужно, чтобы работали два элемента P и М. \\
	    Вероятность безотказной работы: $P_1 = P_P \cdot P_M$.\\
		$P_1 = 0.95 \cdot 0.95 = 0.9025$


        \subsection{Оценка среднего времени пребывания запросов в системе}

			М/М/1 имеет следующий вид.\\

            \tikzset{my2/.style={circle, draw, black, fill=white}}
            \begin{center}
                \begin{tikzpicture}[node distance=3em]
                    \node (s) {};
                    \node[my2, right = of s] (p) {$\mu_P$};
                    \node[my2, right = of p] (m) {$\mu_M$};
                    \node[right = of m] (e) {};
                    \draw[->] (s) -- node[midway,above] {} (p);
                    \draw[->] (p) -- node[midway,above] {} (m);
                    \draw[->] (m) -- node[midway,above] {} (e);
                \end{tikzpicture}
            \end{center}

			$u = \dfrac {1} {\mu - \lambda} $ -- время пребывания заявки в системе.

			В данном случае:

			$u_1 = u_P + u_M = \dfrac {1} {\mu_p - \lambda} + \dfrac {1} {\mu_M - \lambda}$.

        %\subsection{Предельно допустимая интенсивность запросов}
        %\subsection{Затраты на построение системы}
        %\subsection{Облать эффективных решений}
        %	\subsubsection{Определение области Парето}
        %	\subsubsection{Выбор наилушего варианта}
        %\subsection{Сравнение результатов многокритериального выбора структуры вычислительной системы}
        %\subsection{Выбор варианта структуры}
        %\subsection{Рекомендации по улучшению выбранного варианта}


\section{Структура 2}

	\includegraphics[scale=0.5]{./2.png}
    \subsection{Оценка надёжности структуры}
		Для данной систему нужно, чтобы работал коммутатор и либо права ветка,
				  левая ветка полностью: \\
				  $P_P \cdot P_M$ -- вероятность работы одной ветки.\\
				  $1 - P_P \cdot P_M$ -- вероятность, что одна ветка работать не будет.\\
				  $(1 - P_P \cdot P_M)^2$ -- вероятность, что две ветки работать не будут.\\
				  $1 - (1 - P_P \cdot P_M)^2$ -- вероятность того, что хотя бы одна ветка должна работать.\\
				  $P_2 = P_K \cdot (1 - (1 - P_P \cdot P_M)^2)$ \\ 
				  $P_2 = 0.96 \cdot (1 - (1 - 0.95 \cdot 0.95) ^ 2 = 0.8664$

        \subsection{Оценка среднего времени пребывания запросов в системе}
            \begin{center}
                \begin{tikzpicture}[node distance=3em and 3em]
                    \node (s) {};
                    \node[my2, right = of s] (k)  {$\mu_K$};
                    \node[right = of  k] (q2) {};
                    \node[above = of q2] (q1) {};
                    \node[my2, right = of q1] (p1) {$\mu_P$};
                    \node[my2, right = of q2] (p2) {$\mu_P$};
                    \node[my2, right = of p1] (m1) {$\mu_M$};
                    \node[my2, right = of p2] (m2) {$\mu_M$};
                    \node[right = of m1] (e1) {};
                    \node[right = of m2] (e2) {};
                    \draw[->] (s) -- node[midway,above] {} (k);
                    \draw (k) -- node[midway,above]     {} (q2.center);
                    \draw (q1.center) -- (q2.center);
                    \draw[->] (q1.center) -- node[midway,above] {} (p1);
                    \draw[->] (q2.center) -- node[midway,above] {} (p2);
                    \draw[->] (p1)        -- node[midway,above] {} (m1);
                    \draw[->] (p2)        -- node[midway,above] {} (m2);
                    \draw[->] (m1)        -- node[midway,above] {} (e1);
                    \draw[->] (m2)        -- node[midway,above] {} (e2);
                \end{tikzpicture}
            \end{center}		

			$u_{2} = u_K + 2 \cdot (\dfrac {1}{2} (u_P + u_M)) = \dfrac {1} {\mu_K - \lambda} + \dfrac {1} {\mu_P - \frac{\lambda}{2}} + \dfrac {1} {\mu_M - \frac{\lambda} {2}}$

        %\subsection{Предельно допустимая интенсивность запросов}
        %\subsection{Затраты на построение системы}
        %\subsection{Облать эффективных решений}
        %	\subsubsection{Определение области Парето}
        %	\subsubsection{Выбор наилушего варианта}
        %\subsection{Сравнение результатов многокритериального выбора структуры вычислительной системы}
        %\subsection{Выбор варианта структуры}
        %\subsection{Рекомендации по улучшению выбранного варианта}



\section{Структура 14}

	\includegraphics[scale=0.5]{./14.png}


    \subsection{Оценка надёжности структуры}

		Для данной системы нужно, чтобы всегда работал коммутатор и
   	    одна из трёх веток; при работе второй или третьей веток нужна
   	    работа коммутатора и один из трёх элементов памяти.\\
  	    $P_P \cdot P_M$ -- вероятность работы первой ветки.\\
   	    $(1 - P_P \cdot P_M)$ -- вероятность, что первая ветка не работает.\\
   	    $(1 - P_M)$ -- вероятность, что откажет элемент памяти.\\
   	    $(1 - P_M)^3$ -- вероятность, что откажут три элемента памяти.\\
   	    $1 - (1 - P_M)^3$ -- вероятность, что не откажет хотя бы один элемент.\\
   	    $P_K' = P_K \cdot (1 - (1 - P_M)^3)$ -- вероятность, что не откажет хотя бы один элемент при рабочем коммутаторе.\\
   	    $(1 - P_P)^2 \cdot P_K'$ -- вероятность, что ни одна ветка не работает.\\
		$P_{14} = P_K \cdot (1 - (1 - P_P \cdot P_M) \cdot (1 - P_P)^2 P_K') = $\\
		$= P_K \cdot (1 - (1 - P_P \cdot P_M) \cdot (1 - P_P)^2 \cdot P_K \cdot (1 - (1 - P_M)^3))$ \\
		$P_{14} = 0.949$

        \subsection{Оценка среднего времени пребывания запросов в системе}
            \begin{center}
                \begin{tikzpicture}[node distance=3em and 3em]
                    \node (s) {};
                    \node[my2, right = of s] (k)  {$\mu_K$};
                    \node[right = of  k] (q2) {};
                    \node[above = of q2] (q1) {};
                    \node[below = of q2] (q3) {};
                    \node[right = of q3] (q4) {};
                    \node[below = of q4] (q5) {};
                    \node[above = of q4] (q6) {};

                    \node[my2, right = of q1] (p1) {$\mu_P$};
                    \node[my2, right = of p1] (m1) {$\mu_M$};

                    \node[my2, right = of q5] (p2) {$\mu_P$};
                    \node[my2, right = of q6] (p3) {$\mu_P$};

					\node[right = of p3] (q7) {};
					\node[below = of q7] (q8) {};
					\node[below = of q8] (q9) {};

                    \node[my2, right = of q8] (k1) {$\mu_K$};

					\node[my2, above right = of k1] (m2) {$\mu_M$};
					\node[my2, below = of m2] (m3) {$\mu_M$};
					\node[my2, below = of m3] (m4) {$\mu_M$};

                    \node[right = of m1] (e1) {};
                    \node[right = of m2] (e2) {};
                    \node[right = of m3] (e3) {};
                    \node[right = of m4] (e4) {};

                   \draw[->] (s) -- node[midway,above] {} (k);
                   \draw     (k) -- node[midway,above] {} (q2.center);
                   \draw     (q1.center) -- (q2.center) -- (q3.center);
				   \draw     (q3.center) -- (q4.center) -- (q5.center) -- (q6.center);
                   \draw[->] (m1)        -- node[midway,above] {} (e1);

                   \draw[->] (q1.center) -- node[midway,above] {} (p1);
                   \draw[->] (p1)        -- node[midway,above] {} (m1);


                   \draw[->] (q5.center) -- node[midway,above] {} (p2);
                   \draw[->] (q6.center) -- node[midway,above] {} (p3);
				   \draw (p2) -- (q9.center) -- (q8.center) -- (q7.center) -- (p3);
				   \draw[->] (q8.center) -- (k1);

                   \draw[->] (k1)        -- node[midway,above] {} (m2);
                   \draw[->] (k1)        -- node[midway,above] {} (m3);
                   \draw[->] (k1)        -- node[midway,above] {} (m4);
                   \draw[->] (m2)        -- node[midway,above] {} (e2);
                   \draw[->] (m3)        -- node[midway,above] {} (e3);
                   \draw[->] (m4)        -- node[midway,above] {} (e4);
                \end{tikzpicture}
            \end{center}


		$u = u_k + u_1 + u_2 = 
		\dfrac {1} {\mu_K - \lambda} + 
		\dfrac {1} {\mu_P - \frac {\lambda} {2}} + \dfrac {1} {\mu_M - \frac {\lambda} {2}} +
		\dfrac {2} {\mu_P - \frac {\lambda}{4}} +
		\dfrac {1} {\mu_K - \frac {\lambda}{2}} +
		\dfrac {3} {\mu_K - \frac {\lambda}{6}}$

        %\subsection{Предельно допустимая интенсивность запросов}
        %\subsection{Затраты на построение системы}
        %\subsection{Облать эффективных решений}
        %	\subsubsection{Определение области Парето}
        %	\subsubsection{Выбор наилушего варианта}
        %\subsection{Сравнение результатов многокритериального выбора структуры вычислительной системы}
        %\subsection{Выбор варианта структуры}
        %\subsection{Рекомендации по улучшению выбранного варианта}


\section{Структура 19}
	\includegraphics[scale=0.5]{./19.png}

    \subsection{Оценка надёжности структуры}

		Для данной системы нужно, чтобы всегда работал первый коммутатор и одна из веток;\\
	    во второй ветку нужно, чтобы работал один из коммутаторов.\\
	    $P_P \cdot P_M$ -- вероятность работы первой ветки.\\
	    $P_K \cdot P_M$ -- вероятность работы связки коммутатор-память.\\
	    $1 - P_K \cdot P_M$ -- вероятность, что связка коммутатор память-работать не будет.\\
	    $(1 - P_K \cdot _PM)^2$ -- вероятность, что ни одна связка коммутатор-память работать не будет.\\
	    $1 - (1 - P_K \cdot P_M)^2$ -- вероятность, что будет работать хотя бы одна связка.\\
	    $P_P \cdot (1 - (1 - P_K \cdot P_M)^2)$ -- вероятность того, что будет работать блок обработки и его зависимости.\
	    $1 - P_P \cdot (1 - (1 - P_K \cdot P_M)^2)$ -- вероятность того, что эта часть работать не будет.\\
	    $(1 - P_P \cdot P_M) \cdot (1 - P_P \cdot (1 - (1 - P_K \cdot P_M)^2))$ -- вероятность того, что две ветки не будут работать.\\
	    $P_{19} = P_K \cdot (1 - (1 - P_P \cdot P_M) \cdot (1 - P_P \cdot (1 - (1 - P_K \cdot P_M)^2)))$.\\
		$P_{19} = 0.955$

        \subsection{Оценка среднего времени пребывания запросов в системе}
            \begin{center}
                \begin{tikzpicture}[node distance=3em and 3em]
                    \node (s) {};
                    \node[my2, right = of s] (k)  {$\mu_K$};
					\node[my2, above right = of k] (p1) {$\mu_P$};
					\node[my2, right = of p1] (m1) {$\mu_M$};
					\node[my2, below right = of k] (p2) {$\mu_P$};
					\node[my2, right = of p2] (k1) {$\mu_K$};
					\node[my2, right = of k1] (m2) {$\mu_M$};
					\node[my2, below = of k1] (k2) {$\mu_K$};
					\node[my2, right = of k2] (m3) {$\mu_M$};
					\draw[->] (s) -- (k);
					\draw[->] (k) -- (p1);
					\draw[->] (p1) -- (m1);
					\draw[->] (k) -- (p2);
					\draw[->] (p2) -- (k1);
					\draw[->] (p2) -- (k2);
					\draw[->] (k1) -- (m2);
					\draw[->] (k2) -- (m3);
				\end{tikzpicture}
			\end{center}

			$u = \dfrac {1} {\mu_K - \lambda} + \dfrac {2} {\mu_K - \frac {\lambda}{4}} +
			 \dfrac {1} {\mu_P - \frac {\lambda} {2}} + \dfrac {1} {\mu_P - \frac {\lambda}{2}} +
			 \dfrac {1} {\mu_M - \frac {\lambda} {2}} + 
			 \dfrac {2} {\mu_M - \frac {\lambda} {4}}$


        %\subsection{Предельно допустимая интенсивность запросов}

        %\subsection{Затраты на построение системы}
        %\subsection{Облать эффективных решений}
        %	\subsubsection{Определение области Парето}
        %	\subsubsection{Выбор наилушего варианта}
        %\subsection{Сравнение результатов многокритериального выбора структуры вычислительной системы}
        %\subsection{Выбор варианта структуры}
        %\subsection{Рекомендации по улучшению выбранного варианта}


\section{Структура 16}

	\includegraphics[scale=0.5]{./16.png}
    \subsection{Оценка надёжности структуры}

		$P_K' = (1 - (1 - P_M)^2) \cdot P_K$ -- работа коммутатора с одним из элементов памяти. \\
		$P_P' = (1 - (1 - P_K')^2) \cdot P_P$ -- работа одного из блоков обработчика.\\
		$P_{16} = P_K \cdot (1 - (1 - P_P')^2) = $ \\
		$= P_K \cdot (1 - (1 - (1 - (1 - (1 - (1 - P_M)^2) \cdot P_K)^2) \cdot P_P )^2)$\\
		$P_{16} = 0.957$


        \subsection{Оценка среднего времени пребывания запросов в системе}
            \begin{center}
                \begin{tikzpicture}[node distance=2em and 2em]
                    \node (s) {};
                    \node[my2, right = of s] (k)  {$\mu_K$};
                    \node[right = of  k] (q2) {};
                    \node[above = of q2] (q1) {};
                    \node[below = of q2] (q3) {};
					\node[my2, right = of q1] (p1) {$\mu_P$};
					\node[my2, right = of q3] (p2) {$\mu_P$};
					\node[right = of p1] (t1) {};
					\node[below = of t1] (t2) {};
					\node[below = of t2] (t3) {};
					
					\node[right = of t2] (r2) {};
					\node[above = of r2] (r1) {};
					\node[below = of r2] (r3) {};
					\node[my2, right = of r1] (k1) {$\mu_K$};
					\node[my2, right = of r3] (k2) {$\mu_K$};
					\node[right = of k1] (e1) {};
					\node[below = of e1] (e2) {};
    				\node[below = of e2] (e3) {};

					\node[right = of e1] (w1) {};
					\node[right = of e2] (w2) {};
					\node[right = of e3] (w3) {};
					\node[my2, right = of w1] (m1) {$\mu_M$};
					\node[my2, right = of w3] (m2) {$\mu_M$};

					\node[right = of m1] (o1) {};
					\node[right = of m2] (o2) {};

					\draw[->] (s) -- (k);
					\draw (k) -- (q2.center) -- (q1.center);
					\draw (q2.center) -- (q3.center);
					\draw (q1.center) -- (p1);
					\draw (q3.center) -- (p2);
					\draw (p1) -- (t1.center) -- (t2.center) -- (t3.center) -- (p2);
					\draw (t2.center) -- (r2.center);
					\draw (e2.center) -- (w2.center);
					\draw (k1) -- (r1.center) -- (r2.center) -- (r3.center) -- (k2);
					\draw (k1) -- (e1.center) -- (e2.center) -- (e3.center) -- (k2);
					\draw (m1) -- (w1.center) -- (w2.center) -- (w3.center) -- (m2);
					\draw[->] (m1) -- (o1.center);
					\draw[->] (m2) -- (o2.center);

				\end{tikzpicture}
			\end{center}

			$u = \dfrac {1} {\mu_k - \lambda} + \dfrac {1} {\mu_P - \frac{\lambda}{2}} + \dfrac {1} {\mu_K - \frac {\lambda}{2}} + \dfrac {1}{\mu_M - \frac {\lambda}{2}}$



        %\subsection{Затраты на построение системы}
        %\subsection{Облать эффективных решений}
        %	\subsubsection{Определение области Парето}
        %	\subsubsection{Выбор наилушего варианта}
        %\subsection{Сравнение результатов многокритериального выбора структуры вычислительной системы}
        %\subsection{Выбор варианта структуры}
        %\subsection{Рекомендации по улучшению выбранного варианта}
\section{Предельно допустимая интенсивность запросов}


Максимальная интенсивность, при которой не происходит перегрузка, легко
определяется так: в каждом знаменателе при расчёте среднего времени
пребывания в случае без перегрузки положительное число. Исходя из
 этого, имеем:

	\begin{description}
		\item[1:]  $\lambda < min(\mu_P, \mu_M)$
		\item[2:]  $\lambda < min(2\mu_P, 2\mu_M, \mu_K)$
		\item[14:] $\lambda < min(4\mu_P, 2\mu_M, 6\mu_K)$
		\item[19:] $\lambda < min(2\mu_P, 4\mu_M, 4\mu_K)$
		\item[16:] $\lambda < min(2\mu_P, 2\mu_M, 2\mu_K)$
	\end{description}

	Таким образом, интенсивность, не вызывающая перегрузки системы:
	$\lambda_0 < min(\mu_P, \mu_M, \mu_K)$.

	Тогда получаем:

    \begin{tabular}{|c|c|}
    \hline
    $N$    & $\lambda$ \\ \hline
    \hline
    \bf 1  & $0.0435$ \\ \hline
    \bf 2  & $0.0769$ \\ \hline
    \bf 14 & $0.1304$ \\ \hline
    \bf 19 & $0.0870$ \\ \hline
    \bf 16 & $0.0435$ \\ \hline
    \hline
    \bf 0  & $0.0435$ \\ \hline
    \end{tabular}

\subsection{Расчёт среднего времени ожидания}

Мы имеем формулы вида $\sum\limits_i \dfrac{1}{k_i \mu_i - b_i \lambda}$ для
расчёта среднего времени пребывания. В таком случае формулой для
соответствующего времени ожидания будет

$$\sum\limits_i \dfrac{1}{k_i \mu_i - b_i\lambda} - \dfrac{1}{k_i \mu_i} =
u - \sum\limits_i \dfrac{b_i}{k_i}$$

\begin{description}
\item[1.]   $w = u - b_P - b_M$
\item[2.]   $w = u - b_P - b_M - 2 b_K$
\item[14.]  $w = u - b_P - b_K - b_M$
\item[19.]  $w = u - b_P - 2 b_K - b_M$
\item[16.]  $w = u - b_P - b_K - b_M$
\end{description}

Подставляем $\lambda = 0.7 \lambda_0 = 0.7 \mu_P$:

\begin{tabular}{|c|c|c|}
\hline
$N$    & $u$     & $w$     \\ \hline
\bf 1  & $98.18$ & $62.18$ \\ \hline
\bf 2  & $84.50$ & $38.50$ \\ \hline
\bf 14 & $50.87$ & $ 9.87$ \\ \hline
\bf 19 & $59.30$ & $15.80$ \\ \hline
\bf 16 & $98.28$ & $57.28$ \\ \hline
\end{tabular}
\subsection{Расчёт затрат на построение}

Требуется всего лишь подсчитать количество элементов каждого вида.

\begin{description}
\item[1.]  $C_P + C_M$
\item[2.]  $2 \cdot C_P + C_M + C_K$
\item[14.] $3 \cdot C_P + 3 \cdot C_M + C_K$
\item[19.] $2 \cdot C_P + 3 \cdot C_M + 2 \cdot C_K$
\item[16.] $C_P + 2 \cdot C_M + 2 \cdot C_K$
\end{description}

На конкретных значениях получим:

\begin{tabular}{|c|c|}
\hline
$N$    & $C$ \\ \hline
\bf 1  & $11$ \\ \hline
\bf 2  & $20$ \\ \hline
\bf 14 & $37$ \\ \hline
\bf 19 & $36$ \\ \hline
\bf 16 & $25$ \\ \hline
\end{tabular}

\subsection{Определение области эффективных решений}

Представим все варианты в порядке ухудшения каждой характеристики:

% p1 = 0.9025
% p2 = 0.8664
% p14 = 0.949
% p19 = 0.955
% p16 = 0.957

\begin{tabular}{|c|c|c|c|}
\hline
$P$ & $u$ & $\Lambda$ & $C$ \\ \hline
\hline
16  & 14  & 14        &  1 \\ \hline
19  & 19  & 19        & 16 \\ \hline
14  &  2  &  2        &  2 \\ \hline
1   &  1  &  1        & 19 \\ \hline
2   & 16  & 16        & 14 \\ \hline
\end{tabular}

Обнаруживаем, что ни один вариант не лучше никакого другого по всем критериям.

\subsection{Поиск наилучшего варианта}

Для поиска варианта требуется выбрать способ нормализации значений: нельзя
сравнивать доллары и секунды, нужно перейти к безразмерным величинам.

Существуют, к примеру, такие способы нормализации:

\begin{tabular}{|c|c|}
\hline
\bf Название & \bf Формула \\ \hline
Естественная & $\dfrac{f(i) - \min_i f(i)}{\max_i f(i) - \min_i f(i)}$ \\ \hline
Относительная & $\dfrac{f(i)}{\max_i f(i)}$ \\ \hline
Контекстуальная & $\dfrac{f(i) - t}{T - t}$ \\ \hline
\end{tabular}

Под $t$ и $T$ при контекстуальной нормализации подразумеваются заранее
определённые максимальные и минимальные значения: к примеру, имеющийся бюджет
может быть определён как наибольшая допустимая стоимость, а заранее определённый
ожидаемый поток посетителей~--- как наименьшая интенсивность входящих заявок.

Мы выбираем контекстуальную нормализацию с такими минимальными и максимальными
значениями:

\begin{description}
\item[Надёжность] Максимальная возможная надёжность~--- $100\%$, минимальная~---
та, при которой не происходит никакого резервирования и заявка должна пройти
через обработчик и устройство памяти: $P_M \cdot P_P = 86.49\%$.
\item[Среднее время пребывания] Очевидно, что в идеальном случае заявка
находится в системе ровно столько, сколько необходимо, чтобы она прошла через
устройства обработки и хранения: $u = u_M + u_P = 13 + 23 = 36$. Худшим случаем
является бесконечное время ожидания, однако зададим конечную величину, значения
выше которой мы считаем неприемлемыми: $108$, в три раза больше минимальной.
\item[Интенсивность] Определим границы как $0.01$ и $0.4$.
\item[Стоимость] В стоимость входят как минимум один обработчик и одно
устройство памяти: $C = C_P + C_M = 15 + 6 = 21$. За максимальную стоимость
примем $250$, данное в условии к следующей лабораторной работе.
\end{description}

Условимся, что $1.0$~--- наилучшее значение для данного критерия, $0.0$~---
наихудшее. Те критерии, которые требуется минимизировать, а не максимизировать,
следует преобразовать так: $f_n(i) := 1.0 - f_n(i)$.

Получаем:

\begin{tabular}{|c||c|c|c|c|}
\hline
$N$ & $P$    & $u$     & $\Lambda$ & $C$ \\ \hline
\hline
1   & $0.00$ & $0.136$ & $0.09$    & $1.00$ \\ \hline
2   & $0.04$ & $0.326$ & $0.17$    & $0.93$ \\ \hline
14  & $0.54$ & $0.793$ & $0.31$    & $0.81$ \\ \hline
19  & $0.43$ & $0.676$ & $0.20$    & $0.87$ \\ \hline
16  & $0.42$ & $0.135$ & $0.09$    & $0.96$ \\ \hline
\end{tabular}

\subsubsection{Минимаксный критерий}

Минимаксный метод заключается в том, что к матрице нормализованных значений
критериев приписывается дополнительный столбец, в котором размещается наименьшее
значение на данной строке. В качестве принимаемого решения выбирается то, на
строке которого в дополнительном столбце наибольшее число.

$$\min_i f_n(i) \rightarrow \max$$

\begin{tabular}{|c||c|c|c|c||c|}
\hline
$N$ & $P$    & $u$     & $\Lambda$ & $C$    &        \\ \hline
\hline
1   & $0.00$ & $0.136$ & $0.09$    & $1.00$ & $0.00$ \\ \hline
2   & $0.04$ & $0.326$ & $0.17$    & $0.93$ & $0.04$ \\ \hline
14  & $0.54$ & $0.793$ & $0.31$    & $0.81$ & $0.31$ \\ \hline
19  & $0.43$ & $0.676$ & $0.20$    & $0.87$ & $0.20$ \\ \hline
16  & $0.42$ & $0.135$ & $0.09$    & $0.96$ & $0.09$ \\ \hline
\end{tabular}

Мы выбрали решение $14$: самая слабая его сторона~--- пропускная способность, но
у остальных она ещё меньше. Эта система стоит значительно больше, чем остальные,
однако стоимость для нас не так значима.

\subsubsection{Мультипликативный критерий}

Мультипликативный метод заключается в максимизации произведения значений разных
критериев:

$$\prod_i f_n(i) \rightarrow \max$$

\begin{tabular}{|c||c|c|c|c||c|}
\hline
$N$ & $P$    & $u$     & $\Lambda$ & $C$    &          \\ \hline
\hline
1   & $0.00$ & $0.136$ & $0.09$    & $1.00$ & $0.0000$ \\ \hline
2   & $0.04$ & $0.326$ & $0.17$    & $0.93$ & $0.0021$ \\ \hline
14  & $0.54$ & $0.793$ & $0.31$    & $0.81$ & $0.1075$ \\ \hline
19  & $0.43$ & $0.676$ & $0.20$    & $0.87$ & $0.0506$ \\ \hline
16  & $0.42$ & $0.135$ & $0.09$    & $0.96$ & $0.0049$ \\ \hline
\end{tabular}

Очевидный победитель~--- решение $14$.

\subsubsection{Аддитивный критерий}

Требуется максимизировать сумму значений разных критериев:

$$\sum_i f_n(i) \rightarrow \max$$

\begin{tabular}{|c||c|c|c|c||c|}
\hline
$N$ & $P$    & $u$     & $\Lambda$ & $C$    &         \\ \hline
\hline
1   & $0.00$ & $0.136$ & $0.09$    & $1.00$ & $1.226$ \\ \hline
2   & $0.04$ & $0.326$ & $0.17$    & $0.93$ & $1.466$ \\ \hline
14  & $0.54$ & $0.793$ & $0.31$    & $0.81$ & $2.453$ \\ \hline
19  & $0.43$ & $0.676$ & $0.20$    & $0.87$ & $2.176$ \\ \hline
16  & $0.42$ & $0.135$ & $0.09$    & $0.96$ & $1.605$ \\ \hline
\end{tabular}

На первом месте по аддитивному критерию~--- решение $14$, самое дорогое, но в
остальных аспектах лидирующее. На втором месте $19$, которое по всем параметрам
рядом с $14$.

\subsubsection{Метод отклонения от идеала}

Идеальный прибор мы определяем как обладающий максимальными значениями,
описанными при выборе метода нормализации.

Его характеристики, таким образом, выглядят так:

\begin{tabular}{|c||c|c|c|}
\hline
$P$    & $u$     & $\Lambda$ & $C$ \\ \hline
\hline
$1.00$ & $1.00$  & $1.00$    & $1.00$ \\ \hline
\end{tabular}

Теперь определим качество прибора как его близость к идеальному:

$$\sum_i f_I(i) - f_n(i) \rightarrow \min$$

Несложно заметить, что в силу выбора нормализации результат в данном случае
совпадает с результатом выбора по аддитивному критерию.

\subsection{Выводы}

В результате проделанной работы мы применили свои знания в теории надёжности,
математическом моделировании и базовой арифметике для получения численных метрик
вычислительной системы, а также изучили основные методы принятия решений в
условиях, где не заданы заранее критерии оптимизации.

\end{document}
