\documentclass[12pt, a4paper] {ncc}
\usepackage[utf8] {inputenc}
\usepackage[T2A]{fontenc}
\usepackage[english, russian] {babel}
\usepackage[usenames,dvipsnames]{xcolor}
\usepackage{listings,a4wide,longtable,amsmath,amsfonts,graphicx,tikz}
\usepackage{indentfirst}
\usepackage{bytefield}
\usepackage{multirow}
\usepackage{float}
\usepackage{caption}
\usepackage{subcaption}
\captionsetup{compatibility=false}
\usepackage{tabularx}

\usetikzlibrary{positioning}

\usepackage[left=2cm,right=2cm,top=2cm,bottom=2cm,bindingoffset=0cm]{geometry}

\begin{document}
\setcounter{figure}{0}
\frenchspacing
\pagestyle{empty}
\begin{center}
                            Университет ИТМО    \\
                        Кафедра вычислительной техники

\vspace{\stretch{2}}

\end{center}
\vspace{\stretch{2}}
\begin{center}
			Лабораторная работа № 2\\
		по дисциплине:\\
	<<Теория принятия решений>>
\end{center}
\vspace{\stretch{3}}
\begin{flushright}
                                    Студент:\\
                                    {\it Куклина М.Д., P3401}\\
                                    Преподаватель: \\
                                    {\it Богатырев В.А. }
\end{flushright}
\vspace{\stretch{4}}
\begin{center}
                             Санкт-Петербург, 2018
\end{center}
\newpage

\section*{Исходные данные}
На основании прошлой работы была выбрана структура 14.
Без резервирования имеем систему:


	\includegraphics[scale=0.5]{./14.png}

Параметры при известных входных данных:

\begin{tabular}{|c||c|c|c|c|}
\hline
$P$       & $u$     & $\Lambda$ & $C$ \\ \hline
\hline
$94.9 \%$ & $50.87$ & $0.1304$  & $37$ \\ \hline
\end{tabular}


\subsection*{Поэлементное резервирование}

\subsubsection*{Анализ}

Требуется обнаружить такую комбинацию $(n_K, n_P, n_M)$, что, пусть мы остаёмся
в рамках бюджета, достигается $\min u, \min C, \max P, \max \Lambda$.

Чтобы определить надёжность требуемой системы, требуется в формуле надёжности и
сходной системы каждое вхождение $P_a$ заменить на $\left(1 - (1 -
P_a)^{n_a}\right)$: элемент заменяется на $n_a$ идентичных ему, и этой формулой
задаётся вероятность того, что работает хотя бы один из элементов.

Рассмотрим, что происходит со среднем временем ожидания при поэлементном
резервировании. Итак, каждый элемент в сумме $u = \sum_i \dfrac{1}{\mu_i -
\lambda_i}$ заменяется, если ему соответствует прибор типа $a$, на $n_a$
идентичных ему элементов. При этом поток в каждый из этих элементов будет в
$n_a$ раз меньше, чем у исходного, как и вероятность попадания в этот
конкретный элемент: $\dfrac{1} {\mu_i - \lambda_i} \rightarrow \sum^{n_a}
\frac{1}{n_a} \dfrac{1}{\mu_i - \frac {\lambda_i}{n_a}} =
\dfrac{1}{\mu_i - \frac{\lambda_i}{n_a}}$. Таким образом, в каждом слагаемом в
выражении для среднего времени ожидания необходимо лишь заменить $\lambda$ на
$\frac{\lambda}{n}$.

Это позволяет быстро определить, как переход к поэлементному резервированию
влияет на допускаемый поток входящих заявок: раз входной поток в каждый узел
типа $a$ становится в $n_a$ раз меньше, значит, он может быть в $n_a$ раз
больше, чем раньше. Соответственно, в каждом аргументе к $\min$ мы увеличиваем
дозволенный поток в $n_a$ раз.

Со стоимостью ещё проще: достаточно в формуле для её вычисления $C_a$ домножить
на $n_a$.

Результирующие формулы таковы:

$$\begin{aligned}
P &= \left(1 - (1 - P_K)^{n_K}\right) \cdot
     \left(1 - \left(1 -
               \left(1 - (1 - P_M)^{n_M}\right) \cdot
               \left(1 - (1 - P_P)^{n_P}\right)
               \right)^3\right) \\
u &= 
       \dfrac {1} {\mu_K - \frac{\lambda}{n_K}} +
       \dfrac {1} {\mu_P - \frac {\lambda}{2n_P}} +
	   \dfrac {1} {\mu_M - \frac {\lambda} {2n_M}} +
       \dfrac {2} {\mu_P - \frac {\lambda}{4}n_P} +
       \dfrac {1} {\mu_K - \frac {\lambda}{2}n_K} +
       \dfrac {3} {\mu_K - \frac {\lambda}{6}n_K}\\
\Lambda &= \lambda < \min\left(n_K\mu_K, 3n_P\mu_P, 3n_M\mu_M\right) \\
C       &= 3 \cdot n_P C_P + 3 \cdot n_M C_M + n_K C_K
\end{aligned}$$

Подставляя вместо всех $n$ единицы, получаем отсутствие поэлементного
резервирования и, как и следует ожидать, исходные формулы.

\subsubsection*{Получение значений}

Экспериментально установим некоторые близкие к идеальным значения и зададим
такие ограничения, чтобы лишь небольшой набор комбинаций $n$ под них подходил.
Это будут: $P \ge 1 - 1e-9$, $C \le 250$, $u \le 43.4$.

Получаем такие комбинации:

\begin{tabular}{|c|c|c||c|c|c|c|}
\hline
$n_K$ & $n_M$ & $n_P$ & $1-P, 10^{-9}$ & $u$      & $\Lambda$ & $C$   \\ \hline
\hline
 8    &  3    &  4    & $2.17$         & $43.120$ & $0.52$    & $250$ \\ \hline
 8    &  5    &  3    & $2.09$         & $43.389$ & $0.39$    & $241$ \\ \hline
 9    &  5    &  3    & $0.51$         & $43.378$ & $0.39$    & $243$ \\ \hline
10    &  5    &  3    & $0.42$         & $43.369$ & $0.39$    & $245$ \\ \hline
11    &  5    &  3    & $0.4098$       & $43.362$ & $0.39$    & $247$ \\ \hline
12    &  5    &  3    & $0.4095$       & $43.356$ & $0.39$    & $249$ \\ \hline
\end{tabular}

Легко заметить, что все решения, кроме первого,~--- это поставить пять блоков
памяти, три блока обработки и некоторое количество коммутаторов.

Нормализуем согласно таким соображениям:

\begin{itemize}
\item Худшие случаи выбираем такие, какие использовали при фильтрации;
\item Лучший случай для времени пребывания~--- отсутствие очередей, то есть
проход по одному разу по коммутатору, обработчику и устройству памяти: $41$;
\item Лучший случай для надёжности~--- $100\%$;
\item Граничные случаи для интенсивности входных заявок~--- $0.35 и 0.6$;
\item Лучший случай для стоимости~--- $240$.
\end{itemize}

Тогда нормализованные значения таковы:

\begin{tabular}{|c|c|c||c|c|c|c|}
\hline
$n_K$ & $n_M$ & $n_P$ & $P$      & $u$     & $\Lambda$ & $C$    \\ \hline
\hline
 8    &  3    &  4    & $0.7826$ & $0.110$ & $0.69$    & $0.00$ \\ \hline
 8    &  5    &  3    & $0.7910$ & $0.005$ & $0.17$    & $0.90$ \\ \hline
 9    &  5    &  3    & $0.9490$ & $0.009$ & $0.17$    & $0.70$ \\ \hline
10    &  5    &  3    & $0.9584$ & $0.012$ & $0.17$    & $0.50$ \\ \hline
11    &  5    &  3    & $0.9590$ & $0.016$ & $0.17$    & $0.30$ \\ \hline
12    &  5    &  3    & $0.9591$ & $0.018$ & $0.17$    & $0.10$ \\ \hline
\end{tabular}

\subsubsection*{Поиск наилучшего решения}

\paragraph{Главный критерий}

Считая, что надёжность в $1.00 - 10^{-9}$ достаточно хороша, не ставим своей
целью её дальнейшее увеличение. Вместо этого обеспокаиваемся величиной
входного потока и средним временем пребывания. В таких условиях побеждает
вариант $(8, 3, 4)$.

\paragraph{Мультипликативный критерий} Посчитаем значения произведений частных
показателей:

\begin{tabular}{|c|c|c||c|}
\hline
$n_K$ & $n_M$ & $n_P$ &          \\ \hline
\hline
 8    &  3    &  4    & $0.000000$ \\ \hline
 8    &  5    &  3    & $0.000605$ \\ \hline
 9    &  5    &  3    & $0.001010$ \\ \hline
10    &  5    &  3    & $0.000978$ \\ \hline
11    &  5    &  3    & $0.000783$ \\ \hline
12    &  5    &  3    & $0.000293$ \\ \hline
\end{tabular}

Побеждает вариант $(9, 5, 3)$ как наиболее сбалансированный по цене и остальным
показателям.

\paragraph{Аддитивный критерий} Значения сумм частных показателей:

\begin{tabular}{|c|c|c||c|}
\hline
$n_K$ & $n_M$ & $n_P$ &          \\ \hline
\hline
 8    &  3    &  4    & $1.5826$ \\ \hline
 8    &  5    &  3    & $1.8660$ \\ \hline
 9    &  5    &  3    & $1.8280$ \\ \hline
10    &  5    &  3    & $1.6404$ \\ \hline
11    &  5    &  3    & $1.4450$ \\ \hline
12    &  5    &  3    & $1.2471$ \\ \hline
\end{tabular}

Обнаруживаем, что побеждает вариант $(8, 5, 3)$, низкая стоимость которого даёт
существенный прирост общему значению.

\paragraph{Метод отклонения от идеала} В силу способа нормализации по идеалу мы
обнаруживаем, что результат вычисления по аддитивному критерию совпадает с
методом отклонения от идеала. Таким образом, мы уже нашли ответ: $(8, 5, 3)$.

\paragraph{Метод последовательной уступки} Определяем значение главного критерия
и осуществляем уступку по нему. Так как значение уступки по выбранным нами
главным критериям требуется слишком большое и требуется выбрать только одно
значение, выберем другой главный критерий: стоимость. По ней оптимальный вариант
$(8, 5, 3)$. Зададим уступку в $4$ у. е. и произведём оптимизацию по времени
пребывания. Тогда имеем три варианта: $(8, 5, 3)$, $(9, 5, 3)$ и $(10, 5, 3)$.
Побеждает вариант $(10, 5, 3)$ со значением $43.369$. Произведём уступку в
$0.01$ и произведём оптимизацию по надёжности. Имеем $(9, 5, 3)$ и $(10, 5,
3)$, и $(10, 5, 3)$ снова выигрывает. Критериев, по которым можно произвести
дальнейшую оптимизацию, не осталось,~--- побеждает $(10, 5, 3)$.

\paragraph{Метод STEM} TODO: дописать.
% Мне было лень понимать, что написано в методичке. Будет настроение -- допишу.

\subsection*{Общее резервирование}

\subsection*{Анализ}

Рассмотрим формулы для системы, полученной из исходной путём полного
резервирования.

Пусть надёжность исходной системы равна $P_a$. Тогда вероятность её отказа~---
$1 - P_a$. Вероятность того, что не работает $n$ таких систем разом,~---
$(1 - P_a)^n$. Соответственно, вероятность того, что хотя бы одна работает,~---
$1 - (1 - P_a)^n$.

Далее, пусть среднее время пребывания в системе равно $u_a(\lambda)$. Тогда
вероятность попадания заявки в конкретную из $n$ систем равна $\frac 1 n$.
Соответственно, если при полном резервировании мы имеем в $n$ раз меньший поток
в каждую из копий. Так как $u = \sum_i \alpha_i u_i$, имеем $u = \sum_i^n
\frac{1}{n} u_a\left(\frac \lambda n\right) = u_a\left(\frac \lambda n\right)$.

Теперь, пусть ранее максимальный допустимый входной поток был равен $\lambda_a$,
который мы вычисляли как $\min M$. При резервировании же имеем $\lambda =
\min n M$: в силу того, что все потоки в каждой копии уменьшились
пропорционально в $n$ раз по сравнению с имеющимися в исходной системе, каждая
копия способна обрабатывать в $n$ раз больший входной поток. Так как $\min nA =
n \min A$, имеем $\Lambda = n \min M = n \lambda_a$.

Стоимость системы, состоящей из $n$ копий, равна стоимости одной копии $n$ раз:
$C = n C_a$.

Итоговые формулы:

$$\begin{aligned}
P &= 1 - (1 - P_a)^n \\
u &= \dfrac{1} {\mu_K - \frac \lambda n} +
     \dfrac{1} {\mu_P - \frac \lambda {3n}} +
     \dfrac{1} {\mu_M - \frac \lambda {3n}} \\
\Lambda &= n\lambda_a \\
C       &= n C_a
\end{aligned}$$

Мы пользуемся значениями $P_a$, $\lambda_a$ и $C_a$, поскольку они уже
расчитаны.

\subsubsection*{Получение значений}

В силу того, что стоимость выбранной системы равна $65$ у. е., а бюджет
составляет $250$ у. е., мы можем себе позволить только трёхкратное
резервирование: $65 \cdot 3 = 210 = 250 - 40$.

Таким образом, мы можем рассмотреть лишь три различных варианта:

\begin{tabular}{|c||c|c|c|c|}
\hline
$n$ & $P$       & $u$     & $\Lambda$ & $C$   \\ \hline
\hline
1   & $93.77\%$ & $50.87$ & $0.1304$  & $65$  \\ \hline
2   & $99.61\%$ & $45.37$ & $0.2608$  & $130$ \\ \hline
3   & $99.98\%$ & $43.80$ & $0.3912$  & $195$ \\ \hline
\end{tabular}

Нормализуем по таким правилам:

\begin{tabular}{|c|c|c|}
\hline
\bf Показатель   & \bf Худший случай & \bf Лучший случай \\
\hline
Надёжность       & $93.00\%$         & $100\%$ \\
Время пребывания & $60$              & $41$    \\
Интенсивность    & $0.1$             & $0.6$   \\
Стоимость        & $250$             & $50$    \\
\hline
\end{tabular}

Получаем:

\begin{tabular}{|c||c|c|c|c|}
\hline
$n$ & $P$       & $u$     & $\Lambda$ & $C$     \\ \hline
\hline
1   & $0.11000$ & $0.481$ & $0.0608$  & $0.925$ \\ \hline
2   & $0.94429$ & $0.770$ & $0.3216$  & $0.600$ \\ \hline
3   & $0.99714$ & $0.853$ & $0.5824$  & $0.275$ \\ \hline
\end{tabular}

\subsubsection*{Поиск наилучшего решения}

\paragraph{Мультипликативный критерий} Строим таблицу произведений
нормализованных частных показателей:

\begin{tabular}{|c|c|c||c|}
\hline
$n$ &          \\ \hline
1   & $0.0030$ \\ \hline
2   & $0.1403$ \\ \hline
3   & $0.1362$ \\ \hline
\end{tabular}

С небольшим отрывом выигрывает резервирование, при котором у нас есть два
экземпляра системы. Сильно проигрывает отсутствие резервирования.

\paragraph{Аддитивный критерий} Значения сумм:

\begin{tabular}{|c|c|c||c|}
\hline
$n$ &          \\ \hline
1   & $1.5768$ \\ \hline
2   & $2.6359$ \\ \hline
3   & $2.7075$ \\ \hline
\end{tabular}

По аддитивному критерию лучше резервирование с тремя экземплярами. Снова
отсутствие резервирования значительно хуже любой формы его наличия.

\subsection*{Сравнение методов резервирования}

Обнаруживается, что поэлементное резервирование куда более гибкое и позволяет в
рамках заданных ограничений добиться значительно больших результатов. Однако
необходимо учитывать, что полное резервирование существенно проще реализуемо на
практике: достаточно, не изменяя внутренней структуры прибора, добавить схожих с
ним и добиться прироста по нужным параметрам, в то время как поэлементное
резервирование требует наличия контроля над реализацией прибора.



\end{document}
