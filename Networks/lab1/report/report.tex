\documentclass[12pt, a4paper] {ncc}
\usepackage[utf8] {inputenc}
\usepackage[T2A]{fontenc}
\usepackage[english, russian] {babel}
\usepackage[usenames,dvipsnames]{xcolor}
\usepackage{listings,a4wide,longtable,amsmath,amsfonts,graphicx,tikz}
\usepackage{pgfplots}
\usepackage{indentfirst}
\usepackage{bytefield}
\usepackage{multirow}
\usepackage{tabularx}

\begin{document}
\frenchspacing
\pagestyle{empty}
\begin{center}
     Национальный исследовательский университет информационных технологий,
                              механики и оптики\\
                        Кафедра вычислительной техники\\
                          Сети ЭВМ и телекоммуникации
\end{center}
\vspace{\stretch{2}}
\begin{center}
                            Учебно-исследовательская работа №1\\
                <<Передача кодированных данных по каналу связи>>
\end{center}
\vspace{\stretch{3}}
\begin{flushright}
                                          Студентка:\\
                                                         {\it Куклина М., P3301} \\
										  Преподаватель:\\
														 {\it Шинкарук Д.Н. }
\end{flushright}
\vspace{\stretch{4}}
\begin{center}
                             Санкт-Петербург, 2017
\end{center}
\newpage


\section*{Цели работы}
    Исследование влияния свойств канала связи на качество передачи
    сигналов при различных методах физического и логического кодирования,
    используемых в цифровых сетях передачи данных.

\section*{Исходные данные}

\begin{enumerate}
        \item Фамилия студента: \textsl{КУКЛИНА М.Д.};
        \item Представление в HEX первых 4-х байт: \texttt{CA D3 CA CB};
        \item Представление в BIN первых 4-х байт: \texttt{11001010 11010011 11001010 11001011}
\end{enumerate}

\section*{Таблица результатов}

\begin{tabular}{|c|c|c|c|c|c|c|c|c|}
        \hline
        \multicolumn{3}{|c|}{Шестнадцатеричный код сообщения} & \multicolumn{6}{c|}{Метод кодирования} \\
        \hline
        \multicolumn{3}{|c|}{\texttt{\textbackslash CBCAD3CA}} & NRZ & RZ & AMI & M-II & 4B/5B & Scramb \\
        \hline
        \multirow{4}{80pt}{Полоса пропускания идеального канала связи} & \multirow{2}{*}{Гармоники}
          & min &
          8  & 8  & 10 & 40  &  8 & 10  \\
          \cline{3-9} & & max &
          24 & 24 & 30 &  56 &  40 & 31 \\
          \cline{2-9}
        & \multirow{2}{*}{Частоты, МГц} & min &
           1.3  & 1.3 & 1.6 & 6.3 & 1 & 1.6 \\
          \cline{3-9}
            & & max &
           3.8 &  3.8 & 4.7 & 8.8 & 5  & 4.8 \\
          \cline{2-9}
        \hline
        \multicolumn{3}{|c|}{Минимальная полоса идеального канала} &
          2.4 &  2.4  &  3.1 &  2.5 & 4 & 3.2 \\
        \hline
        \multicolumn{2}{|c|}{Уровень шума} & max &
          0.1 &  0.1 &  0.04 &  0.14 & 0.06 & 0.07 \\
        \hline
        \multicolumn{2}{|c|}{Уровень рассинхронизации} & max &
          0.36 &  0.84 &  0.05 & 0.3 & 0.35& 0.19 \\
        \hline
        \multicolumn{2}{|c|}{Уровень гранич. напряж.} & max &
          0.22 &  0.14 &  0.63 &  1 &  0.08 & 0.08 \\
        \hline
        \multicolumn{3}{|c|}{\% ошибок при max уровнях и мин. полосе} &
          6.79 & 7.97 & 0.75 &  0.04 &  1.25 & 1.37 \\
        \hline
        \multicolumn{2}{|c|}{Уровень шума} & avg &
        \multicolumn{6}{c|}{0.085}\\
        \hline
        \multicolumn{2}{|c|}{Уровень рассинхронизации} & avg &
        \multicolumn{6}{c|}{0.3483} \\
        \hline
        \multicolumn{2}{|c|}{Уровень гранич. напряж.} & avg &
        \multicolumn{6}{c|}{0.3583} \\
        \hline
        \multirow{4}{80pt}{Полоса пропускания реального канала связи} & \multirow{2}{*}{Гармоники} & min &
          7  &  6 &  2  & 27 & 2 & 4 \\
        \cline{3-9} & & max &
          48 & 57 & 50  & 58 & 56 & 53 \\
          \cline{2-9}
         & \multirow{2}{*}{Частоты, МГц} & min &
          1.1 & 0.9 & 0.3 & 4.2 & 0.3 & 0.6\\
        \cline{3-9} & & max &
          7.5 & 8.9 & 7.8 & 9.1 & 7.0 & 8.3 \\
        \cline{2-9}
        \hline
        \multicolumn{3}{|c|}{Требуемая полоса реального канала} &
          6.4 & 8 & 7.5 & 4.9 & 6.7 & 7.7 \\
        \hline
\end{tabular}

\section*{Вывод}

При выполнении лабораторной работы проводилось исследование влияния
свойств канала связи на качество передачи сигналов при различных методах
кодирования.
Для идеального канала лучшим методом физического кодирования был выбран M-II в силу
его высокой устойчивости к шумам  и небольшой, в сравнении с остальными
кодами, полосой канала. Из методов логического кодирования (в работе производился расчёт
логических методов над методом физического кодирования NRZ) лучшим можно назвать
метод скремблирования в силу более низкой итоговой полосы канала и низким
уровнем рассинхронизации.
Для реального канала лушим методом кодирования выбран метод M-II в силу самой
минимальной из представленных полосы канала. Логическое кодирование в данном
случае не считается целесообразным из-за высоких показателей требуемой
полосы канала, однако при необходимости выбора был бы выбран метод 4B/5B.

\end{document} 
