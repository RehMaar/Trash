\documentclass[12pt, a4paper] {ncc}
\usepackage[utf8] {inputenc}
\usepackage[T2A]{fontenc}
\usepackage[english, russian] {babel}
\usepackage[usenames,dvipsnames]{xcolor}
\usepackage{listings,a4wide,longtable,amsmath,amsfonts,graphicx,tikz}
\usepackage{pgfplots}
\usepackage{indentfirst}
\usepackage{bytefield}
\usepackage{multirow}
\usepackage{tabularx}

\begin{document}
\frenchspacing
\pagestyle{empty}
\begin{center}
     Национальный исследовательский университет информационных технологий,
                              механики и оптики\\
                        Кафедра вычислительной техники\\
                          Сети ЭВМ и телекоммуникации
\end{center}
\vspace{\stretch{2}}
\begin{center}
                            Учебно-исследовательская работа №3\\
                        <<Основы администрирования маршрутизируюмых компьютерных сетей>
								Варианты 38
\end{center}
\vspace{\stretch{3}}
\begin{flushright}
                                          Студентка:\\
                                                         {\it Куклина М., P3301} \\
										  Преподаватель:\\
														 {\it Шинкарук Д.Н. }
\end{flushright}
\vspace{\stretch{4}}
\begin{center}
                             Санкт-Петербург, 2017
\end{center}
\newpage


\section*{Цели работы}
	Изучение основных методов настройки маршрутизируемых компьютерных сетей на примере сети,
	состоящей из компьютеров под управлением ОС Linux.

\section*{Общая часть}
	\subsection*{Настройка сети}
    	Для реализации задания была создана сеть из двух хостов, состоящих в одной
    	внутренней сети. Для обеспечения сетевой доступности были выполнены следющие команды.
\begin{verbatim}
        # ip a addr 5.7.1.Y/24 dev enp0s3
\end{verbatim}

    	Адреса хостов отличаются значением Y.

	\subsection*{nc}

		Запуск на хосте Б -- сервере.

\begin{verbatim}
        # nc -l -p 1337
\end{verbatim}

		Запуск на хосте А -- клиента.

\begin{verbatim}
        # nc 5.7.1.2 1337
\end{verbatim}
	
	\subsection*{iptables}

	\begin{enumerate}
		\item Запретить передачу пакетов, которые отправлены на заданный TCP-порт.
            \begin{verbatim}
            -A OUTPUT -o enp0s3 -p tcp --dport 1337 -j REJECT
            \end{verbatim}
		\item Запретить приём пакетов, которые отправлены на заданный UDP-порт.
            \begin{verbatim}
            -A INPUT -i enp0s3 -p udp --dport 1337 -j REJECT
            \end{verbatim}
		\item Запретить передачу пакетов, которые отправлены с хоста B.
            \begin{verbatim}
            -A OUTPUT -o enp0s3 -s 5.7.1.2 -j REJECT
            \end{verbatim}
		\item Запретить приём пакетов, которые отправлены на хост A.
            \begin{verbatim}
            -A INPUT -i enp0s3 -d 5.7.1.3 -j REJECT
            \end{verbatim}
		\item Запретить приём ICMP пакетов, размер которых больше 1000 байт и TTL меньше 10.
            \begin{verbatim}
            -A INPUT -i enp0s3 -p icmp -m length ! --length 0:1000 -m ttl  --ttl-lt 10 -j REJECT
            \end{verbatim}
	\end{enumerate}
	
\section{V1: вариант 3}

\section{V2: вариант 8}


\end{document} 
