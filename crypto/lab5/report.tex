\documentclass[12pt, a4paper] {ncc}
\usepackage[utf8] {inputenc}
\usepackage[T2A]{fontenc}
\usepackage[english, russian] {babel}
\usepackage[usenames,dvipsnames]{xcolor}
\usepackage{listings,a4wide,longtable,amsmath,amsfonts,graphicx}
\usepackage{indentfirst}
\usepackage{bytefield}
\usepackage{multirow}
\usepackage{float}
\usepackage{caption}
\usepackage{subcaption}
\captionsetup{compatibility=false}
\usepackage{tabularx}

\usepackage{tikz}
\usetikzlibrary{automata}


\usepackage[left=2cm,right=2cm,top=2cm,bottom=2cm,bindingoffset=0cm]{geometry}

\begin{document}
\setcounter{figure}{0}
\frenchspacing
\pagestyle{empty}
\begin{center}
                            Университет ИТМО    \\
                        Кафедра вычислительной техники

\vspace{\stretch{2}}
						Лабораторная работа № 5 \\
				по дисциплине: \\
				<<Методы и средства защиты компьютерной информации>>\\
Вариант: 5
\end{center}
\vspace{\stretch{2}}
\begin{center}

\end{center}
\vspace{\stretch{3}}
\begin{flushright}
                                    Студент:\\
                                    {\it Куклина Мария, гр. P3401}\\
                                    Преподаватель: \\
                                    {\it Ожиганов А.А}
\end{flushright}
\vspace{\stretch{4}}
\begin{center}
                             Санкт-Петербург, 2018
\end{center}
\newpage

\section{Цель работы}

Расшифровать криптограмму, зашифрованную методом перестановок, получить ключ
шифрования.\\
Расшифровать криптограмму, зашифрованную методом перестановок
по путям Гамильтона, получить ключ шифрования. \\
 Расшифровать криптограмму, зашифрованную методом табличной перестановки, пользуясь для этой цели
таблицей частот диграмм русского языка, получить ключ шифрования.

\section{Простая перестановка}

\textbf{Криптограмма}: \\
ОТ ЭТАРЗ ЧКАП ЯЕНИПЕРОДКЯСТЛЯЕК М ТЕДЮЛАК КНУИ ФРИНЦИО В УЮТ МЕНАШР ЕМИРНЗОАСПЕИНАВАЗАР ОБВТООВ  ТЕЕЧАТКАЗА ТСОИВНБЕРЬ Т ЮУУЮЩОРТБЫСКАЕЙ РАТ ЦИИ КАК КВТСДЕЙВОСИЯ ТЮАЕРШСЫБСЯ  И ТРОЗОСБЕСЬЛЕНАТ ОТНО С ТЭТО БОПОСИЦКФУНВОРОНИВ ЯАНИД НАЖЕЖЫВЛЯ ЯИНИВАРВ  ВОНБЕАЖДРКООМ ИИНУЖЕ

\textbf{Открытый текст}: \\
ЭТА ТОЧКА ЗРЕНИЯ ПОДКРЕПЛЯЕТСЯ ТЕМ КАК ЛЮДИ ФУНКЦИОНИРУЮТ В НАШЕМ МИРЕ РАСПОЗНАВАНИЕ ОБРАЗОВ ОТВЕЧАЕТ ЗА АКТИВНОСТЬ ТРЕБУЮЩУЮ БЫСТРОЙ РЕАКЦИИ ТАК КАК ДЕЙСТВИЯ СОВЕРШАЮТСЯ БЫСТРО И БЕССОЗНАТЕЛЬНО ТО ЭТОТ СПОСОБ ФУНКЦИОНИРОВАНИЯ ВАЖЕН ДЛЯ ВЫЖИВАНИЯ ВО ВРАЖДЕБНОМ ОКРУЖЕНИИ

\textbf{Ключ}:
654123

\subsection{Краткий протокол криптоанализа}

Случайно выберем длину ключа в 6 символов. При рассмотрении текста замечаем, что в начале предложения символы ОТ ЭТАРЗ ЧКА напоминает
слова ЭТА и ТОЧКА. Проверяем слово ЭТА, получаем бессмыслицу. Проверяем слово ТОЧКА, получаем вразумительный текст. 

\section{Перестановка с использованием путей Гамильтона}

\textbf{Криптограмма}: \\
Т  ИОЭЛСНСОИН  ЬЕМТ  ИОЭТЯНССИЕИМ  НОМЭ  ТЕЕЕНЩ ГАК ДОБДН УИРС ЬИПС ИНЯТОТ ИВПТЯРОСИЕИВ  СЕДВ  СПОВОЛОС ИТЯТА  ИМВСИРПНИ ССТВЯТ  ЕОЧЕ ИВЛДОС Я ВТМЕНА СОВ Т ЕООР НА  ТСНМРТОИ  Т АСВН РОЕОН ОВАЛТВДИ ЕЕ СОДЛН ОВЕЛЕЕ ОРБИЬ ГТБАНЯС  ОН АЛВВЗ ЕЕЗИЧАД  ВКЛЕЕОПИИ Т ИЯАЦ  ИСЬВ  ИСЫН  Т ЕМЬРН ОВАЛЛДОВЕСЛЕВ НО

\textbf{Открытый текст}: \\
И ЭТО СНИЛОСЬ МНЕ И ЭТО СНИТСЯ МНЕ И ЭТО МНЕ ЕЩЕ КОГДА НИБУДЬ ПРИСНИТСЯ И ПОВТОРИТСЯ ВСЕ И ВСЕ ДОВОПЛОТИТСЯ И ВАМ ПРИСНИТСЯ ВСЕ ЧТО ВИДЕЛ Я ВО СНЕ ТАМ В СТОРОНЕ ОТ НАС ОТ МИРА В СТОРОНЕ ВОЛНА ИДЕТ ВОСЛЕД ВОЛНЕ О БЕРЕГ БИТЬСЯ А НА ВОЛНЕ ЗВЕЗДА И ЧЕЛОВЕК И ПТИЦА И ЯВЬ И СНЫ И СМЕРТЬ ВОЛНА ВОСЛЕД ВОЛНЕ

\textbf{Ключ}:  462153

\subsection{Граф}

	\begin{tikzpicture}[node distance=3cm]
        \tikzstyle{st} = [circle, fill=white, thick, draw=black, minimum size=2em];
		\node [st] (a) {$1$};
		\node [st, right of = a] (b) {$2$};
		\node [st, below of = a] (c) {$3$};
		\node [st, below of = b] (d) {$4$};
		\node [st, below right of = c] (e) {$5$};
		\node [st, left of = e] (f) {$6$};


		\path (a) edge node{$ $} (b)
			  (a) edge node{$ $} (c)
			  (a) edge node{$ $} (d)
			  (a) edge node{$ $} (f)
			  (b) edge node{$ $} (c)
			  (b) edge node{$ $} (d)
			  (c) edge node{$ $} (f)
			  (d) edge node{$ $} (e)
			  (d) edge node{$ $} (f)
			  (e) edge node{$ $} (f);

	\end{tikzpicture}

	\subsection{Матрица смежности графа}
	\begin{tabular} {|c|c|c|c|c|c|c|}
		\hline
		Вершины & 1 & 2 & 3 & 4 & 5 & 6 \\ \hline
		1	    & 0 & 1 & 1 & 1 & 0 & 1 \\ \hline
		2	    & 1 & 0 & 1 & 1 & 0 & 0 \\ \hline
		3	    & 1 & 1 & 0 & 0 & 0 & 1 \\ \hline
		4	    & 1 & 1 & 0 & 0 & 1 & 1 \\ \hline
		5	    & 0 & 0 & 0 & 1 & 0 & 1 \\ \hline
		6	    & 1 & 0 & 1 & 1 & 1 & 0 \\ \hline
	\end{tabular}

\subsection{Краткий протокол криптоанализа}
	Предполагаем, что длина ключа равна 6. Берём произвольный гамильтонов граф и проверяем его пути. На пути 462351 обнаружилось,
	что первые слова криптограммы Э ИТО похожи на И ЭТО, однако данная подстановка
	дала ключ 351462, но первая часть сообщение обрела вид ' ЭТИ ОНИЛССО МНЬ Е',
	что напоминает 'СНИЛОСЬ МНЕ'. Считаем, что вероятное слово это СНИЛОСЬ.
	Получаем осмысленный текст с ключом 351462.


\section{Табличная перестановка}
	
\textbf{Криптограмма}:
ЯСОДЛПЯНЕ ВЕР ВТУМОД  ИН ЛУТПНАУБВЯ ХМАО ЕИЛВТСОМК Д ЬРОЙДСН АЫЫ

\textbf{Открытый текст}:
ПОДНЯЛСЯ ВЕТЕР В ОДНУ МИНУТУ ПЛАМЯ ОБХВАТИЛО ВЕСЬ ДОМ КРАСНЫЙ ДЫ

\textbf{Ключ}:
63481527

\subsection{Таблица вероятностей}
	\begin{tabular}{|c|c|c|c|c|c|c|c|c|}
		\hline
		i\j & 1 & 2 & 3 & 4 & 5 & 6 & 7 & 8 \\ \hline
		1   & - & 39 & 38 & 42 & 58 & 41 & 37 & 8 \\ \hline
		2   & 36 & - & 51 & 41 & 47 & 43 & 61 & 54 \\ \hline
		3   & 42 & 55 & - & 60 & 47 & 51 & 40 & 40 \\ \hline
		4   & 41 & 46 & 54 & - & 46 & 50 & 50 & 61 \\ \hline
		5   & 49 & 51 & 44 & 54 & - & 25 & 49 & 52 \\ \hline
		6   & 49 & 40 & 61 & 42 & 38 & - & 45 & 45 \\ \hline
		7   & 49 & 52 & 38 & 50 & 52 & 47 & - & 41 \\ \hline
		8   & 60 & 45 & 49 & 54 & 51 & 39 & 44 & - \\ \hline
	\end{tabular}

\subsection{Протокол криптоанализа}
	Из таблицы последовтельности строк: 1 - 5, 2 - 7, 3 - 4, 4 - 8
	5 - 4, 6 - 3, 7 - 2 и - 5 с равной вероятностью, 8 - 1.
    Попробуем переставить 8 за 4, 1 за 8 и 5 за 1. Получим в первом
	столбце ПСОДНЯЛЯ, что является анаграммой слова ПОДНЯЛСЯ. После
	того, как бы расставим строки сообразно слову, получим ключ 63481527.
	
\end{document}
