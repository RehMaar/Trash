\documentclass[12pt, a4paper] {ncc}
\usepackage[utf8] {inputenc}
\usepackage[T2A]{fontenc}
\usepackage[english, russian] {babel}
\usepackage[usenames,dvipsnames]{xcolor}
\usepackage{listings,a4wide,longtable,amsmath,amsfonts,graphicx}
\usepackage{indentfirst}
\usepackage{bytefield}
\usepackage{multirow}
\usepackage{float}
\usepackage{caption}
\usepackage{subcaption}
\captionsetup{compatibility=false}
\usepackage{tabularx}

\usepackage[left=2cm,right=2cm,top=2cm,bottom=2cm,bindingoffset=0cm]{geometry}

\begin{document}
\setcounter{figure}{0}
\frenchspacing
\pagestyle{empty}
\begin{center}
                            Университет ИТМО    \\
                        Кафедра вычислительной техники

\vspace{\stretch{2}}
						Лабораторная работа № 3 \\
				по дисциплине: \\
				<<Методы и средства защиты компьютерной информации>>\\
Вариант: 5
\end{center}
\vspace{\stretch{2}}
\begin{center}

\end{center}
\vspace{\stretch{3}}
\begin{flushright}
                                    Студент:\\
                                    {\it Куклина Мария, гр. P3401}\\
                                    Преподаватель: \\
                                    {\it Ожиганов А.А}
\end{flushright}
\vspace{\stretch{4}}
\begin{center}
                             Санкт-Петербург, 2018
\end{center}
\newpage

\section{Цель работы}

Дешифровать криптограмму, зашифрованную многопетлевым шифром. Определить период шифра предлагаемой
криптограммы. Получить составной ключ, вычислить первичные ключи.


\section{Расшифрованный исходный текст}

\textbf{Открытый текст}:
СВИЯЖСКИЙ БЫЛ ПРЕДВОДИТЕЛЕМ В СВОЕМ УЕЗДЕ ОН БЫЛ ПЯТЬЮ ГОДАМИ СТАРШЕ ЛЕВИНА И ДАВНО ЖЕНАТ В ДОМЕ ЕГО ЖИЛА МОЛОДАЯ ЕГО СВОЯЧЕНИЦА ОЧЕНЬ СИМПАТИЧНАЯ ЛЕВИНУ ДЕВУШКА И ЛЕВИН ЗНАЛ ЧТО СВИЯЖСКИЙ И ЕГО ЖЕНА ОЧЕНЬ ЖЕЛАЛИ ВЫДАТЬ ЗА НЕГО ЭТУ ДЕВУШКУ ОН ЗНАЛ ЭТО НЕСОМНЕННО КАК ЗНАЮТ ЭТО ВСЕГДА МОЛОДЫЕ ЛЮДИ ТАК НАЗЫВАЕМЫЕ ЖЕНИХИ ХОТЯ НИКОГДА НИКОМУ НЕ РЕШИЛСЯ БЫ СКАЗАТЬ ЭТОГО И ЗНАЛ ТОЖЕ И ТО ЧТО НЕСМОТРЯ НА ТО ЧТО ОН ХОТЕЛ ЖЕНИТЬСЯ НЕСМОТРЯ НА ТО ЧТО ПО ВСЕМ ДАННЫМ ЭТА ВЕСЬМА ПРИВЛЕКАТЕЛЬНАЯ ДЕВУШКА ДОЛЖНА БЫЛА БЫТЬ ПРЕКРАСНОЮ ЖЕНОЙ ОН ТАК ЖЕ МАЛО МОГ ЖЕНИТЬСЯ НА НЕЙ ДАЖЕ ЕСЛИ Б ОН И НЕ БЫЛ ВЛЮБЛЕН В КИТИ ЩЕРБАЦКУЮ КАК УЛЕТЕТЬ НА НЕБО И ЭТО ЗНАНИЕ ОТРАВЛЯЛО ЕМУ ТО УДОВОЛЬСТВИЕ КОТОРОЕ ОН НАДЕЯЛСЯ ИМЕТЬ ОТ ПОЕЗДКИ К СВИЯЖСКОМУ ПОЛУЧИВ ПИСЬМО СВИЯЖСКОГО С ПРИГЛАШЕНИЕМ НА ОХОТУ ЛЕВИН ТОТЧАС ЖЕ ПОДУМАЛ ОБ ЭТОМ НО НЕСМОТРЯ НА ЭТО РЕШИЛ ЧТО ТАКИЕ ВИДЫ НА НЕГО СВИЯЖСКОГО ЕСТЬ ТОЛЬКО ЕГО НИ НА ЧЕМ НЕ ОСНОВАННОЕ ПРЕДПОЛОЖЕНИЕ И ПОТОМУ ОН ВСЕ ТАКИ ПОЕДЕТ КРОМЕ ТОГО В ГЛУБИНЕ ДУШИ ЕМУ ХОТЕЛОСЬ ИСПЫТАТЬ СЕБЯ ПРИМЕРИТЬСЯ ОПЯТЬ К ЭТОЙ ДЕВУШКЕ ДОМАШНЯЯ ЖЕ ЖИЗНЬ СВИЯЖСКИХ БЫЛА В ВЫСШЕЙ СТЕПЕНИ ПРИЯТНА И САМ СВИЯЖСКИЙ САМЫЙ ЛУЧШИЙ ТИП ЗЕМСКОГО ДЕЯТЕЛЯ КАКОЙ ТОЛЬКО ЗНАЛ ЛЕВИН БЫЛ ДЛЯ ЛЕВИНА ВСЕГДА ЧРЕЗВЫЧАЙНО ИНТЕРЕСЕН СВИЯЖСКИЙ БЫЛ ОДИН ИЗ ТЕХ ВСЕГДА УДИВИТЕЛЬНЫХ ДЛЯ ЛЕВИНА ЛЮДЕЙ РАССУЖДЕНИЕ КОТОРЫХ ОЧЕНЬ ПОСЛЕДОВАТЕЛЬНОЕ ХОТЯ И НИКОГДА НЕ САМОСТОЯТЕЛЬНОЕ ИДЕТ САМО ПО СЕБЕ А ЖИЗНЬ ЧРЕЗВЫЧАЙНО ОПРЕДЕЛЕННАЯ И ТВЕРДАЯ В СВОЕМ НАПРАВЛЕНИИ ИДЕТ САМА ПО СЕБЕ СОВЕРШЕННО НЕЗАВИСИМО И ПОЧТИ ВСЕГДА ВРАЗРЕЗ С РАССУЖДЕНИЕМ СВИЯЖСКИЙ БЫЛ ЧЕЛОВЕК ЧРЕЗВЫЧАЙНО ЛИБЕРАЛЬНЫЙ ОН ПРЕЗИРАЛ ДВОРЯНСТВО И СЧИТАЛ БОЛЬШИНСТВО ДВОРЯН ТАЙНЫМИ ОТ РОБОСТИ ТОЛЬКО НЕ ВЫРАЖАВШИМИСЯ КРЕПОСТНИКАМИ ОН СЧИТАЛ РОССИЮ ПОГИБШЕЮ СТРАНОЙ ВРОДЕ ТУРЦИИ И ПРАВИТЕЛЬСТВО РОССИИ СТОЛЬ ДУРНЫМ ЧТО НИКОГДА НЕ ПОЗВОЛЯЛ СЕБЕ ДАЖЕ СЕРЬЕЗНО КРИТИКОВАТЬ ДЕЙСТВИЯ ПРАВИТЕЛЬСТВА И ВМЕСТЕ С ТЕМ СЛУЖИЛ И БЫЛ ОБРАЗЦОВЫМ ДВОРЯНСКИМ ПРЕДВОДИТЕЛЕМ И В ДОРОГУ ВСЕГДА НАДЕВАЛ С КОКАРДОЙ И С КРАСНЫМ ОКОЛЫШЕМ ФУРАЖКУ ОН ПОЛАГАЛ ЧТО ЖИЗНЬ ЧЕЛОВЕЧЕСКАЯ ВОЗМОЖНА ТОЛЬКО ЗА ГРАНИЦЕЙ КУДА ОН И УЕЗЖАЛ ЖИТЬ ПРИ ПЕРВОЙ ВОЗМОЖНОСТИ А ВМЕСТЕ С ТЕМ ВЕЛ В РОССИИ ОЧЕНЬ СЛОЖНОЕ И УСОВЕРШЕНСТВОВАННОЕ ХОЗЯЙСТВО И С ЧРЕЗВЫЧАЙНЫМ ИНТЕРЕСОМ СЛЕДИЛ ЗА ВСЕМ И ЗНАЛ ВСЕ ЧТО ДЕЛАЛОСЬ В РОССИИ ОН СЧИТАЛ РУССКОГО МУЖИКА СТОЯЩИМ ПО РАЗВИТИЮ НА ПЕРЕХОДНОЙ СТУПЕНИ ОТ ОБЕЗЬЯНЫ К ЧЕЛОВЕКУ А ВМЕСТЕ С ТЕМ НА ЗЕМСКИХ ВЫБОРАХ ОХОТНЕЕ ВСЕХ ПОЖИМАЛ РУКУ МУЖИКАМ И ВЫСЛУШИВАЛ ИХ МНЕНИЯ ОН НЕ ВЕРИЛ НИ В ЧОХ НИ В СМЕРТЬ НО БЫЛ ОЧЕНЬ ОЗАБОЧЕН ВОПРОСОМ УЛУЧШЕНИЯ БЫТА ДУХОВЕНСТВА И СОКРАЩЕНИЯ ПРИХОДОВ ПРИЧЕМ ОСОБЕННО ХЛОПОТАЛ ЧТОБЫ ЦЕРКОВЬ ОСТАЛАСЬ В ЕГО СЕЛЕ В ЖЕНСКОМ ВОПРОСЕ ОН БЫЛ НА СТОРОНЕ КРАЙНИХ СТОРОННИКОВ ПОЛНОЙ СВОБОДЫ ЖЕНЩИН И В ОСОБЕННОСТИ ИХ ПРАВА НА ТРУД НО ЖИЛ С ЖЕНОЮ ТАК ЧТО ВСЕ ЛЮБОВАЛИСЬ ИХ ДРУЖНОЮ БЕЗДЕТНОЮ СЕМЕЙНОЮ ЖИЗНЬЮ И УСТРОИЛ ЖИЗНЬ СВОЕЙ ЖЕНЫ ТАК ЧТО ОНА НИЧЕГО НЕ ДЕЛАЛА И НЕ МОГЛА ДЕЛАТЬ КРОМЕ ОБЩЕЙ С МУЖЕМ ЗАБОТЫ КАК ПОЛУЧШЕ И ПОВЕСЕЛЕЕ ПРОВЕСТИ ВРЕМЯ ЕСЛИ БЫ ЛЕВИН НЕ ИМЕЛ СВОЙСТВА ОБЪЯСНЯТЬ СЕБЕ ЛЮДЕЙ С САМОЙ ХОРОШЕЙ СТОРОНЫ ХАРАКТЕР СВИЯЖСКОГО НЕ ПРЕДСТАВЛЯЛ БЫ ДЛЯ НЕГО НИКАКОГО ЗАТРУДНЕНИЯ И ВОПРОСА ОН БЫ СКАЗАЛ СЕБЕ ДУРАК ИЛИ ДРЯНЬ И ВСЕ БЫ БЫЛО ЯСНО НО ОН НЕ МОГ СКАЗАТЬ ДУРАК ПОТОМУ ЧТО СВИЯЖСКИЙ БЫЛ НЕСОМНЕННО НЕ ТОЛЬКО ОЧЕНЬ УМНЫЙ НО ОЧЕНЬ ОБРАЗОВАННЫЙ И НЕОБЫКНОВЕННО ПРОСТО НОСЯЩИЙ СВОЕ ОБРАЗОВАНИЕ ЧЕЛОВЕК НЕ БЫЛО ПРЕДМЕТА КОТОРОГО БЫ ОН НЕ ЗНАЛ НО ОН ПОКАЗЫВАЛ СВОЕ ЗНАНИЕ ТОЛЬКО КОГДА БЫВАЛ ВЫНУЖДАЕМ К ЭТОМУ ЕЩЕ МЕНЬШЕ МОГ ЛЕВИН СКАЗАТЬ ЧТО ОН БЫЛ ДРЯНЬ ПОТОМУ ЧТО СВИЯЖСКИЙ БЫЛ НЕСОМНЕННО ЧЕСТНЫЙ ДОБРЫЙ УМНЫЙ ЧЕЛОВЕК КОТОРЫЙ ВЕСЕЛО ОЖИВЛЕННО ПОСТОЯННО ДЕЛАЛ ДЕЛО ВЫСОКО ЦЕНИМОЕ ВСЕМИ ЕГО ОКРУЖАЮЩИМИ И УЖЕ НАВЕРНОЕ НИКОГДА СОЗНАТЕЛЬНО НЕ ДЕЛАЛ И НЕ МОГ СДЕЛАТЬ НИЧЕГО ДУРНОГО ЛЕВИН СТАРАЛСЯ ПОНЯТЬ И НЕ ПОНИМАЛ И ВСЕГДА КАК НА ЖИВУЮ ЗАГАДКУ СМОТРЕЛ НА НЕГО И НА ЕГО ЖИЗНЬ ОНИ БЫЛИ ДРУЖНЫ С ЛЕВИНЫМ И ПОЭТОМУ ЛЕВИН ПОЗВОЛЯЛ СЕБЕ ДОПЫТЫВАТЬ СВИЯЖСКОГО ДОБИРАТЬСЯ ДО САМОЙ ОСНОВЫ ЕГО ВЗГЛЯДА НА ЖИЗНЬ НО ВСЕГДА ЭТО БЫЛО ТЩЕТНО КАЖДЫЙ РАЗ КАК ЛЕВИН ПЫТАЛСЯ ПРОНИКНУТЬ ДАЛЬШЕ ОТКРЫТЫХ ДЛЯ ВСЕХ ДВЕРЕЙ ПРИЕМНЫХ КОМНАТ УМА СВИЯЖСКОГО ОН ЗАМЕЧАЛ ЧТО СВИЯЖСКИЙ СЛЕГКА СМУЩАЛСЯ ЧУТЬ ЗАМЕТНЫЙ ИСПУГ ВЫРАЖАЛСЯ В ЕГО ВЗГЛЯДЕ КАК БУДТО ОН БОЯЛСЯ ЧТО ЛЕВИН ПОЙМЕТ ЕГО И ОН ДАВАЛ ДОБРОДУШНЫЙ И ВЕСЕЛЫЙ ОТПОР ТЕПЕРЬ ПОСЛЕ СВОЕГО РАЗОЧАРОВАНИЯ В ХОЗЯЙСТВЕ ЛЕВИНУ ОСОБЕННО ПРИЯТНО БЫЛО ПОБЫВАТЬ У СВИЯЖСКОГО НЕ ГОВОРЯ О ТОМ ЧТО НА НЕГО ВЕСЕЛО ДЕЙСТВОВАЛ ВИД ЭТИХ СЧАСТЛИВЫХ ДОВОЛЬНЫХ СОБОЮ И ВСЕМИ ГОЛУБКОВ ИХ БЛАГОУСТРОЕННОГО ГНЕЗДА ЕМУ ХОТЕЛОСЬ ТЕПЕРЬ ЧУВСТВУЯ СЕБЯ СТОЛЬ НЕДОВОЛЬНЫМ СВОЕЮ ЖИЗНЬЮ ДОБРАТЬСЯ В СВИЯЖСКОМ ДО ТОГО СЕКРЕТА КОТОРЫЙ ДАВАЛ ЕМУ ТАКУЮ ЯСНОСТЬ ОПРЕДЕЛЕННОСТЬ И ВЕСЕЛОСТЬ В ЖИЗНИ

\textbf{Составной ключ}:
МЬЩАЬУЙХОЪБЯЬЬЦПЭКСЭЭЩШПЛТ ЕЛЩРОЫТЖ

\section{Краткий протокол криптоанализа}
	По значению ИС (0.0173) понимаем, что следует использовать метод Казиски. По этому методу выбираем приод 35, т.к. ему
	соответствует наибольший вес.

	Далее начинаем заменять в соответствии со статистикой наиболее вероятный символ на пробел. Далее исправляем получившийся текст
	в соответствии с грамматикой русского языка. 

\section{Первичные ключи}

	Предположим, что у нас два слова длиной 5 и 7. Составим уравнения.\\
	$K_1 = K_{1,1} + K_{2,1} = \text{М} (12)$ \\
	$K_2 = K_{1,2} + K_{2,2} = \text{Ь} (28)$ \\
	$K_3 = K_{1,3} + K_{2,3} = \text{Щ} (25)$ \\
	$K_4 = K_{1,4} + K_{2,4} = \text{А} (0)$ \\
	$K_5 = K_{1,5} + K_{2,5} = \text{Ь} (28)$ \\
	$K_6 = K_{1,1} + K_{2,6} = \text{У} (19)$ \\
	$K_7 = K_{1,2} + K_{2,7} = \text{Й} (9)$ \\
	$K_8 = K_{1,3} + K_{2,1} = \text{Х} (21)$ \\
	$K_9 = K_{1,4} + K_{2,2} = \text{О} (14)$ \\
	$K_{10} = K_{1,5} + K_{2,3} = \text{Ъ} (26) $\\
	$K_{11} = K_{1,1} + K_{2,4} = \text{Б} (1) $\\
	$K_{12} = K_{1,2} + K_{2,5} = \text{Я} (31) $\\
	$K_{13} = K_{1,3} + K_{2,6} = \text{Ь} (28) $\\
	$K_{14} = K_{1,4} + K_{2,7} = \text{Ь} (28) $\\
	$K_{15} = K_{1,5} + K_{2,1} = \text{Ц} (22) $\\
	$K_{16} = K_{1,1} + K_{2,2} = \text{П} (15) $\\
	$K_{17} = K_{1,2} + K_{2,3} = \text{Э} (29) $\\
	$K_{18} = K_{1,3} + K_{2,4} = \text{К} (10) $\\
	$K_{19} = K_{1,4} + K_{2,5} = \text{С} (17) $\\
	$K_{20} = K_{1,5} + K_{2,6} = \text{Э} (29) $\\
	$K_{21} = K_{1,1} + K_{2,7} = \text{Э} (29) $\\
	$K_{22} = K_{1,2} + K_{2,1} = \text{Щ} (25) $\\
	$K_{23} = K_{1,3} + K_{2,2} = \text{Ш} (24) $\\
	$K_{24} = K_{1,4} + K_{2,3} = \text{П} (15) $\\
	$K_{25} = K_{1,5} + K_{2,4} = \text{Л} (11) $\\
	$K_{26} = K_{1,1} + K_{2,5} = \text{Т} (18) $\\
	$K_{27} = K_{1,2} + K_{2,6} = \text{' '} (32)$\\
	$K_{28} = K_{1,3} + K_{2,7} = \text{Е} (5) $\\
	$K_{29} = K_{1,4} + K_{2,1} = \text{Л} (11) $\\
	$K_{30} = K_{1,5} + K_{2,2} = \text{Щ} (25) $\\
	$K_{31} = K_{1,1} + K_{2,3} = \text{Р} (16) $\\
	$K_{32} = K_{1,2} + K_{2,4} = \text{О} (14) $\\
	$K_{33} = K_{1,3} + K_{2,5} = \text{Ы} (27) $\\
	$K_{34} = K_{1,4} + K_{2,6} = \text{Т} (18) $\\
	$K_{35} = K_{1,5} + K_{2,7} = \text{Ж} (7) $\\


	Замечаем уравнение $K_{1,4} + K_{2,4} = A (0)$. Из него делаем вывод, что $K_{1,4} = K_{2,4} = A (0)$.
	Возьмём все уравнения, где $K_{1,4}$ -- одно из слагаемых. Тогда просто будет вывести слово: ЛОПАСТЬ.
    С его помощью находим, что второе слово: БОКАЛ.  \\

	Первое слово: \textbf{ЛОПАСТЬ}.
	Второе слово: \textbf{БОКАЛ}.

\end{document}
