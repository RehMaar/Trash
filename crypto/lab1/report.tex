\documentclass[12pt, a4paper] {ncc}
\usepackage[utf8] {inputenc}
\usepackage[T2A]{fontenc}
\usepackage[english, russian] {babel}
\usepackage[usenames,dvipsnames]{xcolor}
\usepackage{listings,a4wide,longtable,amsmath,amsfonts,graphicx}
\usepackage{indentfirst}
\usepackage{bytefield}
\usepackage{multirow}
\usepackage{float}
\usepackage{caption}
\usepackage{subcaption}
\captionsetup{compatibility=false}
\usepackage{tabularx}

\usepackage[left=2cm,right=2cm,top=2cm,bottom=2cm,bindingoffset=0cm]{geometry}

\begin{document}
\setcounter{figure}{0}
\frenchspacing
\pagestyle{empty}
\begin{center}
                            Университет ИТМО    \\
                        Кафедра вычислительной техники

\vspace{\stretch{2}}
						Лабораторная работа № 1 \\
				по дисциплине: \\
				<<Методы и средства защиты компьютерной информации>>\\
Вариант: 5
\end{center}
\vspace{\stretch{2}}
\begin{center}

\end{center}
\vspace{\stretch{3}}
\begin{flushright}
                                    Студент:\\
                                    {\it Куклина Мария, гр. P3401}\\
                                    Преподаватель: \\
                                    {\it Ожиганов А.А}
\end{flushright}
\vspace{\stretch{4}}
\begin{center}
                             Санкт-Петербург, 2018
\end{center}
\newpage

\section{Цель работы}

Используя частотный анализ, дешифровать криптограмму, зашифрованную методом
моноалфавитных подстановок.

\section{Расшифрованный исходный текст}

    \texttt{Я ПОШЕЛ ВПРАВО ЧЕРЕЗ КУСТЫ МЕЖДУ ТЕМ НОЧЬ ПРИБЛИЖАЛАСЬ И
			РОСЛА КАК ГРОЗОВАЯ ТУЧА КАЗАЛОСЬ ВМЕСТЕ С ВЕЧЕРНИМИ ПАРАМИ
			ОТОВСЮДУ ПОДНИМАЛАСЬ И ДАЖЕ С ВЫШИНЫ ЛИЛАСЬ ТЕМНОТА МНЕ 
			ПОПАЛАСЬ КАКАЯ ТО НЕТОРНАЯ ЗАРОСШАЯ ДОРОЖКА Я ОТПРАВИЛСЯ 
			ПО НЕЙ ВНИМАТЕЛЬНО ПОГЛЯДЫВАЯ ВПЕРЕД ВСЕ КРУГОМ БЫСТРО 
			ЧЕРНЕЛО И УТИХАЛО ОДНИ ПЕРЕПЕЛА ИЗРЕДКА КРИЧАЛИ}

\section{Ключ}
	\begin{tabular}{|c|c|c|c|c|c|c|c|c|c|c|c|c|c|c|c|}
		\hline
		Нормативный алфавит & А & Б & В & Г& Д& Е& Ж& З& И& К& Л& М& Н& О& П \\ \hline
		Алфавит шифрования  & П & Р & У & Ь& Ю& Ц& Ъ& Л& А& Н& К& Ч& Е& С& Ы \\ \hline \hline
		Нормативный алфавит & Р 	 & С& Т& У& Х& Ч& Ш& Ы& Ь& Я & & & & &\\ \hline
		Алфавит шифрования  & пробел & З& О& Т& Б& Ж& Ф& Ш& Щ& Й & & & & & \\ \hline
		
	\end{tabular}


\section{Краткий протокол криптоанализа}

	\begin{enumerate}
		\item Руководствуясь таблицей статистики, заменяем Я на пробел.
		\item Обнаруживаем, что есть два слова: НПН и НПНПЙ, -- под которые
			  подходят слова КАК и КАКАЯ соответственно.
		\item Предполагаем, что слово УЗЦ обозначает ВСЕ. 
		\item Видимо слово В-ЕС-Е (УЧЦЗОЦ), которое очень похоже на ВМЕСТЕ; У - М, Т - О.
		\item Видим словов ТЕМ--ТА (ОЦЧЕСОП), которое очень похоже на ТЕМНОТА; Е - Н, С - О.
		\item Заменяем пропуски в слове ВН-МАТЕ--НО на ВНИМАТЕЛЬНО; А - И, К - Л, Щ - Ь.
		\item В--АВО заменяет на ВПРАВО; Ы - П, Р - ' '.
		\item ПО-ЕЛ на ПОШЕЛ; Ф - Ш.
		\item НО-Ь на НОЧЬ; Ж - Ч.
		\item Т-ЧА на ТУЧА; Т - У.
		\item КА-АЛОСЬ на КАЗАЛОСЬ; Л - З.
		\item ИЗРЕ-КА на ИЗРЕДКА; Ю - Д.
		\item КУСТ- на КУСТЫ; Ш - Ы.
		\item МЕ-ДУ на МЕЖДУ; Ъ - Ж.
		\item КРУ-ОМ на КРУГОМ; Ь - Г.
		\item -ЫСТРО на БЫСТРО; Р - Б.
		\item НЕ- на НЕТ; Э - Т.
		\item ОТОВС-ДУ на ОТОВСЮДУ; В - Ю.
		\item УТИ-АЛО на УТИХАЛО; Б - Х.

	\end{enumerate}


\end{document}
