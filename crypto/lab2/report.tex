\documentclass[12pt, a4paper] {ncc}
\usepackage[utf8] {inputenc}
\usepackage[T2A]{fontenc}
\usepackage[english, russian] {babel}
\usepackage[usenames,dvipsnames]{xcolor}
\usepackage{listings,a4wide,longtable,amsmath,amsfonts,graphicx}
\usepackage{indentfirst}
\usepackage{bytefield}
\usepackage{multirow}
\usepackage{float}
\usepackage{caption}
\usepackage{subcaption}
\captionsetup{compatibility=false}
\usepackage{tabularx}

\usepackage[left=2cm,right=2cm,top=2cm,bottom=2cm,bindingoffset=0cm]{geometry}

\begin{document}
\setcounter{figure}{0}
\frenchspacing
\pagestyle{empty}
\begin{center}
                            Университет ИТМО    \\
                        Кафедра вычислительной техники

\vspace{\stretch{2}}
						Лабораторная работа № 2 \\
				по дисциплине: \\
				<<Методы и средства защиты компьютерной информации>>\\
Вариант: 5
\end{center}
\vspace{\stretch{2}}
\begin{center}

\end{center}
\vspace{\stretch{3}}
\begin{flushright}
                                    Студент:\\
                                    {\it Куклина Мария, гр. P3401}\\
                                    Преподаватель: \\
                                    {\it Ожиганов А.А}
\end{flushright}
\vspace{\stretch{4}}
\begin{center}
                             Санкт-Петербург, 2018
\end{center}
\newpage

\section{Цель работы}

	Используя индекс соответствия и частотный анализ, дешифровать криптограмму, зашифрованную
	шифром Вижинера.


\section{Расшифрованный исходный текст}

\textbf{Криптограмма}: \\
НКФЬВН  ШБТБТН ЫЯЮБВЧДХШРПСГЬЙЭУРРЧООЮЩЙМ ЙЙАЦМЕСЕНЮНЦАБАФБУСАСПШАЩЖ ФСПУЬАТ ГШПДУОСБДПТПСГАЬЮДООСБДЫЬАЕВЛТПЫЖ ДБТБЙАТ  ШДЩБДЪТЕПЦАУВМТ ЦУЬСНКТЯЖГЦАЙЖУМОФЕФРНГ ЯЕЦПЫЖ АОЬВВМАСНПЯЬТТЪЖТЬО  ВТЪРЯУЬСОЧППЬЛСВЙУЬСЕСВЙФЬТАЦЕЙ ЬРТТЛЙЭОАВЧСЕПЯГЕГ ЫХЮЕИЭСИПЫБ ЯЕЫЯЮБВЧДХШРПСГЬЙААЕЬУЫЙЭУДОЦОМРЩАНТ ЧХЯРРТВПФЩЙВ СЬЛНМЮЦЕУПЦАОЦНКЪЬАЖЧ ЧХННОХ ШБШБЗТТЕАМАОГ ЧЮРЬХСПШЯИУОЬ

\textbf{Открытый текст}: \\
НАДОБНО ОТДАТЬ СПРАВЕДЛИВОСТЬ НЕПРЕОДОЛИМОЙ СИЛЕ ЕГО ХАРАКТЕРА ПОСЛЕ ВСЕГО ТОГО ЧТО БЫ ДОСТАТОЧНО БЫЛО ЕСЛИ НЕ УБИТЬ ТО ОХЛАДИТЬ И УСМИРИТЬ НАВСЕГДА ЧЕЛОВЕКА В НЕМ НЕ ПОТУХЛА НЕПОСТИЖИМАЯ СТРАСТЬ ОН БЫЛ В ГОРЕ В ДОСАДЕ РОПТАЛ НА ВЕСЬ СВЕТ СЕРДИЛСЯ НА НЕСПРАВЕДЛИВОСТЬ СУДЬБЫ НЕГОДОВАЛ НА НЕСПРАВЕДЛИВОСТЬ ЛЮДЕЙ И ОДНАКО ЖЕ НЕ МОГ ОТКАЗАТЬСЯ ОТ НОВЫХ ПОПЫТОК

\section{Ключ}
АКРОБАТ

\section{Краткий протокол криптоанализа}
	Для поиска длины ключа был использован второй метод Фридмана, в котором максимальное среднее значение индекса соответствия
	(0.0632) сильно отрывается от остальных значений, поэтому принимаем, что длина ключа равна 7, что соответствует данному значению ИС.

	Далее пытаемся с таблицей статистики заменить наиболее вероятные символы пробелом. Первое слово получается "НАД БАО" и ключ "АКРЮБПТ".
	Ключ очень напоминает слово "АКРОБАТ". При замене букв таким образом, чтобы получился данный ключ, получился вменяемый текст.


\end{document}
