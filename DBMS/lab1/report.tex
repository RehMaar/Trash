\documentclass[12pt, a4paper] {ncc}
\usepackage[utf8] {inputenc}
\usepackage[T2A]{fontenc}
\usepackage[english, russian] {babel}
\usepackage[usenames,dvipsnames]{xcolor}
\usepackage{listings,a4wide,longtable,amsmath,amsfonts,graphicx,tikz}
\usepackage{pgfplots}
\usepackage{indentfirst}
\usepackage{bytefield}
\usepackage{multirow}
\usepackage{tabularx}


\lstdefinestyle{SQLStyle}{
    basicstyle=\footnotesize\ttfamily,
    language={SQL},
    keywordstyle=\bfseries,
    showstringspaces=false,
    commentstyle={},
    escapebegin=\begin{russian}\commentfont,
    escapeend=\end{russian},
}
\newcommand{\commentfont}{\ttfamily}

\begin{document}
\frenchspacing
\pagestyle{empty}
\begin{center}
     Национальный исследовательский университет информационных технологий,
                              механики и оптики\\
                        Кафедра вычислительной техники\\
					Системы управления базами данных
\end{center}
\vspace{\stretch{2}}
\begin{center}
                            Лабораторная работа №1\\
\end{center}
\vspace{\stretch{3}}
\vspace{\stretch{4}}
\begin{center}
                             Санкт-Петербург, 2017
\end{center}
\newpage


\section{Цели работы}
    Используя сведения из представлений словаря данных получить информацию о любой таблице: 
	Номер по порядку, Имя столбца, Атрибуты (в атрибуты столбца включить тип данных, ограничение типа CHECK).

\section{Листинг}
	
\lstinputlisting[style=SQLStyle,inputencoding=cp1251]{./l.sql}

\section*{Вывод}

В ходе выполнения лабораторной работы было проведено изучение словаря данных Oracle DB. Оказалось, что
словарь данных является описанием БД и содержит именя и атрибуты всех объектов базы; представляет собой
набор таблиц и представлений. 

\end{document}
