\documentclass[a4paper,10pt]{article}
\usepackage[utf8]{inputenc}
\usepackage[english,russian]{babel}
\usepackage{fancyhdr}
\usepackage{caption}

\usepackage{listings,longtable,amsmath,amsfonts,graphicx,tikz,tabularx,pgf}
\usepackage{circuitikz}
\usetikzlibrary{arrows,automata}

\captionsetup{labelsep=period}
\pagestyle{fancy}


\lstset{
    basicstyle=\footnotesize,
    breakatwhitespace=false,
    breaklines=true,
    extendedchars=true,
    keepspaces=true,
    keywordstyle=\bfseries,
    numbers=left,
    numbersep=3pt,
    numberstyle=\tiny,
    showspaces=false,
    showstringspaces=false,
    showtabs=false,
    stepnumber=1,
    stringstyle=\emph,
    tabsize=2
}

\usepackage[left=1.5cm,right=1.5cm,top=2cm,bottom=1.5cm,bindingoffset=0cm]{geometry}

\captionsetup{labelsep=period}
\pagestyle{fancy}

\renewcommand{\headrulewidth}{0pt}
\fancyfoot[L] {\thepage\bf}
\fancyfoot[C] {}

\graphicspath{ {img/} }


% Flip-flop symbol
\pgfdeclareshape{flipflop}{
\anchor{center}{\pgfpointorigin} % within the node, (0,0) is the center
\anchor{text} % this is used to center the text in the node
{\pgfpoint{-.5\wd\pgfnodeparttextbox}{-.5\ht\pgfnodeparttextbox}}
    %\savedanchor\icpina{\pgfpoint{-.75cm}{-.625cm}} % pin 1
    \savedanchor\dffind{\pgfpoint{-.7cm}{.7cm}} % Input D
    \anchor{in}{\dffind}
    \savedanchor\dffoutq{\pgfpoint{.7cm}{.7cm}} % Output Q
    \anchor{out1}{\dffoutq}
    \savedanchor\dffoutnq{\pgfpoint{.7cm}{-.7cm}} % Output ~Q
    \anchor{out2}{\dffoutnq}

    \foregroundpath{ % border and pin numbers are drawn here
        \pgfsetlinewidth{0.03cm}
        \pgfpathrectanglecorners{\pgfpoint{.7cm}{1cm}}{\pgfpoint{-.7cm}{-1cm}}
        \pgfusepath{draw} %draw rectangle
        \pgfsetlinewidth{0.03cm}
        %\pgfpathmoveto{\pgfpoint{-1cm}{-.3cm}}
        %\pgfpatharc{-90}{90}{.3cm}
        %\pgfusepath{draw} %draw semicircle
        %\pgftext[bottom,at={\pgfpoint{-.75cm}{-.55cm}}]{\scriptsize 1}
        %\pgftext[top,at={\pgfpoint{.75cm}{.55cm}}]{\scriptsize 5}
		\pgftext[top,at={\pgfpoint{-.5cm}{.5cm}}]{}
		\pgftext[top,at={\pgfpoint{.5cm}{.7cm}}]{\scriptsize $Q_0$}
		\pgftext[top,at={\pgfpoint{.5cm}{-.6cm}}]{\scriptsize $\bar{Q_0}$}
	}
}


\begin{document}
    \begin{titlepage}
       \begin{center}
            \large
            САНКТ-ПЕТЕРБУРГСКИЙ НАЦИОНАЛЬНЫЙ ИССЛЕДОВАТЕЛЬСКИЙ УНИВЕРСИТЕТ ИНФОРМАЦИОННЫХ ТЕХНОЛОГИЙ, МЕХАНИКИ И ОПТИКИ \\


            \vspace{3cm}


            Кафедра вычислительной техники
            \vspace{4cm}

            \textsc{ \textbf{Отчёт по лабораторной работе  № 3} \\
            по дисциплине <<Теория автоматов>>\\}
            Вариант №4\\[8mm]

            \bigskip
        \end{center}
        \vspace{3cm}

        \hfill\begin{flushright}
             Студент: \\ Куклина М. \\ P3301 \\ 
             \vfill
             Преподаватель:\\ Ожиганов А.А.
        \end{flushright}
        \vfill
        \vfill
        \vfill
        \vfill
        \vfill
        \begin{center}
            Санкт-Петербург \\2017 г.
        \end{center}
    \end{titlepage}
\newpage

\section*{Цель и постановка задачи}
    \subsection*{Цель}
		Освоение метода перехода от абстрактного автомата к структурному автомату.

    \subsection*{Постановка задачи}
        Абстрактный автомат задан табличным способом. Причем абстрактный автомат 
        Мили представлен таблицами переходов и выходов, а абстрактный автомат Мура - 
        одной отмеченной таблицей переходов. Для синтеза структурного автомата 
        использовать функционально полную систему логических элементов И, ИЛИ, НЕ и 
        автомат Мура, обладающий полнотой переходов и полнотой выходов. 
        Синтезированный структурный автомат представить в виде ПАМЯТИ и 
        КОМБИНАЦИОННОЙ СХЕМЫ. 

\section*{Исходный абстрактный автомат}
	\begin{table}[h!]
		\center
		\begin{tabular}{|c|c|c|c|c|}
			\hline
			$\delta$ & $a_1$ & $a_2$ & $a_3$ & $a_4$ \\ \hline 
			 $z_1$	 & $a_2$ & $a_2$ & $a_4$ & $a_2$ \\ \hline 
			 $z_2$   & $a_3$ & $a_1$ & $a_2$ & $a_3$ \\ \hline
		\end{tabular}
		\caption{Функция переходов}
    \end{table}
	\begin{table}[h!]
		\center
		\begin{tabular}{|c|c|c|c|c|}
			\hline
			$\lambda$ & $a_1$ & $a_2$ & $a_3$ & $a_4$ \\ \hline 
			 $z_1$	  & $w_2$ & $w_2$ & $w_1$ & $w_1$ \\ \hline 
			 $z_2$    & $w_2$ & $w_1$ & $w_2$ & $w_2$ \\ \hline
		\end{tabular}
		\caption{Функция выходов}
    \end{table}

\section*{Граф исходного автомата}
    \begin{tikzpicture}[->,>=stealth',shorten >=2pt,auto,node distance=4cm, semithick] 
        \tikzstyle{every state}=[draw=black,text=black,fill=white,thick,scale=1]
        \foreach \s in {1,3} {
                \node[state] (a\s) at ({360/4 *(\s+1)}:0.25\textwidth) {$a_\s$};
        }
        \node[state] (a2) at ({360/4 *(4+1)}:0.25\textwidth) {$a_2$};
        \node[state] (a4) at ({360/4 *(2+1)}:0.25\textwidth) {$a_4$};

		\path (a1) edge[bend left]    node {$z_1/w_2$} (a2)
		      (a1) edge               node[pos=0.8] {$z_2/w_2$} (a3)
		      (a2) edge[loop]         node {$z_1/w_2$} (a2)
		      (a2) edge[bend left]    node {$z_2/w_1$} (a1)
		      (a3) edge[bend left]    node {$z_1/w_1$} (a4)
		      (a3) edge               node {$z_2/w_2$} (a2)
		      (a4) edge               node[pos=0.7,right] {$z_1/w_2$} (a2)
		      (a4) edge[bend left]    node {$z_2/w_2$} (a3);
	\end{tikzpicture}

\section*{Переход к структурному автомату}
	\subsection*{Кодирование абстрактного автомата}
		\begin{table}[h!]
			\center
			\begin{tabular}{|c|c|}
				\hline
				  $\delta$ & $x_0$ \\ \hline
					$z_1$  &   0   \\ \hline
					$z_2$  &   1   \\ \hline
			\end{tabular}
			\caption{Кодирование входов автомата}
		\end{table}
		\begin{table}[h!]
			\center
			\begin{tabular}{|c|c|}
				\hline
				  $\delta$ & $y_0$ \\ \hline
					$w_2$  &   0   \\ \hline
					$w_1$  &   1   \\ \hline
			\end{tabular}
			\caption{Кодирование выходов автомата}
		\end{table}

		\begin{table}[h!]
			\center
			\begin{tabular}{|c|c|c|}
				\hline
				  $\lambda$ & $Q_0$ & $Q_1$ \\ \hline
					$a_2$  &   0   &   0   \\ \hline
					$a_3$  &   0   &   1   \\ \hline
					$a_1$  &   1   &   0   \\ \hline
					$a_4$  &   1   &   1   \\ \hline
			\end{tabular}
			\caption{Кодирование состояний автомата}
		\end{table}

		Получившийся структурный автомат имеет один вход, один выход и четыре состояния. \\

		\begin{tikzpicture}
			\node (a) {$x_0$};
			\node[draw,rectangle,right of=a,text depth=1cm] (s) {$CK$};
			\node (c)[right of=s] {$y_0$};

			\draw[->] (a) -- (s);
			\draw[->] (s) -- (c);
		\end{tikzpicture}

	\subsection*{Структурный автомат}
		\begin{table}[h!]
		\center
		\begin{tabular}{|c|c|c|c|c|}
			\hline
    		 $Q_0Q_1$ & 00 & 01 & 10 & 11 \\ 
			 $x_0$    &    &	&	 &    \\ \hline
    			 0	  & 00 & 11 & 00 & 00 \\ \hline 
    			 1    & 10 & 00 & 01 & 01 \\ \hline
		\end{tabular}
		\caption{Функция переходов}
    \end{table}
	Функция переходов автомата: $Q_0Q_1 = \delta(Q_0,Q_1,x_0)$.

	\begin{table}[h!]
		\center
		\begin{tabular}{|c|c|c|c|c|}
			\hline
			  $Q_0Q_1$& 00 & 01 & 10 & 11 \\ 
			  $x_0$   &    &    &    &    \\ \hline
	      	     0    &  0 &  1 & 0  & 1  \\ \hline 
		         1    &  1 &  0 & 0  & 0  \\ \hline
		\end{tabular}
		\caption{Функция выходов}
    \end{table}	
	Функция выходов автомата: $y_0 = \lambda(Q_0,Q_1,x_0)$. \\

	По таблице выходов строим ДНФ:
	$y_0 = \bar{Q_0} \bar{Q_1} x_0 \lor \bar{Q_0} Q_1 \bar{x_0} \lor Q_0 Q_1 \bar{x_0}$

    \subsection*{Сигналы функции возбуждения для триггеров}
		\subsubsection*{D-триггер}
    		\begin{table}[h!]
				\center
    			\begin{tabular}{|c|c|c|}
					\hline
					 Q & 0 & 1 \\
					 x &   &   \\ \hline
					 0 & 0 & 0 \\ \hline
					 1 & 1 & 1 \\ \hline
				\end{tabular}
    			\caption{Закон функционирования D-триггера}
    		\end{table}

			На основе закона функционирования D-триггера по таблице переходов структурного автомата
			строим таблицу сигналов функции возбуждения. \\

    		\begin{table}[h!]
				\center
            		\begin{tabular}{|c|c|c|c|c|}
            			\hline
                		 $Q_0Q_1$ & 00       & 01       & 10       & 11 \\ 
            			 $x_0$    &          &	        &	       &    \\ \hline
                			 0	  & 00       & 11       & 00       & 00 \\ \hline 
                			 1    & 10       & 00       & 01       & 01 \\ \hline
								  & $D_0D_1$ & $D_0D_1$ & $D_0D_1$ & $D_0D_1$ \\ \hline
            		\end{tabular}
    			\caption{Таблица сигналов функции возбуждения: $D_0D_1 = \mu(Q_0,Q_1,x_0)$}
    		\end{table}

        	ДНФ для сигналов функции возбуждения: \\
				$D_0 = \bar{Q_0} \bar{Q_1} x_0 \lor \bar{Q_0} Q_1 \bar{x_0}$ \\
				$D_1 = \bar{Q_0} Q_1 \bar{x_0} \lor Q_0 \bar{Q_1} x_0 \lor Q_0 Q_1 x_0$.

		Для построения функциональной схемы рассмотрим ДНФ: \\
    	$y_0 = \bar{Q_0} \bar{Q_1} x_0 \lor \bar{Q_0} Q_1 \bar{x_0} \lor Q_0 Q_1 \bar{x_0}$ \\
		$D_0 = \bar{Q_0} \bar{Q_1} x_0 \lor \bar{Q_0} Q_1 \bar{x_0}$ \\
		$D_1 = \bar{Q_0} Q_1 \bar{x_0} \lor Q_0 \bar{Q_1} x_0 \lor Q_0 Q_1 x_0$.

		Или: \\
    	$y_0 = D_0 \lor Q_0 \phi$ \\
		$D_0 = \bar{Q_0} ( Q_1 \oplus x_0)$ \\
		$D_1 = \bar{Q_0} \phi \lor Q_0  x_0 $. \\
		$\phi = Q_1 \bar{x_0}$

		\begin{figure}[h!]
			\begin{circuitikz}
				\draw
                    % (0,2) node[and port] (myand1) {}
                    % (0,0) node[and port] (myand2) {}
                    % (2,1) node[xnor port] (myxnor) {}
					(0,4) node[and port] (and1) {1}
					(0,0) node[xor port] (xor5) {5}

					(2,7) node[and port] (and2) {2}
					(2,5) node[and port] (and3) {3}
					(2,3) node[and port] (and4) {4}
					(2,1) node[and port] (and8) {8}

					(4,6) node[or port]   (or6) {6}
					(4,3) node[or port]   (or7) {7}

					(6,1) node[flipflop] (dff2) {$DFF$}
					(6,6) node[flipflop] (dff1) {$DFF$}
					(8,2) node			  (out8) {}
					(8,7) node			  (out6) {}
					(8,3) node			  (out7) {}

					 (and1.in 1) node[anchor=east] {$Q_1$}
					 (and1.in 2) node[anchor=east] {$\bar{x_0}$}
					 (xor5.in 1) node[anchor=east] {$Q_1$}
					 (xor5.in 2) node[anchor=east] {$x_0$}
					 (and2.in 1) node[anchor=east] {$Q_0$}
					 (and2.in 2) node[anchor=east] {$x_0$}
					 (and3.in 1) node[anchor=east] {$\bar{Q_0}$}
					 (and1.out) -| (and3.in 2)
					 (and1.out) -| (and4.in 1)
					 (and4.in 2) node[anchor=east] {$Q_0$}
					 (and8.in 1) node[anchor=east] {$\bar{Q_0}$}
					 (xor5.out) -| (and8.in 2)
					 (and2.out) -| (or6.in 1)
					 (and3.out) -| (or6.in 2)
					 (and4.out) -| (or7.in 1)
					 (and8.out) -| (or7.in 2)
					 %(or7.out) node[anchor=south] {$y_0$}
					 %(or6.out) node[anchor=south] {$Q_1$}
					 (or7.out) -| (out7)
					 (out7) node[anchor=south] {$y_0$}

					 (and8.out) -| (dff2.in)%
					 (dff2.out1) -| (out8)
					 (out8) node[anchor=south] {$Q_0$}

					 (or6.out) -| (dff1.in)
					 (dff1.out1) -| (out6)
					 (out6) node[anchor=south] {$Q_1$}
					;
			\end{circuitikz}
		\end{figure}


		\subsubsection*{T-триггер}
    		\begin{table}[h!]
				\center
    			\begin{tabular}{|c|c|c|}
					\hline
					 Q & 0 & 1 \\
					 x &   &   \\ \hline
					 0 & 0 & 1 \\ \hline
					 1 & 1 & 0 \\ \hline
				\end{tabular}
    			\caption{Закон функционирования T-триггера}
    		\end{table}

			На основе закона функционирования T-триггера по таблице переходов структурного автомата
			строим таблицу сигналов функции возбуждения. \\

    		\begin{table}[h!]
				\center
        		\begin{tabular}{|c|c|c|c|c|}
        			\hline
            		 $Q_0Q_1$ & 00       & 01       & 10       & 11 \\ 
        			 $x_0$    &          &	        &	       &    \\ \hline
            			 0	  & 00       & 10       & 10       & 11 \\ \hline 
            			 1    & 10       & 01       & 11       & 10 \\ \hline
							  & $T_0T_1$ & $T_0T_1$ & $T_0T_1$ & $T_0T_1$ \\ \hline
        		\end{tabular}
    			\caption{Таблица сигналов функции возбуждения: $T_0T_1 = \mu(Q_0,Q_1,x_0)$}
    		\end{table}

			ДНФ для сигналов функции возбуждения: \\
			$T_0 = \bar{Q_0} \bar{Q_1} x_0 \lor \bar{Q_0} Q_1 \bar{x_0} \lor Q_0 \bar{Q_1} \bar{x_0} \lor Q_0 \bar{Q_1} x_0 \lor Q_0 Q_1 \bar{x_0} \lor Q_0 Q_1 x_0 $ \\
			$T_1 = \bar{Q_0} Q_1 x_0 \lor Q_0 \bar{Q_1} x_0 \lor Q_0 Q_1 \bar{x_0}$.

			Или: \\
			$T_0 = \bar{Q_0} (Q_1 \oplus x_0)\lor Q_0$ \\
			$T_1 = \bar{Q_0} Q_1 x_0 \lor Q_0 (Q_1 \oplus x_0)$

	 		Рассмотрим выведенные ДНФ: \\
        	$y_0 = \bar{Q_0} \bar{Q_1} x_0 \lor \bar{Q_0} Q_1 \bar{x_0} \lor Q_0 Q_1 \bar{x_0}$ \\
			$T_0 = \bar{Q_0} \bar{Q_1} x_0 \lor \bar{Q_0} Q_1 \bar{x_0} \lor Q_0 \bar{Q_1} \bar{x_0} \lor Q_0 \bar{Q_1} x_0 \lor Q_0 Q_1 \bar{x_0} \lor Q_0 Q_1 x_0 $ \\
			$T_1 = \bar{Q_0} Q_1 x_0 \lor Q_0 \bar{Q_1} x_0 \lor Q_0 Q_1 \bar{x_0}$. \\

\section*{Вывод}
\end{document}
